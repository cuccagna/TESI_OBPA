%!TEX encoding = UTF-8 Unicode
%!TEX TS-program = xelatex
% !TEX root = main.tex

\documentclass[12pt, oneside]{book}

\usepackage{ifxetex}

\ifxetex
  \usepackage{fontspec}
  \usepackage{polyglossia}
  \setmainlanguage{italian}
\else
  \usepackage[T1]{fontenc}
  \usepackage[utf8]{inputenc}
  \usepackage[italian]{babel} 
  %\usepackage{lmodern}
\fi


%style=numeric-comp
\usepackage{csquotes}
\usepackage[backend=bibtex,style=numeric-comp, backref=true,doi=true,url=true]{biblatex}
%Le seguenti 3 istruzioni sono necessarie  per gestire un bug che altrimenti
%impedirebbe di generare la bibliografia correttamente in textlive16 in
%caso di url lunghi.
\makeatletter
\def\blx@maxline{77}
\makeatother
\bibliography{./chapters/bibliography.bib}

%\usepackage{microtype}

% Math package
\usepackage{amsmath}
\usepackage{amssymb}
\usepackage{amsthm}
\usepackage{mathtools}
\usepackage{cases}
\usepackage{upgreek}
\usepackage{braket}
\usepackage{enumitem}
\usepackage{lastpage}
\usepackage{listings}
\usepackage{algorithm}
\usepackage[noend]{algpseudocode}
\usepackage[]{float}
\usepackage[compatibility=false]{caption}
\usepackage{subfig}
\usepackage{verbatim}

\usepackage{array}
\usepackage{makeidx}
\makeindex %per creare indice analitico

\DeclarePairedDelimiter{\abs}{\lvert}{\rvert}
\DeclarePairedDelimiter{\norma}{\lVert}{\rVert}
\DeclarePairedDelimiter\ceil{\lceil}{\rceil}

\makeatletter
\renewcommand{\ALG@name}{Algoritmo}
\makeatother
\DeclareCaptionFormat{ruled}{\leavevmode\leaders\hrule height 0.8pt depth0pt\hfill\mbox{}\endgraf#1#2 #3 \vspace{-0.7\baselineskip}\leavevmode\leaders\hrule height 0.6pt\hfill\null\vspace*{-0.8\baselineskip}}
%ACRONIMI
\usepackage{acronym}%\usepackage[nohyperlinks]{acronym} senza link
%\renewcommand*\acfsfont{\textbf}
\renewcommand*{\acsfont}[1]{\textit{#1}}%stile della sigla dell'acronimo
\renewcommand*{\acffont}[1]{\textbf{#1}} %stile del nome  dell'acronimo la prima volta che compare


% Redefine margins
\usepackage[paper=a4paper,includehead,includefoot,margin=1in]{geometry}

\usepackage{graphicx}
\graphicspath{{pictures/}}
\DeclareGraphicsExtensions{.pdf,.jpg,.png}


\usepackage[table]{xcolor}% http://ctan.org/pkg/xcolor

\usepackage{array,booktabs,arydshln,multirow}

\usepackage{tikz} %for automata
\usetikzlibrary{automata,positioning}

\usepackage{pgfplots}

% Fancy headers
\usepackage{fancyhdr}
 \let\cleardoublepage\clearpage  %per eliminare le pagine bianche. 

\setlength{\parindent}{0pt}
\setlength{\parsep}{10pt}


%\newfontfamily\quotefont{Arial}
\newcommand{\apostrophe}{{\quotefont'}}

\newenvironment{rientra} %definisco un ambiente personalizzato da usare per avere un rientro
  {\par\setlength{\leftskip}{0.7cm}\setlength{\rightskip}{0cm}\noindent\ignorespaces}

\theoremstyle{definition}
\newtheorem*{definizione*}{Definizione}
\newtheorem{definizione}{Definizione}[chapter]
%\setcounter{definizione}{1} %per fare partire il numero della definizione da dove vuoi tu

%\theoremstyle{definition}
%\newtheorem*{ nonnumeratadefinizione }{ Definizione }

\theoremstyle{plain}
\newtheorem*{teorema*}{Teorema} %per avere teoremi non numerati
\newtheorem{teorema}{Teorema}[chapter]
\newtheorem{lemma}{Lemma} %\newtheorem{lemma}[teorema]{Lemma} lemma che segue la numerazione dei teoremi
\newtheorem{corollary}[teorema]{Corollary}

\newcolumntype{P}[1]{>{\centering\arraybackslash}p{#1}} %allineamento orizzontale centrale di p (nota che P è maiuscolo) per una cella di tabella.
\newcolumntype{M}[1]{>{\centering\arraybackslash}m{#1}} %allineamento verticale  oltre che orizzontale .
\newcolumntype{N}{@{}m{0pt}@{}}

\algnewcommand\algorithmicinput{\textbf{Input:}}
\algnewcommand\Input{\item[\algorithmicinput]}
\algnewcommand\algorithmicoutput{\textbf{Output:}}
\algnewcommand\Output{\item[\algorithmicoutput]}
\algnewcommand{\LineComment}[1]{\Statex \(\triangleright\) #1}


\usepackage{hyperref}
 \hypersetup{colorlinks,
           citecolor=cyan,  
            filecolor=black,  
            linkcolor=black,  
            urlcolor=cyan } 
%-----------------------------------------------------------------------------------------------
%	INIZIO DOCUMENTO
%----------------------------------------------------------------------------------------
\begin{document}
\pagestyle{empty}
%----------------------------------------------------------------------------------------
%	PAGINA TITOLO
%----------------------------------------------------------------------------------------
\title{STUDIO ED IMPLEMENTAZIONE DELL'OBSERVATION PACK ALGORITHM}
\author{NICOLA CIACO}
\date{\today}
%\maketitle

\begin{titlepage}
\begin{center}
\begin{figure}[!h]
  	\centering
 	\includegraphics[width=2.5cm]{./pictures/frontespizio/logo_unipa.png}
\end{figure}
{{\Large{\textsc{Università degli Studi di Palermo}}}} \\
\rule[0.1cm]{15.8cm}{0.1mm}
\rule[0.5cm]{15.8cm}{0.6mm}
{\small{\bf SCUOLA POLITECNICA\\
Corso di Laurea Magistrale in Ingegneria Informatica\\
Dipartimento di Innovazione Industriale e Digitale \\Ingegneria Chimica, Gestionale, Informatica e Meccanica}}\\
\vspace{10mm}
\end{center}
\vspace{8mm}
\begin{center}
%\uppercase{Estrazione di conoscenza mediante algoritmi di apprendimento attivo per %l'inferenza regolare}
%SPERIMENTAZIONE DI ALGORITMI DI ACTIVE LEARNING NELL'INFERENZA INDUTTIVA REGOLARE
{\Large{\bf ESTRAZIONE DI CONOSCENZA MEDIANTE}}\\
\vspace{3mm}
{\Large{\bf  ALGORITMI DI APPRENDIMENTO ATTIVO }}\\
 \vspace{3mm}
 {\Large{\bf PER L'INFERENZA REGOLARE}}\\
\end{center}
\vspace{17mm}
\par
\noindent
\begin{minipage}[t]{0.70\textwidth}\raggedright
{\large Tesi di laurea di:\\
\vspace{3mm}
\textbf {Nicola Ciaco}\\
\vspace{10mm}
Matricola:\\
\vspace{3mm}
\textbf {0582164}\\

}
\end{minipage}
\begin{minipage}[t]{0.47\textwidth}\raggedright
{\large Relatore:\\
\vspace{3mm}
\textbf{Chiar.mo Prof.}\\ \textbf{Salvatore Gaglio}\\
\vspace{5mm}
Correlatori:\\
\vspace{3mm}
\textbf{Ing. Marco Ortolani}\\
\vspace{3mm}
\textbf{Ing. Pietro Cottone}\\
\vspace{5mm}
Controrelatore:\\
\vspace{3mm}
\textbf{Chiar.mo Prof.}\\ \textbf{Giuseppe Lo Re}
}
\end{minipage}
\hfill
\vspace{6mm}
\begin{center}
{\large Anno Accademico\\2017/18 }%inserire l'anno accademico a cui si è iscritti
\end{center}
\end{titlepage}

\newgeometry{top=2.5cm, bottom=2.5cm, left=2.5cm, right=2.5cm, bindingoffset=10mm,includehead,includefoot}

\frontmatter % Use roman page numbering style (i, ii, iii, iv...) for the pre-content pages

%%%%%%%%
% Sommario %
%%%%%%%%
\cleardoublepage
\phantomsection
\addcontentsline{toc}{chapter}{Sommario}
%!TEX encoding = UTF-8 Unicode
%!TEX root = ./../main.tex
%!TEX TS-program = xelatex

\begin{center}
\begin{figure}[!h]
  	\centering
 	\includegraphics[width=1.5cm]{../pictures/frontespizio/logo_unipa_piccolo.png}
\end{figure}
\textsc{\textbf{UNIVERSIT\`A DEGLI STUDI DI PALERMO}} \\
SCUOLA POLITECNICA\\
\small{Laurea Magistrale in Ingegneria Informatica}\\
\end{center}
\begin{center}
\uppercase{Estrazione di conoscenza mediante algoritmi di apprendimento attivo per l'inferenza regolare}
%SPERIMENTAZIONE DI ALGORITMI DI ACTIVE LEARNING NELL'INFERENZA INDUTTIVA REGOLARE
\end{center}

\begin{minipage}[t]{0.65\textwidth}\raggedright
{ \scriptsize{TESI DI LAUREA DI:}\\
\scriptsize{\textbf{Dott. Nicola Ciaco}}
}
\end{minipage}
\begin{minipage}[t]{0.47\textwidth}\raggedright
{\scriptsize{RELATORE:}\\
\vspace{0.7mm}
\scriptsize{\textbf{Chiar.mo Prof. Salvatore Gaglio}}\\
\vspace{0.7mm}
\scriptsize{CORRELATORI:}\\
\vspace{0.7mm}
\scriptsize{\textbf{Ing. M. Ortolani, Ing. P. Cottone}}\\
%\scriptsize{\textbf{Ing. Pietro Cottone}}
%\vspace{3mm}
}
\end{minipage}
\begin{center}
{\small Anno Accademico 2017/18 \\}%inserire l'anno accademico a cui si è iscritti
\vspace{3mm}
\small{\textbf{Sommario}}\\
\end{center}


\label{cap:sommario}
Un sistema intelligente è caratterizzato dalla capacità di apprendere in modo automatico, la quale a sua volta presuppone la capacità sia di rappresentare la conoscenza nota a priori sia di \textit{inferirne} di nuova. Allo stato dell'arte attuale, le tecniche più efficaci per l'\textit{estrazione di conoscenza} sono state sviluppate nell'ambito di quella che è nota come Statistical Learning Theory (SLT). Per quanto tali metodi forniscano risultati apprezzabili in termini di accuratezza, essi presentano tuttavia anche dei non trascurabili svantaggi, quali ad esempio il fatto che il loro funzionamento si basa sull'inferire dei parametri ottimali per un modello che è visto necessariamente come una \textit{black-box}, rendendo ardua o impossibilitando di fatto l'\textit{interpretazione} dei campioni a partire dai quali il modello è stato costruito.

La presente tesi si colloca invece nel solco di un framework teorico alternativo, noto come Algorithmic Learning Theory (ALT), secondo cui i dati non sono considerati come campioni casuali da mappare in uno spazio vettoriale, bensì come specifiche istanze del modello nascosto che è oggetto di inferenza. In particolare, lo studio si è concentrato su una tecnica di estrazione di conoscenza che va sotto il nome di \textit{Inferenza Grammaticale} (GI), un processo di apprendimento che si basa sull'\textit{induzione} nel contesto dei linguaggi formali e del relativo formalismo grammaticale. In tale contesto, l'obiettivo è selezionare il migliore modello rappresentato mediante il tipico riconoscitore di una grammatica formale, ovvero l'automa a stati finiti, consistente con i campioni iniziali, che a loro volta sono interpretabili come stringhe generabili dalla grammatica. Nella fattispecie, la classe dei linguaggi oggetto di studio per la presente tesi è quella dei linguaggi regolari e si parla pertanto più specificamente di \textit{Inferenza Induttiva Regolare} (IIR), i cui limiti teorici sono stati messi in luce dal lavoro di Gold \cite{Gold67}. Gli algoritmi della letteratura iniziale sull'argomento riconducevano l'apprendimento ad una ricerca euristica in un spazio rappresentato come un grafo contenente gli automi consistenti con i campioni forniti in cui il modello, inizialmente iperspecializzato, viene progressivamente generalizzato. Tale approccio va sotto il nome di \textit{passive learning} ed è caratterizzato da un'inevitabile esplosione combinatoria che ne rende la complessità ingestibile a meno di non sacrificare le garanzie teoriche di terminazione e ottimalità.

Nella presente tesi si è quindi scelto di seguire un approccio proposto nella letteratura più recente, duale rispetto al precedente, secondo cui un modello, inizialmente molto generale e quindi poco accurato, viene progressivamente specializzato per rappresentare con precisione i dati forniti. Questo paradigma, noto come \textit{active learning}, presuppone l'esistenza di un \textit{oracolo} che guida l'apprendimento rispondendo ad alcuni tipi di query sottopostegli attivamente dal sistema che cerca di inferire il modello. La base teorica del paradigma è stata fornita dagli studi di Angluin \cite{Angluin87} che stabiliscono che l'oracolo debba appartenere alla classe dei cosiddetti Minimally Adequate Teacher (MAT) che, per garantire l'abilità di fornire risposte utili ai principali tipi di query considerate, richiede la conoscenza preliminare dell'automa oggetto d'inferenza.
Questo è un requisito molto forte e fin tropo stringente, che limita l'utilizzo in contesti reali dell'inferenza induttiva regolare declinata nell'accezione dell'\textit{active learning}.
 

Il cuore del lavoro svolto in questa tesi ha quindi riguardato la sostituzione di un tradizionale oracolo con una sua approssimazione mediante un classificatore statistico costruito a partire da esempi positivi e negativi del linguaggio target nell'ottica di permetterne un utilizzo nelle applicazioni reali. L'obiettivo è quello di tracciare un parallelo tra i modelli ottenibili seguendo il paradigma della teoria dell'apprendimento statistico e quelli formulabili algoritmicamente e di verificare sperimentalmente se l'uso combinato possa condurre al superamento dei limiti di entrambi. In altre parole, si vuole costruire per i campioni dati un modello strutturale, da usare in luogo dell'oracolo in modo da superarne i requisiti stringenti.  Il modello statistico prescelto per affrontare il delineato problema di classificazione binaria è stato quello delle Support Vector Machines (SVM) perchè rappresentano un modello molto potente. Inoltre malgrado siano state proposte in letteratura delle loro applicazioni nel contesto dei linguaggi formali esse non sono state contestualizzate nell’ambito dell’inferenza induttiva regolare mediante \textit{active learning} oppure erano limitate all'apprendimento di specifiche classi di linguaggi sottoinsiemi di quelli regolari.


Al fine di valutare la correttezza della soluzione ottenuta, durante il lavoro di tesi si è progettata un'implementazione in linguaggio C++11, integrando il codice nella preesistente libreria di passive learning Gi-learning \cite{Cot16}. L'implementazione è stata testata sperimentalmente inizialmente su una nota classe di automi che rappresentano linguaggi semplici, ovvero i Tomita \cite{Tomita82, Dupont94}, ottenendo ottimi risultati paragonabili al caso ideale.
Infine le performance e la precisione dei modelli ottenuti dall'approccio qui proposto sono state vagliate su data set di automi estratti casualmente, la cui complessità è paragonabile ai sistemi da apprendere nei casi pratici. I risultati ottenuti rivelano che all'aumentare della complessità del target si ha un progressivo decadimento del grado di approssimazione ed è quindi possibile concludere che un livello di fiducia alto sull'utilizzo dell'algoritmo è ottenibile solo fino ad un certo livello di complessità del target.




%----------------------------------------------------------------------------------------
%	INDICE
%----------------------------------------------------------------------------------------
\tableofcontents
\listoffigures
\listoftables


%----------------------------------------------------------------------------------------
%	INTRODUZIONE & RING
%----------------------------------------------------------------------------------------


%----------------------------------------------------------------------------------------
%	STILE PAGINA
%----------------------------------------------------------------------------------------


%\newcommand{\HRule}{\rule{\linewidth}{0.5mm}} % New command to make the lines in the title page

% PDF meta-data
%\hypersetup{pdftitle={CIAO}}
%\hypersetup{pdfsubject=\subjectname}
%\hypersetup{pdfauthor=\authornames}
%\hypersetup{pdfkeywords=\keywordnames}}



%----------------------------------------------------------------------------------------
%	THESIS CONTENT - CHAPTERS
%----------------------------------------------------------------------------------------

%%%%%%%%%
% ACRONYMS %
%%%%%%%%%
\cleardoublepage
\phantomsection
\addcontentsline{toc}{chapter}{Acronimi}
% ACRONYM %
%!TEX encoding = UTF-8 Unicode
%!TEX root = ./../main.tex
%!TEX TS-program = xelatex
%
%
\chapter*{Acronimi}
\begin{acronym}[WYSIWYM]
\acro{ObP}[ObP]{Observation Pack}
\acro{IIR}[IIR]{Inferenza Induttiva Regolare}
\acro{GI}[GI]{Inferenza Grammaticale}
\acro{P}[P]{enunciati premessa}
\acro{BK}[BK]{conoscenze di background}
\acro{H}[H]{ipotesi induttiva}
\acro{MQ}[MQ]{Membership Query}
\acro{EQ}[EQ]{Equivalence Query}
\acro{L}[$\mathcal{L}$]{linguaggio target}
\acro{MAT}[MAT]{Minimally Adeguate Teacher}
\acro{FSA}[FSA]{Finite State Automata}
\acro{DFA}[DFA]{Deterministic Finite Automata}
\acro{FSM}[FSM]{Finite State Machine}
\acro{NFA}[NFA]{Non-Deterministic Finite Automata}
\acro{DT}[DT]{Discrimination Tree}
\acro{LCA}[LCA]{least common ancestor}
\acro{SVM}[SVM]{Support Vector Machine}
\end{acronym}



%%%%%%%%%%%
% INTRODUZIONE %
%%%%%%%%%%%
\cleardoublepage
\phantomsection
\addcontentsline{toc}{chapter}{Introduzione}
%!TEX encoding = UTF-8 Unicode
%!TEX root = ./../main.tex
%!TEX TS-program = xelatex

\chapter*{Introduzione} % Introduction chapter title
\label{cap:intro}

Un sistema intelligente è caratterizzato dalla capacità di apprendere in modo auto-
matico, la quale a sua volta presuppone la capacità sia di rappresentare la conoscenza
nota a priori sia di inferirne di nuova. Le tecniche di \textit{estrazione di conoscenza} oggetto di studio  in questa sede riguardano l'apprendimento induttivo di linguaggi formali. L’induzione è un procedimento che elabora informazioni parziali riguardanti le proprietà di un insieme, fornendo una generalizzazione, cioè estendendo l’insieme su cui le proprietà valgono. Questo procedimento in generale non è logicamente giustificato, e di conseguenza non è certo che l’informazione fornita in uscita sia vera. All’interno degli studi sull’induzione nasce l’inferenza grammaticale, che si occupa di studiare le modalità con cui può essere individuato un linguaggio formale, quando è conosciuto un insieme di stringhe che appartengono ad un linguaggio sconosciuto e, eventualmente, un insieme di stringhe che non appartengono al linguaggio. Supponiamo che una sorgente di informazioni fornisca delle stringhe binarie: 
\begin{equation*}
    01, 0101, 010101, \dots
\end{equation*}
Ci si può domandare se c’è una regola formale con la quale la sorgente genera le stringhe, se eventualmente questa regola sia individuabile guardando solo l’insieme delle stringhe, e quali siano i fattori che influenzano l’identificabilità della regola. L’inferenza grammaticale cerca di trovare risposte a queste domande, nell’ipotesi che esiste una regola con cui le stringhe sono state create, e che si tratti di una grammatica generativa di Chomsky. \\
Nella fattispecie, la classe dei linguaggi oggetto di studio per la presente tesi è quella dei linguaggi regolari e si parla pertanto più specificamente di \textit{Inferenza Induttiva Regolare} (IIR), i cui limiti teorici sono stati messi in luce dal lavoro di Gold \cite{Gold67}. Gli algoritmi della letteratura iniziale sull’argomento riconducevano l’apprendimento ad una ricerca euristica in un spazio rappresentato come un grafo contenente gli automi consistenti con i campioni forniti in cui il modello,inizialmente iperspecializzato, viene progressivamente generalizzato. Tale approccio va sotto il nome di \textit{passive learning} ed è caratterizzato da un’inevitabile esplosione combinatoria che ne rende la complessità ingestibile a meno di non sacrificare le garanzie teoriche di terminazione e ottimalità o di stringenti requisiti sui campioni in ingresso.


Nella presente tesi si è quindi scelto di seguire un approccio proposto nella letteratura più recente, duale rispetto al precedente, secondo cui un modello, inizialmente molto generale e quindi poco accurato, viene progressivamente specializzato per rappresentare con precisione i dati forniti. Questo paradigma, noto come \textit{active learning}, presuppone l'esistenza di un \textit{oracolo} che guida l'apprendimento rispondendo ad alcuni tipi di query sottopostegli attivamente dal sistema che cerca di inferire il modello. La base teorica del paradigma è stata fornita dagli studi di Angluin \cite{Angluin87} che stabiliscono che l'oracolo debba appartenere alla classe dei cosiddetti Minimally Adequate Teacher (MAT) che, per garantire l'abilità di fornire risposte utili ai principali tipi di query considerate, richiede la conoscenza preliminare dell'automa oggetto d'inferenza.
Questo è un requisito molto forte e fin troppo stringente, che limita l'utilizzo in contesti reali dell'inferenza induttiva regolare declinata nell'accezione dell'\textit{active learning}.

Il lavoro in questa tesi nasce con l'intenzione di indagare lo scenario nel quale il tradizionale oracolo è sostituito con una sua approssimazione mediante un classificatore statistico costruito a partire da esempi positivi e negativi del linguaggio target nell'ottica di permetterne un utilizzo nelle applicazioni reali. L'obiettivo è quello di tracciare un parallelo tra i modelli ottenibili seguendo il paradigma della teoria dell'apprendimento statistico, Statistical Learning Theory (SLT), e quelli formulabili algoritmicamente nell'ambito dell'inferenza grammaticale con \textit{active learning} e di verificare sperimentalmente se l'uso combinato possa condurre al superamento dei limiti di entrambi ossia rispettivamente un modello poco significativo per i dati di partenza e il requisito di un oracolo onnisciente. In altre parole, si vuole costruire per i campioni dati un modello strutturale, da usare in luogo dell'oracolo in modo da superarne i requisiti stringenti.  Il modello statistico prescelto per affrontare il delineato problema di classificazione binaria è stato quello delle Support Vector Machines (SVM) perchè rappresentano un modello molto potente. Inoltre malgrado siano state proposte in letteratura delle loro applicazioni nel contesto dei linguaggi formali esse non sono state contestualizzate nell’ambito dell’inferenza induttiva regolare mediante \textit{active learning} oppure erano limitate all'apprendimento di specifiche classi di linguaggi sottoinsiemi di quelli regolari.

Al fine di valutare la correttezza della soluzione ottenuta, durante il lavoro di tesi si è progettata un'implementazione in linguaggio C++11, integrando il codice nella preesistente libreria di passive learning Gi-learning \cite{Cot16}. L'implementazione è stata testata sperimentalmente inizialmente su una nota classe di automi che rappresentano linguaggi semplici, ovvero i Tomita \cite{Tomita82, Dupont94}, ottenendo ottimi risultati paragonabili al caso ideale.
Infine le performance e la precisione dei modelli ottenuti dall'approccio qui proposto sono state vagliate su data set di automi estratti casualmente, la cui complessità è paragonabile ai sistemi da apprendere nei casi pratici. I risultati ottenuti rivelano che all'aumentare della complessità del target si ha un progressivo decadimento del grado di approssimazione ed è quindi possibile concludere che un livello di fiducia alto sull'utilizzo dell'algoritmo è ottenibile solo fino ad un certo livello di complessità del target.

Il lavoro qui esposto si divide in cinque parti. Nel primo capitolo si parlerà dell'inferenza induttiva, e riferendosi alla classificazione proposta in \cite{Mic86a}, si inquadrerà questo meccanismo nel complesso meccanismo dell'apprendimento.
Inoltre, dopo avere messo a confronto l'induzione con la deduzione e l'abduzione verranno passati in rassegna le peculiarità del processo induttivo.
Nel secondo capitolo si definirà l'inferenza induttiva grammaticale e saranno scandagliati brevemente i risultati teorici e i limiti dell'\ac{IIR}. Inoltre si descriverà brevemente il \textit{Passive Learning} tecnica duale all' \textit{Active Learning}. Infine sarà presentato brevemente il paradigma dell'\textit{Active Learning} e sarà introdotto e approfondito L* \cite{Angluin87} che può essere considerato il capostipite degli algoritmi di \textit{active learning}.
Nel terzo capitolo verrà esposto in maniera dettagliata la ratio che muove uno dei più efficienti algoritmi di \textit{active learning}:l'\ac{ObP} . Inoltre verranno riportate le scelte discostanti dal riferimento principale dell'algoritmo \cite{Howar12} e le motivazioni.
Il quarto capitolo è il cuore della tesi dove sarà descritto in dettaglio il lavoro  svolto per costruire l'oracolo approssimato e il suo utilizzo concreto all'interno del prescelto algoritmo di \textit{active learning} \ac{ObP}.
Nel quinto capitolo si descriveranno criticamente i risultati sui test eseguiti sul programma al fine di valutarne le prestazioni sia in termini di accuracy del classificatore ottenuto che in termini di similarità tra il \ac{DFA} inferito e il \ac{DFA} target. Inoltre si esaminerà come e quando è possibile variare alcuni parametri dell'algoritmo al fine di migliorarne le prestazioni e in quali contesti (complessità del linguaggio target,numero di esempi del linguaggio da apprendere) è possibile avere un livello di fiducia alto sull'utilizzo dell'algoritmo. 




\mainmatter 		% Begin numeric (1,2,3...) page numbering
\pagestyle{fancy} 	% Return the page headers back to the "fancy" style


%%%%%%%%%
% CAPITOLO1  %
%%%%%%%%%
%!TEX encoding = UTF-8 Unicode
%!TEX root = ./../main.tex
%!TEX TS-program = xelatex

\chapter{Inferenza Induttiva} % Main chapter title
\label{cap:uno}

Il metodo induttivo o induzione è un procedimento logico per cui dalla constatazione di fatti particolari si risale ad affermazioni o formulazioni generali.
Si suole quindi indicare con il termine induzione \index{induzione} il passaggio dal \textit{particolare} al \textit{generale}.
Con il termine \textbf{Inferenza Induttiva} si indica un processo che partendo da degli esempi specifici congettura delle regole generali. L'inferenza induttiva gioca un ruolo fondamentale nel più vasto scenario dell'apprendimento ed in ogni contesto che si prefigge la scoperta di strutture universali. L'applicazione di questo metodo scientifico ,intrinseco agli esseri intelligenti, all'interno delle macchine ha portato alla nascita di diversi filoni di ricerca. Uno dei più rilevanti tratta dei complessi meccanismi che consentono ad un uomo di imparare un linguaggio. 

\section{Apprendimento}
 
\subsection{Definire l'apprendimento}
\label{sub:defapp}
Insieme alla capacità di pianificare cioè elaborare piani, la capacità di apprendere è ritenuta uno dei segni distintivi di un sistema intelligente. Un sistema si può considerare autonomo fintantochè le sue azioni sono determinate dalle esperienze pregresse e dalle percezioni correnti, invece che dal suo progettista (si pensi agli agenti stimolo-risposta). Senza la capacità di apprendimento un sistema non sarà in grado di operare con successo in qualsiasi ambiente ma solo in quelli previsti dal suo progettista. Nonostante la grande importanza dell'apprendimento, una conclusione largamente diffusa è che non sia possibile darne una definizione precisa: si procede invece analizzando gli effetti che l'apprendimento ha eventualmente prodotto.

Due concetti rivestono un ruolo importante nell'apprendimento: 
\begin{itemize}
\item Il miglioramento delle capacità del sistema che apprende
\item L'acquisizione di nuova conoscenza
\end{itemize}
Simon \cite{Sim83} approfondisce cosa significa migliorare le capacità di un sistema mediante l'apprendimento: \textit {L'apprendimento identifica delle modifiche in un sistema che sono adattive, nel senso che consentono al sistema di svolgere lo stesso goal, o goals analoghi, in maniera migliore nel futuro}. E' doveroso però osservare  che esistono sistemi che migliorano nel tempo senza essere soggetti a nessun processo di apprendimento e che esistono degli scenari in cui non è facile calare la definizione data da Simon.\\

L'altro fattore che contraddistingue l'apprendimento è l'acquisizione di nuove conoscenze che presuppone a monte una rappresentazione della conoscenza in maniera descrittiva o iconica per potere rappresentare la nuova conoscenza avvenuta mediante l'apprendimento.  In quest'ottica \emph{l'apprendimento è creare e modificare rappresentazioni di ciò che è stato sperimentato}. Laddove con sperimentare si intende sia l'informazione proveniente dall'apparato sensoriale dell'agente sia ciò che il sistema recepisce mediante processi interni (ad esempio ripetere più volte una frase tra sè e sè ci consente d'impararla nonostante non è avvenuto nessuno stimolo dai nostri sensi).Da questo punto di vista apprendere significa costruire una rappresentazione della realtà anzichè un miglioramento delle capacità dell'agente, aspetto quest ultimo considerato come una conseguenza.\\

E' possibile quindi constatare il grado di apprendimento di un sistema misurando i miglioramenti nel portare a compimento un certo job dopo l'apprendimento. (Miglioramenti che implicitamente sono considerati una conseguenza della rappresentazione interna dell'agente della realtà esterna). Si fa presente che in questa caratterizzazione si assume che l'agente abbia un obbiettivo e che tale obbiettivo sia conosciuto all'osservatore che valuta l'apprendimento. 
\subsection{Una possibile schematizzazione}
Esistono diverse possibilità di classificare i fattori che influenzano l’apprendimento. Michalski \cite{Mic86b} propone una divisione  basata sulle
caratteristiche del sistema che apprende.

Michalski esegue una prima distinzione in base alla quantità di conoscenze
iniziali di cui il sistema è dotato. Ai due estremi della classificazione troviamo
le reti neurali artificiali e i sistemi esperti. Nel contesto dei sistemi dotati di
scarse conoscenze iniziali le reti neurali sono uno strumento largamente usato:
le connessioni dei neuroni che costituiscono il sistema, sono determinate in
maniera essenziale dagli esempi presentati e solo marginalmente
dai valori iniziali (di solito casuali) delle connessioni. Nella progettazione di un sistema esperto invece una grande quantità di informazione viene fornita al sistema.
Un altro approccio, suggerito da Michalski, propone di suddividire i
sistemi artificiali che apprendono in base al tipo di manipolazione eseguita dal \textbf{learner}(sistema che apprende) sull’informazione proveniente dall'esterno. In ogni processo di apprendimento il \emph{learner} trasforma l’informazione fornita da un \textit{teacher}, o più in generale da un \textbf{informant} sorgente di informazione, in una nuova forma che viene poi memorizzata per usi futuri. Questa trasformazione dell’informazione, che fa uso anche delle conoscenze già possedute dal \emph{learner} viene chiamata \textbf{inferenza}. Il tipo di trasformazione eseguita determina la strategia di apprendimento di cui il sistema fa uso. Si possono distinguere,seguendo esattamente Michalski in \cite{Mic86a} , cinque diverse strategie:
\begin{itemize}
\item Apprendimento per \textit{imitazione}
\item Apprendimento per \textit{istruzioni}
\item Apprendimento per \textit{deduzione}
\item Apprendimento per \textit{analogia}
\item Apprendimento per \textit{induzione} 
\end{itemize}

Queste strategie sono elencate in ordine crescente di complessità del learning e decrescente di difficoltà del teaching.
Michalski attua questa classificazione restrigendo l'ambito di applicabilità al \textbf{\textit{concept learning}} una branca del machine learning. Un sistema intelligente deve essere abile nel classificare alcuni oggetti,eventi o comportamenti come equivalenti per raggiungere un determinato goal.  Detto in maniera succinta un sistema intelligente deve essere capace di individuare i \textit{concetti}.
\begin{definizione}
Un \textbf{concetto} è una classe di equivalenza  per cui esiste un metodo operativo che permette di discriminare le istanze come appartenenti o non appartenti al concetto.
\end{definizione} 
Dove le istanze sono le singole entità della classe di equivalenza (del concetto),cioè gli esempi  presentati dall'informant. Un \textit{learner} impara un concetto quando ,tramite una procedura effettiva, è in grado di distinguere le entità che appartengono al concetto da quelle che non appartengono.Adesso si prenderanno brevemente in esame le cinque strategie di apprendimento contestualizzandole nel \textit{concept learning}:
\begin{description}
\item[Apprendimento per imitazione] Questo è il caso estremo in cui il \textit{learner} non deve effettuare alcuna inferenza sulle informazioni che gli provengono dall' \textit{informant}. Infatti questo metodo è anche detto da Michalski impianto diretto di conoscenza (meglio conosciuto ancora come rote learning) proprio perchè il \textit{learner} non deve fare altro che inidicizzare l'informazione per poterla poi recuperare. In questo caso l'\textit{informant} fornirà una descrizione del concetto in input al \textit{learner}. Questa strategia è usata quando uno specifico algoritmo per riconoscere un concetto è implementato su un calcolatore(oppure vi è a disposizione un database di fatti che permette di riconoscere il concetto) . Ad esempio nei primi programmi che giocavano a scacchi si salvavano i risultati dell'esplorazione del grafo di  ricerca (in alcuni punti che rappresentano possibili situazioni in una partita) in un albero di gioco in modo che quando una situazione già memorizzata si fosse presentata in una partita reale si potesse risparmiare spazio e tempo di esecuzione.
\item[Apprendimento per istruzioni]In questo caso il \textit{learner} acquisisce un concetto da un \textit{teacher}, o da un’altra forma organizzata di informazione, come una publicazione o un libro, ma non copia direttamente in memoria l’informazione acquisita. Nell’apprendimento per istruzioni le trasformazioni sull’informazione eseguite dal \textit{learner} sono la selezione e la riformulazione a livello sintattico. Il processo di apprendimento può consistere
nel selezionare i fatti più importanti e poi trasformarli in una forma più appropriata. Un programma che costruisce una database di fatti e
regole sulla base di una conversazione con un utente è un esempio di sistema che apprende per istruzioni.
\item[Apprendimento per deduzione]Il \textit{learner} acquisisce un concetto deducendolo dalle conoscenze fornite dall' \textit{informant} insieme a quelle che il sistema già possedeva. Inoltre, questa strategia include ogni processo nel quale la conoscenza appresa è il risultato di una trasformazione che preserva la verità delle informazioni generate dall' \textit{informant} e di ciò che viene inferito. All’interno dell' apprendimento dei concetti, l’apprendimento per deduzione tramite il processo inferenziale trasforma una definizione non adoperabile per discriminare il concetto, in una definizione operativa adatta a questo scopo. Ad esempio dal fatto che una giarra sia un oggetto stabile e trasportabile, si può dedurre che la brocca ha un fondo piatto e un manico.
\item[Apprendimento per analogia]Il \textit{learner} acquisisce un nuovo concetto modificando la definizione di un concetto simile già noto. Anzichè formulare una descrizione del concetto ex novo, il sistema adatta una descrizione esistente modificandola appropriatamente per il nuovo scopo. Ad esempio se già si conosce una regola che definisce il concetto di arancia, per imparare il concetto di mandarino si possono notificare le differenze e le similitudini tra arancia e mandarino. L’apprendimento per analogia può essere visto come un incrocio tra l’apprendimento deduttivo e quello induttivo. Attraverso l’inferenza induttiva si possono determinare le caratteristiche generali o le trasformazioni che unificano i concetti confrontati. Poi, attraverso un’inferenza deduttiva si possono derivare le proprietà caratterizzanti possedute dal concetto che deve essere appreso
\item[Apprendimento per induzione]In questa strategia  il \textit{learner} acquisisce un concetto effettuando inferenza induttiva sui fatti forniti dall' \textit{informant} o in base a delle osservazioni su tali fatti. Esistono due differenti forme di questa strategia:
\begin{enumerate}
\item \textbf{Apprendimento da esempi}\\
Al \textit{learner} partendo da degli esempi specifici (istanze del concetto) ed eventualmente dei controntroesempi induce una descrizione del concetto catturando la struttura generale . Si assume che il concetto esiste e che esiste anche un metodo effettivo per testare l'appartenenza di un'istanza ad un concetto.  Il compito del \textit{learner} è determinare una descrizione del concetto  analizzando le singole istanze del concetto. Questa strategia è utilizzata nell' \ac{IIR} 
\item \textbf{Apprendimento per osservazione e scoperta}\\
Il \textit{learner}  analizza le entità  in input e determina che qualche sottoinsieme di queste entità può essere raggruppato in un singolo concetto.Poichè , diversamente dall'apprendimento da esempi, non c'è un \textit{teacher} che conosce in anticipo i concetti questa strategia è talvota menzionata come \emph{unsupervisioned learning}. Un esempio è il \emph{clustering} cioè il partizionamento di una collezione di oggetti all'interno di gruppi o classi che avviene in maniera gerarchica; l'eredità gioca un ruolo importante: se un entità è riconosciuta appartenere ad un determinato concetto erediterà da esso e dai concetti più in alto nella gerarchia  tutte le proprietà . Ad esempio se si apprende che Freddy è un elefante, allora si può,senza vedere Freddy, dire che ha la proboscide e tutte le proprietà degli elefanti e più in generale  anche degli erbivori e dei mammiferi.
\end{enumerate}
\end{description}

\section{Induzione}
L'induzione è quel procedimento logico che permette di passare dal particolare all'universale. Questa definizione è troppo semplice e non spiega tutte le componenti in gioco nel processo induttivo. A tal fine si seguirà ancora \cite{Mic86a}. Qui le  principali componenti induttive sono più precisamente  distinte e specificate nel contesto della manipolazione simbolica:\\

\begin {description}
\item{Dati i seguenti elementi di partenza}
\begin {itemize}
\item Gli \ac{P} che comprendono fatti, generalizzazioni intermedie, specifiche osservazioni che forniscono informazioni su oggetti,fenomeni,processi eccetera. Costituiscono l'input del processo inferenziale.
\item Le \ac{BK} che contengono concetti generali o specifici del dominio, che permettono di interpretare gli enunciati premessa e le regole rilevanti per l'inferenza. Ed includono concetti precedentemente imparati, vincoli del dominio, relazioni di causalità,goals dell'inferenza,e metodi per valutare la bontà di una congettura in base al goal (criterio di preferenza)

\end{itemize}
\item{si determina alla fine dell'inferenza induttiva}
\begin{itemize}
\item Una \ac{H} che implica gli enunciati premessa nel contesto delle conoscenze di background ed è l'ipotesi migliore in base al criterio di preferenza.
\end{itemize}
\end{description}
Si dice che \ac{H} implica fortemente  \ac{P}  nel contesto di \ac{BK} se usando \ac{BK} e l'inferenza deduttiva  \ac{P} è una conseguenza logica di \ac{H}. Schematizzando si ottiene l'equazione
\begin{equation}
\label{eqn:conslogica}
\text{\ac{H}} \,  \lor \,  \text{\ac{BK}} \implies \text{\ac{P}}
\end{equation}
che è vera con tutte le possibili \textit{interpretazioni}. In contrasto \ac{H} implica debolmente gli enunciati premessa  nel contesto delle \ac{BK} se usando le \ac{BK} e l'inferenza deduttiva \ac{P} è solo una conseguenza plausibile ma non una conseguenza logica.
Michalski fornisce un esempio di quanto appena detto:\\

\textbf{Enunciati premessa}
\begin{rientra}
 Aristotele era greco \\Socrate era greco\\Platone era greco\\
 \end{rientra}
 
\textbf{Conoscenze di background}
\begin{rientra}
Socrate, Aristotele e Platone erano filosofi \\Sono vissuti nell'antichità\\I greci sono persone\\
I filosofi sono persone\\ \emph{Criterio di preferenza}: Si preferiscono le ipotesi più corte e più utili per decidere la nazionalità dei filosofi\\
\end{rientra}

Le \textbf{ipotesi induttive} sono:
\begin{enumerate}
\item I filosofi che hanno vissuto nell'antichità erano greci
\item Tutti i filosofi sono greci
\item Tutte le persone sono greche
\end{enumerate}

L'ipotesi da preferire,in base al criterio di preferenza, è la 2, perchè è più breve della 1 e più specifica della 3; consente a differenza della 1 di determinare la nazionalità di tutti i filosofi. Si può di mostrare che questa ipotesi induttiva è un'ipotesi forte,poichè \ac{P} risulta essere una conseguenza logica di \ac{H} e delle \ac{BK}.

Supponiamo di aggiungere alla premessa gli enunciati  Locke era inglese e Hume era inglese e di modificare le \ac{BK} aggiungendo il fatto che sia Locke che Hume erano filosofi. In questo caso una ipotesi induttiva forte potrebbe essere che tutti i filosofi erano greci, con l’eccezione di Locke e Hume. Mentre una ipotesi induttiva debole potrebbe essere che alcuni filosofi erano greci. Dal fatto che Platone era un filosofo e sulla base di questa nuova ipotesi debole non consegue che Platone era greco, consegue solo che c'è la possibilità che Platone fosse greco.

Senza pretesa di esaustività si accenna ad altri tipi di inferenza presenti nel pensiero logico con lo scopo di fare emergere le peculiarità dell'inferenza induttiva. L'inferenza sta alla base dell'apprendimento. Seguendo \cite{Mic93} l'apprendimento si può sintetizzare in $apprendimento=inferenza+memorizzazione$ (definizione leggermente diversa da quella data in \ref{sub:defapp} ) quindi una completa teoria dell'apprendimento deve includere una completa teoria dell'inferenza  \cite{Mic93}. Viene innanzitutto generalizzata l'equazione \eqref{eqn:conslogica} valida solo per l'induzione  ottenendo:
\begin{equation}
\label{eqn:fond}
\text{Q} \,  \lor \,  \text{\ac{BK}} \models \text{C}
\end{equation}
detta \textbf{equazione fondamentale} per l'inferenza.
Poi Michalsky effettua una prima suddivisione tra i metodi d'inferenza:
\begin{enumerate}
\item conclusivi
\item contingenti
\end{enumerate} 
 Nel secondo caso nell' equazione \eqref{eqn:fond} C è solo una plausibile,parziale,probabilistica conseguenza logica delle \ac{BK} e di \textit{Q}. Nell'inferenza conclusiva invece la conseguenza logica è garantita.
Le proprietà dell'inferenza induttiva  sono confrontate  con quelle dell'inferenza deduttiva ed emerge che sono duali come si vede in figura \ref{fig:dedind}
\begin{figure}[htp]
	\centering
	\includegraphics[ width=0.5\textwidth]{DedInd}
	\caption[Relazione tra deduzione e induzione]{Relazione tra deduzione e induzione}
   \label{fig:dedind}
\end{figure} 
La relazione logica \eqref{eqn:fond} succintamente cattura la relazione tra i due tipi d'inferenza. L'inferenza deduttiva deriva logicamente \textit{C} date \ac{BK} e \textit{Q}. L'inferenza induttiva invece va ad ipotizzare \textit{Q} date \ac{BK} e \textit{C}. La deduzione è il processo di determinare una conseguenza logica a partire da una conoscenza data, ed è \emph{truth-preserving} ( \textit{C} deve essere vero se \ac{BK} e \textit{Q} sono vere). In contrasto l'induzione sta ipotizzando un \textit{Q} che insieme con \ac{BK} implica l'input \textit{C}, ed è \emph{false-preserving} (se \textit{C} è falso  allora anche \textit{Q} deve essere falso. Cioè se  l'input in ingresso è falso anche le ipotesi congetturate saranno false). La deduzione contingente invece suona come debole in quanto è debolmente \textit{true-preserving} cioè produce conseguenze che possono essere vere in alcune situazioni e false in altre. Analogamente l'induzione contingente è debolmente \textit{false-preserving}.

In \cite{Mic93} l'inferenza viene considerata come un processo che prende in  Input un enunciato  e tramite le \ac{BK} già possedute (ed eventualmente la conoscenza dei criteri di preferenza per il goal che  permette di restringere tutte le possibili ipotesi tra le quali scegliere) fornisce un enunciato in Output .In quest'ottica le proprietà dell'induzione sono messe in risalto dal confronto con quelle della deduzione e dell'abduzione fornendo per ciascuno di essi degli esempi chiarificatori. 
\begin{enumerate}


\item DEDUZIONE tabella \ref{tab:ded}
 Riferendosi all'equazione fondamentale \eqref{eqn:fond} l'Input sta per \textit{Q} e l'Output sta per \textit{C}. L'Input consiste in un enunciato che afferma l'appartenenza di un elemento \textit{a} ad \textit{X}. Le \ac{BK} sono costituite da un enunciato che assegna una certà proprietà \textit{q} agli elementi dell'insieme \textit{X}, e da una regola logica detta \textit{regola di specializzazione universale}. L'inferenza consiste solo nell'applicazione di tale regola che essendo una tautologia \footnote{E' un enunciato che ha sempre valore logico vero} fa si che il risultato dell'inferenza deduttiva assuma pure valore logico vero. Questo è un esempio di inferenza deduttiva conclusiva dato che l'Output è sempre una conseguenza logica dell'Input e delle \ac{BK}  
\begin{table}[htp]
\centering 
\begin{tabular}{|M{0.1\textwidth}|M{0.5\textwidth}|M{0.3\textwidth}|} 

\hline 
\textbf{Input} & $a \in X$ & $a$ è un elemento di $X$ \\
 \hline  
\multirow{2}*{\textbf{BK}}  &  $\forall x \in X,q(x)$  & Tutti gli elementi di $X$ hanno la proprietà $q$. \\[6ex] \cline{2-3} & $ \forall x \in X,q(x) \implies (a \in X \implies q(a))$ &  Se tutti gli elementi di $X$ hanno la proprietà $q$, allora ogni elemento di $x$, e quindi anche $a$, deve avere la proprietà $q$ \\
\hline 
\textbf{Output}  &  $q(a)$ & $a$ ha la proprietà $q$ \\
\hline 
 \end{tabular}
 \caption[Deduzione]{Deduzione}
\label{tab:ded}
\end{table} \\

\item INDUZIONE tabella \ref{tab:ind} Riferendosi all'equazione fondamentale \eqref{eqn:fond} l'Input è la conseguenza \textit{C} e l'Output è \textit{Q} (l'ipotesi).Si può dimostrare che l'Input è conseguenza logica dell'Output (l'ipotesi) e delle \ac{BK} quindi dato che l'equazione fondamentale \ref{eqn:fond} è rispettata l'inferenza è conclusiva (forte).  Infatti nel processo inferenziale è \emph{false-preserving}  se l'Input fosse falso (\textit{a} non ha la proprietà \textit{q}) allora l'Output avrebbe dovuto essere pure falso. Da rimarcare è che l'Output dell'inferenza induttiva (sia che sia conclusiva che contingente) non ha un valore di verità sempre vero ma può essere vero o falso (anche se l'Input e le \ac{BK} sono vere) da cui deriva il termine ipotesi per connotare l'Output.Essa si basa sull'assunzione che determinate regolarità osservate in un fenomeno continueranno a manifestarsi nella stessa forma anche in futuro e quindi generalizza ciò che è vero per alcune istanze ad un insieme più grande. Invece nell'inferenza deduttiva  conclusiva  è garantito logicamente che l'Output assuma valore di verità vero se l'Input e le \ac{BK} sono pure vere perchè ciò che è vero in generale resta vero in un caso specifico contemplato dalla regola generale. 

Nell'esempio riportato le conoscenze di \ac{BK} sono le stesse della deduzione. Tuttavia l'Output(l'ipotesi) è ottenuto tracciando all'indietro la \textit{regola di specializzazione universale}. Quindi l'inferenza consiste nel supporre l'implicazione presente nella regola di specializzazione valida anche nel verso opposto, perciò si dice che l'induzione è una regola d'inferenza all'indietro a la deduzione una regola d'inferenza in avanti.
\begin{table}[htp]
\centering 
\begin{tabular}{|M{0.1\textwidth}|M{0.5\textwidth}|M{0.3\textwidth}|} 

\hline 
\textbf{Input} & $q(a)$ & $a$ ha la proprietà $q$ \\
 \hline  
\multirow{2}*{\textbf{BK}}  &  $a \in X$  & $a$ è un elemento dell'insieme $X$. \\[6ex] \cline{2-3} & $ \forall x \in X,q(x) \implies (a \in X \implies q(a))$ &  Se tutti gli elementi di $X$ hanno la proprietà $q$, allora ogni elemento di $x$, e quindi anche $a$, deve avere la proprietà $q$ \\
\hline 
\textbf{Output}  &  $ \forall x \in X,q(x)$ & Tutti gli elementi di $X$ hanno la proprietà $q$ \\
\hline 
 \end{tabular}
 \caption[Induzione]{Induzione}
\label{tab:ind}
\end{table} \\

\item ABDUZIONE tabella \ref{tab:abd} In riferimento all'equazione \eqref{eqn:fond} l'Output è \textit{Q} e l'Input è \textit{C}.Si può dimostrare che l'Input è conseguenza logica dell'Output (l'ipotesi) e delle \ac{BK} quindi dato che l'equazione fondamentale \ref{eqn:fond} è rispettata l'inferenza è conclusiva (forte).Come nel caso dell'induzione l'inferenza abduttiva conclusiva è \textit{false-preserving}. Come nell'induzione l'Output è solo un' ipotesi e quindi il suo valore di verità è incerto e c'è solo una probabilità che sia vero. L'abduzione, come l'induzione, non contiene in sé la sua validità logica e deve essere confermata per via empirica.
Nell'abduzione come nell'induzione la regola implicativa di specializzazione universale viene tracciata all'indietro. Tuttavia c'è un'importante differenza infatti nell'induzione  la regola implicativa nelle \ac{BK} costituisce una tautologia mentre nel caso dell'abduzione rappresenta una verità solo nel dominio di conoscenza e non una verità universale.

Nell'esempio specifico si assume che un elemento \textit{a} gode della proprietà \textit{q}. Le \ac{BK} consistono in un unico enunciato, che esprime il fatto che tutti gli elementi di un certo insieme \textit{X} hanno la proprietà \textit{q}. L'inferenza abduttiva produce in Output un enunciato che asserisce l'appartenenza di  \textit{a} ad  \textit{X}. Intuitivamente tutti gli elementi che appartengono ad un insieme \textit{X} hanno una proprietà; dall'input si ha che un elemento \textit{a} ha quella proprietà; siccome tutti gli elementi appartenenti all'insieme \textit{X} possiedono quella stessa proprietà si suppone che  \textit{a} appartiene all'insieme \textit{X} 
\begin{table}[htp]
\centering 
\begin{tabular}{|M{0.1\textwidth}|M{0.5\textwidth}|M{0.3\textwidth}|} 

\hline 
\textbf{Input} & $q(a)$ & $a$ ha la proprietà $q$ \\
 \hline  
\textbf{BK}  &  $\forall x,x \in X \implies q(x)$  & Se $x$ è un elemento di $X$ allora $x$ ha la proprietà $q$. \\ \cline{2-3}
\hline 
\textbf{Output}  &  $a \in X$ & $a$ è un elemento di $X$ \\
\hline 
 \end{tabular}
 \caption[Abduzione]{Abduzione}
\label{tab:abd}
\end{table} \\
\end{enumerate} 

\subsection{Metodologia di ricerca induttiva}
Si introduce brevemente,seguendo ancora \cite{Mic86a} il \textit{learning da esempi} induttivo di un concetto come un problema di ricerca in uno spazio. L'algoritmo inferenziale induttivo riceve in ingresso degli esempi (ed eventualmente anche controesempi) ,di membri del concetto target(specifiche istanze) , sottoinsieme dello \textbf{spazio delle istanze} che costituisce l'insieme di tutte le possibili istanze osservabili. Lo \textbf{spazio dei concetti} costituisce invece l'insieme di tutti i possibili concetti (tutte le possibili soluzioni). I concetti quasi sempre necessitano di una descrizione, un linguaggio che formalmente consente di definire operativamente un concetto e per questo si parla in maniera interscambiabile di \textbf{spazio delle descrizioni}. Un concetto è consistente se accetta alcuni esempi positivi e rifiuta tutti quelli negativi; è completo invece quando accetta tutti gli esempi positivi. La macchina inferenziale induttiva ha lo scopo di selezionare un'ipotesi dallo \textbf{spazio delle ipotesi} che sia consistente e completa con gli esempi visti. Lo spazio delle ipotesi è un sottoinsieme dello spazio dei concetti. All'aumentare degli esempi visti lo spazio delle ipotesi si riduce, tuttavia le ipotesi valide possono comunque essere numerose e spesso è necessario utilizzare dei criteri di preferenza per scegliere l'ipotesi corrente. E' necessario anche definire dei criteri di terminazione per sancire la ricerca conclusa.  In sintesi il \textit{concept learning induttivo} può essere descritto come una ricerca euristica nello spazio delle descrizioni della migliore ipotesi tra tutte quelle consistenti e complete rispetto agli esempi forniti.

%!TEX encoding = UTF-8 Unicode
%!TEX root = ./../main.tex
%!TEX TS-program = xelatex

\chapter{Inferenza Grammaticale} % Second chapter title
\label{cap:due}

L' \ac{GI} è considerata una branca del \textit{machine learning} sebbene il primo algoritmo di \ac{GI} sia più datato della nascita del concetto di machine learning. Più in dettaglio \ac{GI} è un'istanza dell'inferenza induttiva e in particolare dell'apprendimento per induzione da esempi introdotti in \ref{sub:tipiapp}, e può quindi essere descritto come il problema di congetturare un linguaggio target sconosciuto  a partire da un \textit{training set} che di solito comprende un insieme finito di stringhe $S^{+}$ appartenenti ad un \ac{L} ,definite su un alfabeto $\Sigma$, dette positive ed eventualmente anche un insieme finito di stringhe negative $S^{-}$ che non appartengono ad \ac{L}.

Sebbene il nome \textit{inferenza grammaticale} potrebbe suggerire che l'\textit{output} di un algoritmo d'\textit{inferenza grammaticale} sia una grammatica non è questo il caso --- qualsiasi altra descrizione di un linguaggio come un automa, un'espressione regolare, ecc. può essere pure usata. In questa tesi l'attenzione è volta agli algoritmi di \ac{GI} il cui \textit{linguaggio target}  appartiene alla classe più semplice della gerarchia di Chomsky: i linguaggi regolari. In quest ultimo caso si suole parlare di \textbf{inferenza grammaticale regolare} e talvolta di \textbf{inferenza induttiva regolare} (\ac{IIR}).

\section{Connotare l'IIR}
Una caratterizzazione del \textit{learning da esempi}  induttivo  ,seguendo \cite{Mic86a}, è stata già fornita in \ref{sub:appindes}. In questa sede è opportuno puntualizzare ed approfondire alcuni dei punti di cui si compone e soprattutto contestualizzarla all'\ac{IIR} dato che la classificazione in \ref{sub:appindes} fa riferimento all'apprendimento generico di un concetto e nell'\ac{IIR} si specializza l'oggetto dell'inferenza che diventa un linguaggio regolare.  I punti di cui si compone sono\footnote{Da adesso in poi e per tutto il resto della tesi ci si discosterà dalla definizione di consistenza di Mychalski esposta in \ref{sub:appindes} a meno che non venga espressamente indicato. Con consistenza si intenderà che, dato un \ac{DFA} $A$ e un insieme di istanze $S=S^{+} \cup S^{-} \,\, \forall x \in S^{+} :   \lambda^{A}(x)=1 \land \forall x \in S^{-} :   \lambda^{A}(x)=0$ }:

\begin{description}
\item[Spazio delle istanze]L’algoritmo inferenziale
induttivo riceve in ingresso degli esempi (ed eventualmente anche controesempi),di membri del concetto target(specifiche istanze) , sottoinsieme dello spazio delle istanze che costituisce l’insieme di tutte le possibili istanze osservabili. Nell'\ac{IIR} le istanze sono delle stringhe definite sull'alfabeto $\Sigma$. 
\item[Spazio dei concetti]E' l'insieme di tutte le possibili soluzioni e nell'\ac{IIR} rappresenta l'insieme dei linguaggi regolari.
\item[Spazio delle descrizioni]Costituisce lo spazio che contiene le descrizioni operative degli elementi dello spazio dei concetti. Nell'\ac{IIR} si possono usare le espressioni regolari, gli \textit{NFA} o i \ac{DFA} anche se di norma si usano i \ac{DFA} perchè è garantita l'esistenza di un \ac{DFA} canonico.
\item[Spazio delle ipotesi]Lo spazio delle ipotesi contiene quei concetti consistenti con gli esempi osservati ed è quindi un sottoinsieme dello spazio dei concetti.
\item[Criteri di successo]Sono i criteri che permettono di decretare che il processo d'inferenza è concluso. Esistono varie tecniche e varianti ma i due metodi principali sono:
\begin{itemize}
\item{\textit{Identification in the limit}}
\item{\textit{PAC-learning}}
\end{itemize}
\begin{definizione*}[Identification in the limit]Un linguaggio \ac{L} è identificato al limite da un algoritmo di inferenza  se ad un certo punto del processo inferenziale l' ipotesi intermedia $H$ generata è una descrizione di \ac{L} e da quel punto in poi $H$ non muta al variare delle istanze presentate in ingresso all'algoritmo.
\end{definizione*}
Si è supposto che le istanze presentate in ingresso ad ogni iterazione dell'algoritmo di inferenza siano crescenti e includenti le istanze contenute alle iterazioni precedenti, e che le istanze possano divenire anche infinite. Allora si ha identificazione al limite \cite{Gold67} se a partitire da una determinata iterazione dell'algoritmo induttivo \ac{H} resta invariata. \\
Nel \textit{PAC-learning} \cite{Val84} invece si richiede una identificazione solo parziale e probabilistica di \ac{L}. Sia \ac{L} il target e \ac{H} l'ipotesi induttiva generata da un algoritmo di \ac{IIR} . Sia $D$ una distribuzione di probabilità di tutte le stringhe su $\Sigma^{*}$ allora si ha la seguente definizione:
\begin{definizione*}[True error]Il \textbf{\textit{tasso di errore}} o \textbf{\textit{true error}} $error_{D}(H)$ di \ac{H} rispetto alla distribuzione  $D$ e ad \ac{L} è
\begin{equation*}
    error_{D}(H) = \sum_{x \in \ac{L} \oplus L(\ac{H})}^{}D(x)
\end{equation*}
\end{definizione*}
Informalmente il \textit{true error} è la probabilità che una stringa estratta casualmente dallo spazio delle istanze --- in accordo alla distribuzione di probabilità $D$ --- appartenga allo spazio dove \ac{H} ed \ac{L} differiscono . L'algoritmo induttivo nel \textit{PAC-learning} prende due parametri in ingresso l'accuratezza $\epsilon$ e la confidenza $\delta$ entro cui operare.

\begin{definizione}[PAC learning]\label{def:PacL}Una classe di linguaggi $\mathbb{L}$ è \textbf{PAC-learnable} se esiste un algoritmo $A$ tale che $\forall L \in \mathbb{L}$, per ogni distribuzione $D$ su $\Sigma^{*} \text{,} \forall \epsilon(0<\epsilon<1) \text{,} \forall \delta(0<\delta<1), A$ su istanze fornite in accordo alla distribuzione $D$ produce con probabilità $1-\delta$ un'ipotesi \ac{H} tale che $error_{D}(H) \leq \epsilon$.
\end{definizione}
\end{description}

E' possibile trovare una caratterizzazione alternativa ma analoga del problema induttivo rispetto a quella data in \cite{Angluin83} in cui si effettua una classificazione anche in base a come sono presentate le istanze e al metodo impiegato nello spazio di ricerca:
\begin{description}
\item[Presentazione delle istanze]Se l'insieme di istanze disponibili per l'algoritmo d'inferenza induttiva regolare, S , per apprendere una \ac{DFA}  target $A$ sono tali che $\forall x \in S \,\lambda^{A}(x) = 1$ si parla di \textbf{presentazione positiva} cioè tutti gli esempi(istanze) sono accettanti nel \ac{DFA} $A$. In questo caso si scrive $S=S^{+}$. Questo \textit{setting} è  denominato \textit{text learning}\cite[p. 217]{DeLaHiguera10}. Si parla invece di \textbf{presentazione completa} quando $S=S^{+} \cup S^{-}$ cioè esistono sia esempi accettanti che rigettanti in simboli $\exists x \in S :   \lambda^{A}(x)=1 \land \exists x \in S :   \lambda^{A}(x)=0$. Quest ultimo caso è menzionato come \textit{informed learning} \cite[p. 237]{DeLaHiguera10} .\\

Si effettua un'ulteriore suddivisione del tipo di presentazione in base a come gli esempi vengono presentati all'algoritmo d'inferenza:
\begin{itemize}
\item \textbf{Presentazione Given-Data}\\Le istanze sono un insieme finito presentate totalmente fin dall'inizio del processo.
\item \textbf{Presentazione completa}\\Le istanze (positive o complete) sono una sequenza infinita e sono presentate in successione (in maniera incrementale).  
\item Esiste un \textbf{\textit{Oracolo}} che è in grado di rispondere a delle \textit{membership query} ed \textit{equivalence query} ed è la macchina inferenziale che direttamente interroga attivamente l'\textit{Oracolo}. Questa modalità sarà approfondita nel capitolo \ref{sec:acl}.
\end{itemize}
\item[Metodo inferenziale]Gli algoritmi di \ac{GI} e di \ac{IIR} si possono ricondurre a dei problemi di ricerca in un grafo in cui ogni nodo è un'astrazione che rappresenta uno dei possibili linguaggi dello spazio dei concetti. \ac{L} ,cioè il target, è da ricercare in questo spazio.  Allora con algoritmo d'inferenza \textit{astratto} si intende un algoritmo che ha valenza solo teorica perchè una effettiva realizzazione sarebbe troppo onerosa computazionalmente. Il più importante di questi algoritmi è l'algoritmo di \textit{induzione per numerazione} per il quale esiste la seguente congettura largamente condivisa:

\begin{teorema*}La classe dei linguaggi ricorsivi identificabili al limite da un algoritmo di inferenza grammaticale , è anche identificabile al limite da un algoritmo di induzione per numerazione
\end{teorema*} 
 L'induzione per numerazione cerca ad ogni iterazione dell'algoritmo tra le descrizioni di tutti i concetti consistenti con gli esempi in ingresso e ne sceglie uno secondo qualche criterio di preferenza. Tali descrizioni sono tipicamente un insieme di cardinalità infinita e ciò rende questo metodo impraticabile.\\
 Per algoritmi induttivi \textit{concreti} si intende invece quegli algoritmi efficientemente implementabili. Tra questi possiamo distinguere tra:
 \begin{itemize}
 \item \textbf{Algoritmi esaustivi}\\
 Sono algoritmi deterministici che sotto opportune condizioni degli esempi d'ingresso garantiscono di trovare il linguaggio target
 \item \textbf{Algoritmi euristici}\\
 Si tralascia di seguire dei percorsi di ricerca nel grafo per aumentare l'efficienza ,di solito a scapito della perdita della garanzia di successo. Tipicamente si usano dei meccanismi aleatori, come nel caso degli algoritmi genetici che manipolano delle informazioni simboliche attraverso dei processi aleatori numerici.
 \end{itemize}  
\end{description}

\section{Limiti dell' IIR}
\label{sec:limIIR}
Uno dei problemi dell'\ac{GI} e di \ac{IIR} è che nessun sottoinsieme finito di un insieme infinito ha informazioni sufficienti che gli consentono di inferire con assoluta certezza a quale insieme il sottoinsieme appartiene. Nella fattispecie ,dato un insieme finito di stringhe $S$,  appartenente a un linguaggio infinito \ac{L} , non si può inferire \ac{L} da $S$ con assoluta certezza. Il motivo è che $S$ è un sottoinsieme di molti altri, tipicamente infiniti, linguaggi diversi da \ac{L}. Questo problema è stato trattato da Gold in \cite{Gold67} nel paradigma dell'\textit{identification in the limit} ottenendo i seguenti risultati:
\begin{teorema}
La classe dei linguaggi superfiniti non può essere identificata al limite attraverso una presentazione positiva di esempi.
\end{teorema}
Ne consegue che i linguaggi regolari di cui i linguaggi superfiniti sono un sottoinsieme non sono parimenti identificabili al limite solo tramite esempi positivi. Considerando anche gli esempi negativi il seguente teorema sempre in  \cite{Gold67}  stabilisce il limite alla classe di linguaggi inferibili:
\begin{teorema}
Non è possibile inferire l'intera classe dei linguaggi ricorsivi attraverso una presentazione completa degli esempi.
\end{teorema}
Invece i linguaggi primitivi ricorsivi e quindi anche i linguaggi regolari che sono un sottoinsieme sono identificabili al limite come attesta il seguente risultato:
\begin{teorema}
La classe dei linguaggi primitivi ricorsivi è identificabile al limite mediante una presentazione completa degli esempi
\end{teorema}

Un altro risultato implicito che si evince dal lavoro di Gold è che non possiamo mai essere sicuri di apprendere il linguaggio corretto a meno che il numero di esempi non sia infinito. Con una presentazione finita l'unica cosa di cui possiamo essere certi è della consistenza della soluzione inferita con il \textit{training set} (la presentazione). Se il \textit{training set}  è grande aumenta la fiducia nell'approssimazione della soluzione inferita con il linguaggio target sconosciuto ma senza mai esserne certi e questa assunzione è nota come \textbf{Inductive Learning Hypothesis} \cite{Abela02}. Con il \textit{PAC-learning} si ottiene pure un'approssimazione del linguaggio target ma è possibile quantificare quanto vicini saranno la soluzione ottenuta ed il target e con quale probabilità.\\

Un altro problema che va affrontato è che tipicamente il numero di ipotesi consistenti  con gli esempi a disposizione è  infinito, quindi è necessario stabilire un \textbf{inductive preference bias} \cite{Abela02} cioè un criterio di preferenza da usare per selezionare quale delle congetture consistenti  rispetto ai dati scegliere. Nel caso dell'\ac{IIR}  l'\textit{inductive preference bias} che si adotta è il principio del \textbf{\textit{Rasoio di Occam}} che consiste nello scegliere sempre la soluzione più semplice che nell'\ac{IIR} significa scegliere il \ac{DFA} minimo\footnote{Si è soliti scegliere quasi sempre i \ac{DFA} come descrizione di un linguaggio regolare proprio perchè assicurano l'esistenza e l'unicità di un \ac{DFA} minimo}. Tale scelta da maggiori garanzie che il processo di generalizzazione abbia avuto luogo scongiurando il rischio di \textit{overfitting} cioè di un \ac{DFA} sovradattato agli esempi dati.

Il seguente risultato negativo è relativo a quest ultimo requisito di trovare il \ac{DFA} minimo:
\begin{teorema}
\label{teo:minhard}
Trovare il più piccolo automa finito deterministico consistente  rispetto a  un insieme di esempi completo è un problema NP-hard \cite{Gold78}.
\end{teorema}

\section{Passive learning}
 
Gli studi su \ac{GI} si possono suddivedere in due filoni, uno volto ad indagare i limiti teorici dell'apprendimento di linguaggi in determinati paradigmi ,come in \ref{sec:limIIR}, e l'altro volto a superare sotto opportune assunzioni i limiti teorici emersi rendendo gli algoritmi utilizzabili in termini di efficienza computazionale. Quest ultimo tema sarà affrontato adesso volgendo in particolare l'attenzione sugli algoritmi di \textbf{passive learning} e sulla strategia che li governa. La trattazione non sarà volutamente esaustiva dato che l'oggetto di studio di questo lavoro è l'\textit{active learning} che è una tecnica duale rispetto al \textit{passive learning}.

\subsection{Ricerca nel reticolo}
\label{sub:ricret}
I risultati negativi del teorema \ref{teo:minhard} sull'apprendimento del \ac{DFA} minimo possono essere superati delineando il problema di ricerca del  \ac{DFA} minimo come un problema di ricerca in un uno spazio degli stati in cui ogni stato è la modellazione di un \ac{DFA}. La strategia più semplice sarebbe quella enumerativa che ha come grafo di ricerca tutto lo spazio delle descrizioni e preleva da questo spazio l' ipotesi minima. Ma tale tecnica è impraticabile perchè tale spazio ha cardinalità infinita (basta pensare che solo i \ac{DFA} consistenti e completi con un dato \textit{training set} sono di per sè infiniti). Allora si definisce come grafo di ricerca il cosidetto \textbf{reticolo booleano} indicato con \textbf{$Lattice(PTA(S^+))$}che ha come nodo radice \textit{PTA$(S^+)$}. La radice può essere modellata come un insieme che comprende tutti gli stati dell'automa \textit{PTA$(S^+)$}, e l'operatore che consente di generare i nodi figli della radice (e  applicando ricorsivamente l'operatore anche ai figli ottenuti consente di derivare anche gli altri nodi) è una relazione d'ordine parziale  $\approx$ (cioè binaria transitiva,simmetrica e riflessiva) che applicata al \textit{PTA$(S^+)$} consente di derivare gli automi quoziente $PTA(S^+)/\!\!\approx$ che non sono altro che una fusione di alcuni degli stati del \ac{DFA} di partenza in base alla relazione d'ordine parziale definita. Inoltre l'automa quoziente $A/\!\!\approx$ ottenuto applicando la relazione $\approx$ sul nodo che è un automa $A$ è tale che $L(A) \subseteq L(A/\!\!\approx)$ quindi tutti gli automi ottenuti nel reticolo sono consistenti con $S^{+}$ ed estendendo il linguaggio da cui sono originati effettuano il processo induttivo generalizzando. Questa tecnica è detta di \textit{state-merging} in quanto ad ogni passo effettua una \textbf{fusione} degli stati dell'automa cui la relazione è applicata per ottenere un nuovo nodo e tutto il processo è noto in letteratura come \textbf{passive learning}.\\
Un aspetto che finora è stato trascurato riguarda se il \ac{DFA} minimo è sempre contenuto nel \textit{lattice}  altrimenti la procedura di ricerca potrebbe essere infruttuosa. Il seguente teorema \cite{Pao78} da una risposta affermativa se alcune condizioni sono verificate:

\begin{teorema}
Se l'insieme di istanze positive $S^{+}$ è strutturalmente completo rispetto al \ac{DFA} minimo target $M$ allora $M \in Lattice(PTA(S^{+}))$. 
\end{teorema}
laddove
\begin{definizione}[Insieme di esempi positivi strutturalmente completo]
\label{def:strcom}
Un insieme di esempi positivi $S^{+}$ è detto \textbf{strutturalmente completo} rispetto a un automa $M$ accettante $L$ se
\begin{itemize}
\item ogni stato di $\mathbb{F}_{\mathbb{A}}^{M}$ è impiegato da almeno una stringa di $S^{+}$
\item ogni transizione di $M$ è utilizzata nell'accettazione di almeno un esempio di $S^{+}$
\end{itemize}
\end{definizione}

Tuttavia la cardinalità del $Lattice(PTA(S^{+}))$ se $\norma{PTA(S^{+})} = n$ è :
\begin{equation*}
C_{n} = \sum_{i=0}^{n-1}\binom{n-1}{i}C_{i}  \quad \text{con } C_{0}=1
\end{equation*}
e rende impraticabile qualsiasi algoritmo di ricerca per enumerazione. Anche una ricerca in ampiezza è impraticabile. Inoltre  per avere la garanzia dell'automa minimo nel \textit{lattice} è necessario fare delle assunzioni, analogamente per effettuare una ricerca efficiente nel \textit{lattice} del \ac{DFA} minimo sarà necessario imporre dei vincoli.

\subsection{Algoritmi concreti} 
\label{sub:algcon}
 Qui si presentano le peculiarità dei due principali algoritmi di \textit{passive learning} in uno scenario di \textit{informed learning}. La trattazione ivi esposta è qualitativa e di alto livello ed i dettagli implementativi sono tralasciati.
 \subsubsection{Red-Blue Framework}
 \label{subsub:rbf}
 Questo framework riduce significativamente il numero di possibili \textit{merge} tra stati senza ridurre il numero di possibili soluzioni. Gli algoritmi con questo \textit{setting} mantengono un cuore di stati \textit{red} e una frontiera di stati \textit{blue} che sono gli stati immediatamente raggiungibili (figli diretti) a partire dagli stati \textit{red}. Un algoritmo di \textit{red-blue state-merging} esegue \textit{merges} solo tra stati \textit{blue} e stati \textit{red} e se non sono possibili \textit{red-blue merges} l'algoritmo \textbf{\textit{promuove}} uno stato \textit{blue} a \textit{red}. Gli stati \textit{red} inoltre sono gli stati del \ac{DFA} target che sono già stati identificati.
L'idea è stata descritta per la prima volta in \cite{Abbadingo98} in cui una prima versione dell'algoritmo \textit{EDSM}--- che valuta il \textit{merge} tra tutte le coppie di stati dell' ipotesi corrente \cite[p. 6]{Abbadingo98} ---  è migliorata nel \textit{Blue-fringe Algorithm} che secondo il principio di blanda località summenzionato valuta il merge solo tra stati di colore \textit{red} e \textit{blue}.\\
Sebbene la nomenclatura \textit{red-blue framework} non venga ufficialmente introdotta in \cite{Abbadingo98} con questo nome, è adottata in molti autorevoli testi e conferenze di riferimento su \ac{GI} come \cite{DeLaHiguera10} e  \cite[p. 70] {RBF10}.

\subsubsection{RPNI}
L'algoritmo \textit{ REGULAR POSITIVE  AND NEGATIVE INFERENCE} (RPNI) \cite{Oncina92} è uno dei più noti algoritmi di \textit{state-merging} che consente di trovare un \ac{DFA} --- in uno scenario di \textit{informed learning} --- consistente e completo con gli esempi $S = S^{+} \cup S^{-}$ in tempo polinomiale. E' un algoritmo esaustivo  che supera il problema della dimensionalità del reticolo booleano eseguendo una ricerca in profondità con backtracking a partire dal $PTA(S^{+})$ guidata dagli esempi negativi.\\
Se S è caratteristico per \ac{L} allora RPNI assicura che venga trovato il \ac{DFA} minimo:
\begin{definizione}[Insieme caratteristico]
\label{def:car}
 Un insieme di esempi completo $S = S^{+} \cup S^{-}$ è \textbf{caratteristico} per un linguaggio $L$ se:
\begin{itemize}
\item $S^{+}$ è \textit{strutturalmente completo} (definizione \ref{def:strcom})  per il \ac{DFA} che accetta $L$
\item  $S^{-}$ impedisce il \textit{merge} di due stati $p,q \text{ di un } \ac{DFA} \in Lattice(PTA(S^{+})) \text{tali che } p \not\equiv q$ durante un'esplorazione del lattice.
\end{itemize}
\end{definizione}
Se viene a mancare la condizione in \ref{def:car} di insieme caratteristico non è assicurata la terminazione con il \ac{DFA} minimo ma è comunque garantito un \ac{DFA} consistente rispetto agli \textit{esempi nel training set} grazie alla proprietà d'inclusione descritta  in \ref{sub:ricret}.

Il funzionamento a grandi linee di RPNI è:
\begin{enumerate}
\item Costruire $PTA(S^{+})$, numerare gli stati in base all'ordine lessicografico delle stringhe di $S^{+}$ e inizializzare l'insieme degli stati \textit{red} con lo stato corrispondente a $\epsilon$ e gli stati \textit{blue} con i suoi figli diretti.

\item Per ogni stato \textit{blue} si effettua sul \ac{DFA} corrente (inizialmente è $PTA(S^{+})$)il \textit{merge} con ogni stato \textit{red} estraendo gli stati in ordine lessicografico ottenendo un \ac{DFA} temporaneo $A$.
\begin{enumerate}
\item Se $A[S^{-}] \not\in \mathbb{F}_{\mathbb{A}}^{A}$ il \textit{merge} è valido, i restanti \textit{merge} dello stato \textit{blue} corrente con gli altri eventuali stati \textit{red} non vengono valutati, ed $A$ è il nuovo automa da considerare.
\item Se per il corrente stato \textit{blue} non si trova nessun \textit{merge} compatibile (con nessuno dei nodi \textit{red}), si esegue la sua \textit{promozione} a nodo \textit{red} e gli stati che sono figli diretti di questo nuovo nodo \textit{red} sono aggiunti ai nodi \textit{blue}
\end{enumerate}
\item Torna al punto 2 fin quando l'insieme degli stati \textit{blue} non è vuoto.
\end{enumerate}

Questa procedura può generare condizioni di non-determinismo cioè gli automi intermedi e finale prodotti possono essere degli NFA, per ovviare a questo inconveniente si può rendere il \textit{merge} una procedura ricorsiva che oltre a fondere gli stati rossi e blu effettua eventualmente delle ulteriori fusioni se la condizione di determinismo dell'automa creato dal corrente \textit{merge} non è posseduta. Vedasi \cite{DeLaHiguera10} per una versione di RPNI che genera sempre \ac{DFA}.
Infine il costo computazionale dell'algoritmo nella sua versione originale \cite{Oncina92} è $\mathcal{O}( (\norma{S^{+}}+\norma{S^{-}})\norma{S^{+}}^{2} )$

\subsubsection{EDSM}
\textit{EVIDENCE DRIVE STATE MERGING}  è un algoritmo di \textit{state-merging} \textit{euristico} che come RPNI non garantisce di trovare un \ac{DFA} canonico ma solo una soluzione ottima a meno che il \textit{training set} non sia caratteristico \ref{def:car}. L'algoritmo nella sua versione base è presentato in \cite{Abbadingo98} ed esistono delle versioni più efficienti come \textit{Blue-fringe EDSM} sempre in \cite{Abbadingo98} che usando il \textit{red-blue framework} limitano notevolemente il numero dei \textit{merge} e ne incrementano le \textit{performances}, ed è questo ultimo che viene brevemente spiegato qui.
Il miglioramento rispetto a RPNI è che EDSM non è \textit{greedy} nell'esecuzione dei \textit{merges} ma valuta tutti i merges possibili tra stati \textit{red} e \textit{blue} ---invece RPNI selezionerebbe il primo possibile e ignorerebbe gli altri \textit{merges} --- e si sceglie il \textit{merge} migliore in base ad un'euristica. La migliore euristica emersa nella competizione Abbandingo è quella di Price che valuta un \textit{merge} non possibile se almeno una stringa accettante e almeno una rigettante rispettivamente di $S^{+} \text{ e } S^{-}$ terminano nello stesso stato , ed ha invece un punteggio tanto più alto quante più stringhe di $S^{+}$ \text{ o di } $S^{-}$ (ma non di entrambi gli insiemi, quindi l'\textit{or} è esclusivo)  terminano in uno stesso stato del \ac{DFA} \textit{mergiato} che si sta valutando. Si sceglierà il \ac{DFA} col punteggio maggiore.\\
Un'altra differenza sostanziale rispetto a RPNI in EDSM è la priorità data alle \textit{promozioni} a scapito di possibili fusioni: il \textit{merge} viene effettuato soltanto se tutti i possibili \textit{merges} tra tutte le coppie di stati \textit{red} e \textit{blue} è possibile secondo l'euristica, altrimenti si privilegia la \textit{promozione} dello stato \textit{blue} a \textit{red}. 
EDSM costituisce lo stato dell'arte per ciò che riguarda gli algoritmi di \textit{passive learning}.


\section{Active Learning}
\label{sec:acl}
L' \textit{Active Learning} è un caso speciale di \textit{semi-supervisionato machine learning}\footnote{Nel semi-supervisionato learning una piccola quantità di dati è etichettata e la restante, la maggioranza, è senza etichetta.} in cui un algoritmo di \textit{learning} può interagire con l'utente o  qualche sorgente d'informazione per ottenere informazioni significative come ad esempio l'etichetta di un'istanza. Nel contesto dell' \ac{IIR} l' \textit{active learning} non esplora il reticolo costruito a partire dal PTA o dall' APTA come fanno gli algoritmi presentati in \ref{sub:algcon}  ma si basa su una stretta interazione tra il \textit{\textbf{learner}} e il \textit{\textbf{teacher}}  detto anche \textit{\textbf{Oracolo}} o \textit{\textbf{informant}}. L' \textit{active learning} è una tecnica duale rispetto al \textit{passive learning}: il \textit{passive learning} è un approccio \textit{top-down} che esplora lo spazio di ricerca a partire dal nodo iniziale (PTA o APTA costruito  dagli esempi iniziali) e tramite dei \textit{merges} genera nuovi nodi e nel caso degenere ha come ultimo nodo l'automa universale, invece l'\textit{active learning} effettua una ricerca \textit{bottom-up} che inizia dall'automa universale e mediante lo \textbf{\textit{split}} degli stati raffina l'ipotesi. Inoltre è il \textit{learner} che sceglie attivamente gli esempi e  il \textit{teacher} può selezionare attentamente i controesempi significativi: proprio per queste ragioni spesso il numero di esempi per apprendere un concetto e in generale il tempo di esecuzione del processo di apprendimento è minore negli algoritmi di \textit{active learning} che in quelli di \textit{passive learning}.   
L'\textit{active learning} nasce per ragioni teoriche come ad esempio per dimostrare che non è possibile apprendere in maniera efficiente alcune classi di linguaggi con una presentazione given-data: nell' \textit{active learning} nel cui contesto è il \textit{learner} che seleziona gli esempi  è sufficiente dimostrare che il numero di query non può essere polinomiale . Ma è anche applicabile in numerosi contesti pratici come la Robotica in cui un agente può costruire una mappa usando l'interazione tra i sensori e l'ambiente come \textit{Oracolo} o nella modellazione dell'acquisizione dei linguaggi naturali dove attribuire la figura del \textit{teacher} al genitore risulta naturale. Qui si esaminerà l' \textit{active learning} nell' \ac{IIR}. L* è senz altro il più noto algoritmo  di \textit{active learning} applicato ai linguaggi regolari. Il più recente e performante \textit{Observation Pack} sarà introdotto nel capitolo \ref{cap:quattro}
\subsection{Active learning nell' Inferenza Induttiva Regolare}
Il paradigma dell' \textit{active learning} si basa sull'esistenza di un \textit{Oracolo} che conosce \ac{L} e può rispondere solo a certi tipi di interrogativi sottopostigli dal \textit{learner}. L' \textit{Oracolo} può trovarsi in una situazione in cui più risposte valide sono possibili e in questo caso si deve assumere che non viene rispettata nessuna distribuzione di probabilità nelle risposte date ma che queste sono casuali pertanto nell'analisi dell'algoritmo si deve assumere il caso peggiore cioè un \textit{Oracolo} avverso (nell' algoritmo adoperato nell' \ac{IIR}, il table-filling descritto nella sottosezione \ref{sub:tea}, l'\textit{Oracolo} non garantisce di ritornare la \textit{\textbf{witness}} cioè il controesempio più breve).
I principali tipi di interrogativi possibili a cui si può sottoporre un \textit{Oracolo} sono:
\begin{itemize}
\item \textbf{\ac{MQ}} Una membership query è effettuata proponendo una stringa all'\textit{Oracolo}, che risponde YES se la stringa appartiene a \ac{L} e NO se la stringa non appartiene:\\\\
\centerline{$\text{\ac{MQ}} : \Sigma^{*}  \to \text{\{YES,NO\}}$}

\item \textbf{\ac{EQ}} (forte) Un'equivalence query (forte) è effettuata proponendo un \ac{DFA} ipotesi \ac{H} all'\textit{Oracolo} che risponde YES se il DFA ipotesi \ac{H} è equivalente al \ac{DFA} target altrimenti ritorna una stringa(\textit{witness}) appartenente alla differenza simmetrica tra \ac{L} e $L$(\ac{H}):\\\\
\centerline{\ac{EQ} : \textit{DFA}$ \,\to \text{\{YES\}} \cup \Sigma^{*}$}

\item \textit{\textbf{WEQ}} (debole) Un'equivalence query (debole) è effettuata proponendo un \ac{DFA} ipotesi \ac{H} all'\textit{Oracolo} che risponde YES se il DFA ipotesi \ac{H} è equivalente al DFA target altrimenti ritorna NO :\\\\
\centerline{\textit{WEQ} : \textit{DFA}$ \to  \text{\{YES,NO\}}$}

\item \textit{\textbf{SSQ}} Una subset query è effettuata proponendo un DFA ipotesi \ac{H} all'\textit{Oracolo} che risponde YES se $L$(\ac{H}) è un sottoinsieme di \ac{L} altrimenti ritorna una stringa appartenente a $L$(\ac{H}) che non appartiene ad \ac{L} :\\\\
\centerline{\textit{SSQ} : \textit{DFA}$ \to \text{\{YES\}} \cup \Sigma^{*}$}
\end{itemize}



I seguenti risultati e definizioni sono in \cite{Angluin90}.Si da la seguente definizione preliminare:
\begin{definizione}
Chiamiamo $\rho$ un'esecuzione del \textit{learner} A. Chiamiamo $\Braket{r_1,r_2,\dots r_m}$ la sequenza di risposte alle query $\Braket{q_1,q_2,\dots q_m}$ che l'\textit{Oracolo} fa durante l'esecuzione $\rho$ . Si dice che \textbf{A} è \textbf{polinomialmente  limitato} se esiste un polinomio a due variabili p() che dato qualsiasi formalismo L descrivente  \ac{L} e in qualsiasi esecuzione $\rho$, e a qualsiasi \textit{query point}(indica il momento prima che avvenga una query ed è definito come un numero intero che specifica il numero di query avvenute fino a quel momento) $k$ dell'esecuzione, denotando il tempo di esecuzione prima di quel punto con $t_k$, si ha:
\begin{itemize}
\item $k \le p(\norma{L} , \text{max}\{\abs{r_i} : i \textless k \})$
\item $\abs{q_k} \le p(\norma{L} , \text{max}\{\abs{r_i} : i \textless k \})$
\item $t_k \in \mathcal{O}(p(\norma{L} , \text{max}\{\abs{r_i} : i \textless k \}))$
\end{itemize}
\end{definizione}
Informalmente significa che in qualsiasi \textit{query point k} di qualunque esecuzione, al momento precedente l'effettuazione della query $q_k$, si ha che il numero di query fatte, il tempo di esecuzione e la dimensione della prossima query ($q_k$) sono tutte limitate da un polinomio $p$ dipendente dalla dimensione del target e dalla lunghezza del più lungo controesempio ritornato dall'\textit{Oracolo} fino a quel punto
   
La seguente definizione stabilisce quando una classe di linguaggi è efficientemente \textit{identificabile in the limit} da un algoritmo di  \textit{learning}. 
\begin{definizione}
\label{def:pol}
Una classe di linguaggi $\mathfrak{L}$ è \textbf{polinomialmente \textit{identificabile in the limit con query}} fissati i tipi di query possibili se esiste un \textbf{polinomialmente limitato} \textit{learner} \textbf{A} che dato il formalismo descrivente qualsiasi linguaggio $L \in  \mathfrak{L}$, identifica L in the limit, cioè ritorna  L' equivalente ad L e termina.
\end{definizione}

Adesso ci si chiede se la classe dei linguaggi regolari è polinomialmente identificabile in the limit tramite qualche algoritmo di apprendimento secondo la definizione \ref{def:pol}. E' importante sottolineare che la risposta a questa domanda dipende anche dalla classe cui appartiene l'\textit{Oracolo} cioè dal tipo di interrogativi che è possibile rivolgergli.  A tal proposito si hanno i seguenti risultati :
\begin{teorema}
\label{teo:noi}
La classe dei linguaggi regolari  non è polinomialmente identificabile in the limit da un numero polinomiale di \ac{MQ}, \textit{WEQ} e \textit{SSQ}
\end{teorema}
Quindi come conseguenza del teorema \ref{teo:noi}   la classe dei linguaggi regolari non è polinomialmente identificabile in the limit  neanche sottoponendo all'\textit{Oracolo} esclusivamente \ac{MQ} .

Un'ulteriore risultato è il seguente: 
\begin{teorema}
\label{teo:noe}
La classe dei linguaggi regolari (usando i \ac{DFA} come formalismo descrittivo)  non è polinomialmente identificabile in the limit da un numero polinomiale di \ac{EQ} (forti)
\end{teorema}
Si rimanda alla sezione \ref{sec:lstar} per le condizioni di polinomiale identificabilità in the limit dei linguaggi regolari
\section{L*}
\label{sec:lstar}
L* è il più noto algoritmo di \textit{active learning} nell'ambito dell' \ac{IIR} e garantisce di emettere in output il DFA minimo ( o uno ad esso isomorfo ) accettante \ac{L}. Detto n il numero degli stati del DFA target minimo ed m la lunghezza del più lungo controesempio ritornato dall' \textit{Oracolo} durante l'inferenza, il costo computazionale di L* sarà limitato da una funzione polinomiale di n ed m. In L* il \textit{teacher} appartiene alla classe dei \ac{MAT} in grado di rispondere ad \ac{EQ} e \ac{MQ}. Questi risultati ,che consentono di dire che i linguaggi regolari sono polinomialmente identificabili in the limit (definizione \ref{def:pol}),  sono stati conseguiti da Dana Angluin \cite{Angluin87} e succintamente riportati nel seguente teorema:
\begin{teorema}
Dato un \ac{MAT}  presentante un linguaggio regolare sconosciuto U, il Learner L* termina restituendo in output un automa finito isomorfo al DFA minimo accettante  il linguaggio target U. Inoltre, se n è il numero di stati del DFA minimo accettante U e  m è un limite superiore della lunghezza di ogni controesempio ritornato dal Teacher, allora il costo totale di esecuzione di L* è limitato da un polinomio in n ed m 
\end{teorema}

L* viene presentato all'interno del \textit{red--blue framework} (introdotto in \ref{sub:algcon}) che in algoritmi come \textit{EDSM} (vedasi sottosezione \ref{sub:algcon}) consente di diminuire i \textit{merges}. In L* l'adozione di questo \textit{framework} malgrado non comporti un vantaggio computazionale consente un'esposizione più chiara.  
\subsection{Tabella di Osservazione}
\label{sub:obt}
Una \textbf{tabella di osservazione} è una struttura dati che rappresenta il DFA ipotesi congetturato al passo corrente. Al suo interno sono codificati gli esiti delle \ac{MQ} richieste al \textit{teacher}.

\begin{definizione*}[Tabella di Osservazione] La \textit{tabella di osservazione} è una tripla $\Braket{STA, EXP, OT}$, dove:
\begin{itemize}
\item $\text{STA}=\text{RED} \cup \text{BLUE}$.  STA è un insieme finito di stringhe definite su $\Sigma$ che rappresentano gli stati. STA è \textbf{prefix--closed}\\
RED $\in \Sigma^{*}$ è un insieme finito di stati\\
$\text{BLUE} = \{ua \notin \text{RED} : u \in \text{RED}\}$ è l'insieme dei successori degli stati RED che non sono RED. Rappresentano le transizioni.
\item EXP $\in \Sigma^{*}$  è l'insieme degli esperimenti. E' \textbf{suffix--closed}
\item $\text{OT} : \text{STA} \times \text{EXP} \to \text{\{0,1,*\}}$ è una funziona così definita:\\\\
\centerline{$
OT[u][e] = 
\begin{cases}
1
& \text{se $ue \in \ac{L}$} \\
0 & \text{se $ue \notin \ac{L}$}\\
* & \text{altrimenti}
\end{cases}
$}   
\end{itemize}
\end{definizione*} 
Dalla tabella di osservazione si costruisce una nuova ipotesi e la si sottopone al \textit{teacher}. Se l'ipotesi non è equivalente al \textit{DFA target} il \textit{teacher} torna un controesempio che sarà usato dal \textit{learner} per \textit{splittare} gli stati e modificare la tabella di osservazione per ottenere una nuova ipotesi cofacente al controesempio. Le \ac{MQ} permettono di riempire i buchi generati dall'introduzione di nuovi prefissi dal controesempio.  A partire dalla tabella di osservazione è possibile costruire un DFA ipotesi sole se questa gode di tre proprietà:\\

{\large\textbf{Completezza}}

La completezza garantisce che non ci siano comportamenti parzialmente (o totalmente) sconosciuti per prefissi presenti all'interno della tabella.
\begin{definizione*}[Tabella completa] Una tabella è completa se non ha \textit{buchi}. Un \textit{buco} in una tabella di osservazione è una coppia (u,e) tale che $OT[u][e] = *$.
\end{definizione*}
L'eventuale incompletezza può essere eliminata mediante \ac{MQ} al \textit{teacher}.\\

{\large\textbf{Chiusura}}

La chiusura (algoritmo \ref{alg:lstar-close}) assicura che ogni possibile stato raggiunto con una transizione sia presente tra gli stati finali dell'automa. Dato un elemento $s \in$ STA e gli \textit{n} esperimenti $e \in $ EXP si indica con row(s) la riga in OT indicizzata da s cioè $row(s)=OT[s][e_1] \cdot OT[s][e_2] \cdot \dots OT[s][e_n]$ . Gli stati dell'automa sono un sottoinsieme degli stati RED, quando una transizione da uno stato RED porta ad uno stato BLUE  si deve trovare uno stato RED equivalente a quello BLUE (vedasi algoritmo \ref{alg:lstar-buildautomaton}) (almeno secondo i suffissi trovati fino a quel momento) in modo che la transizione arrivi in questo stato RED(che è presente nell'automa ipotesi perchè gli stati RED trovati fanno parte del \ac{DFA} ipotesi a differenza di quelli BLUE).
\begin{definizione*}[Tabella chiusa] Una tabella è \textbf{chiusa} se $\forall u \in \text{BLUE}, \exists s\\ \in \text{RED} : \text{row(u)} = \text{row(s)}$
\end{definizione*}
Se la tabella di osservazione non fosse chiusa è possibile renderla tale mediante una (o più) \textbf{promozione}, cioè l'inserimento di \textbf{u} responsabile della non chiusura nei RED e $u \cdot \Sigma$ nei BLUE\\

{\large\textbf{Consistenza}}

La consistenza  (algoritmo \ref{alg:lstar-consistent}) impedisce situazioni di indeterminismo nel DFA, nella fattispecie che da uno stato dell'ipotesi per uno stesso simbolo dell'alfabeto si giunga in stati di arrivo diversi. Questa situazione è resa possibile dal fatto che l'algoritmo non impedisce di avere due stati RED $s_1$ ed $s_2$ tali che $\text{row}(s_1) = \text{row}(s_2)$ . 
\begin{definizione*}[Tabella consistente]Una tabella di osservazione è \textbf{consistente} se $\forall s_1,s_2 \in \text{RED} : \text{row}(s_1)=\text{row}(s_2)\implies\forall a \in \Sigma,\text{row}(s_1a)=\text{row}(s_2a)$
\end{definizione*} 

La definizione sopra significa che affinchè vi sia consistenza ogni coppia di stati equivalenti RED cioè di stati in RED con righe uguali deve restare equivalente in STA aggiungendo qualsiasi simbolo dell'alfabeto. 
Se la tabella di osservazione non fosse consistente è possibile renderla tale ampliando l'insieme EXP con la stringa ottenuta dalla concatenazione del simbolo dell'alfabeto e dall'esperimento che hanno generato l'inconsistenza. Ciò assicura che i due stati $s_1$ ed $s_2$ che prima erano equivalenti (e che quindi rappresentavano un unico stato nell'ipotesi) adesso non lo sono più perchè  $\text{row}(s_1) \neq \text{row}(s_2)$ e quindi sarà aggiunto un nuovo stato all'insieme RED cioè un nuovo stato all'ipotesi.

\begin{algorithm}
\caption{LSTAR-BUILDAUTOMATON}\label{alg:lstar-buildautomaton}
\begin{algorithmic}[1]
\Statex
\Input a closed and complete observation table $\Braket{\text{STA,EXP,OT}}$
\Output DFA $\Braket{\Sigma,Q,q_\epsilon,F_A,F_R,\delta}$
\State $Q \gets \{q_u : u \in \text{RED} \land \forall v < u \: \text{row}(v) \ne \text{row}(u)\}$
\LineComment{le stringhe più corte sono minori e per stringhe della stessa lunghezza si intendono minori quelle che lessicograficamente vengono prima}
\State $F_{\mathbb{A}} \gets \{q_u \in Q : \text{OT}[u][\epsilon] = 1\}$
\State $F_{\mathbb{R}} \gets \{q_u \in Q : \text{OT}[u][\epsilon] = 0\}$
\For{$q_u \in Q$}
     \For{$a \in \Sigma$} $\delta(q_u,a) \gets q_w \in Q : \text{row}(ua) = \text{row}(w)$
     \EndFor
\EndFor
\State \textbf{end for}
\State \Return{$\Braket{\Sigma,Q,q_\epsilon,F_A,F_R,\delta}$}
\end{algorithmic}
\end{algorithm}

\begin{algorithm}
\caption{LSTAR}\label{alg:lstar}
\begin{algorithmic}[1]
\Statex
\Input --
\Output DFA $\mathcal{A}$
\State \Call{LSTAR-INITIALISE}{}
\Repeat
     \While{$\Braket{\text{STA,EXP,OT}} \textit{is not closed or not consistent}$}
          \If{$\Braket{\text{STA,EXP,OT}} \textit{is not closed}$}
          \vspace{0.5mm}
               \State $\Braket{\text{STA,EXP,OT}} \gets \Call{LSTAR-CLOSE}{\Braket{\text{STA,EXP,OT}}} $
          \EndIf
          \If{$\Braket{\text{STA,EXP,OT}} \textit{is not consistent}$}
          \vspace{0.5mm}
               \State $\Braket{\text{STA,EXP,OT}} \gets \Call{LSTAR-CONSISTENT}{\Braket{\text{STA,EXP,OT}}} $ 
          \EndIf    
     \EndWhile
     \State \textbf{end while}
     \State $\text{Answer} \gets \Call{EQ}{\Braket{\text{STA,EXP,OT}}}$
     \If{$\text{Answer} \ne \text{YES}$}
          \vspace{0.25mm}
          \State $\Braket{\text{STA,EXP,OT}} \gets \Call{LSTAR-USEEQ}{\Braket{\text{STA,EXP,OT}},\text{Answer}}$
     \EndIf
\Until{$\text{Answer} = \text{YES}$}
\State \textbf{return} \Call{LSTAR-BUILDAUTOMATON}{$\Braket{\text{STA,EXP,OT}}$}
\end{algorithmic}
\end{algorithm}

\begin{algorithm}
\caption{LSTAR-INITIALISE}\label{alg:lstar-initialise}
\begin{algorithmic}[1]
\Statex
\Input --
\Output $\Braket{\text{STA,EXP,OT}}$
\State $\text{RED} \gets \{q_\epsilon\}$
\State $\text{BLUE} \gets \{q_a : a \in \Sigma\}$
\State $\text{EXP} \gets \{\epsilon\}$
\State $\text{OT}[\epsilon][\epsilon] \gets \Call{MQ}{\epsilon}$
\vspace{1mm}
\For{$a \in \Sigma$} $\text{OT}[a][\epsilon] \gets \Call{MQ}{a}$
\EndFor
\State \textbf{return} $\Braket{\text{STA,EXP,OT}}$
\end{algorithmic}
\end{algorithm}

\begin{algorithm}
\caption{LSTAR-CLOSE}\label{alg:lstar-close}
\begin{algorithmic}[1]
\Statex
\Input $\Braket{\text{STA,EXP,OT}}$
\Output $\Braket{\text{STA,EXP,OT}}$ updated
\For{$s \in \text{BLUE} \: \textit{such that} \: \forall u \in \text{RED} \: \text{row}(s) \ne \text{row}(u)$}
\LineComment{Non per $\forall s$ ma per uno solo quindi la tabella in output può ancora essere non chiusa}
     \State $\text{RED} \gets \text{RED} \cup \{s\}$
     \State $\text{BLUE} \gets \text{BLUE} \: \setminus \: \{s\}$
     \For{$a \in \Sigma$} BLUE $\gets \text{BLUE} \cup \{s\cdot a\}$
     \EndFor
     \For{$u, e \in \Sigma^{*}  \: \textit{such that} \: \text{OT}[u][e] \textit{is a hole}$} $\text{OT}[u][e] \gets \Call{MQ}{ue}$
\EndFor
\EndFor
\State \textbf{end for}
\State \textbf{return} $\Braket{\text{STA,EXP,OT}}$
\end{algorithmic}
\end{algorithm}

\begin{algorithm}
\caption{LSTAR-CONSISTENT}\label{alg:lstar-consistent}
\begin{algorithmic}[1]
\Statex
\Input $\Braket{\text{STA,EXP,OT}}$
\Output $\Braket{\text{STA,EXP,OT}}$ updated
\State find $s_1, s_2 \in$ RED, $a \in \Sigma$ and $e \in$ EXP \textit{such that} row$(s_1) = \text{row}(s_2)$ and
\State $\text{OT}[s_1 \cdot a][e] \ne \text{OT}[s_2 \cdot a][e]$
\LineComment{se $s_1a \: \text{ed} \: s_2a$ differiscono per più di un esperimento basta considerarne uno}
\State $\text{EXP} \gets \text{EXP} \cup \{a \cdot e\}$ 
\For{$u, e \in \Sigma^{*}  \: \textit{such that} \: \text{OT}[u][e] \textit{is a hole}$} $\text{OT}[u][e] \gets \Call{MQ}{ue}$
\EndFor
\State \textbf{return} $\Braket{\text{STA,EXP,OT}}$
\end{algorithmic}
\end{algorithm}

\begin{algorithm}
\caption{LSTAR-USEEQ}\label{alg:lstar-useeq}
\begin{algorithmic}[1]
\Statex
\Input $\Braket{\text{STA,EXP,OT}} \text{ ,  Answer}$
\Output $\Braket{\text{STA,EXP,OT}}$ updated
\For{$p \in \text{PREF(Answer)}$}
\Comment{Anche Answer fa parte dei prefissi}
     \State $\text{RED} \gets \text{RED} \cup \{p\}$
\Comment{Se un pref. è già in OT renderlo RED se non lo è}
     \For{$a \in \Sigma : pa \notin \text{PREF(Answer)}$} $\text{BLUE} \gets \text{BLUE} \cup \{pa\}$ 
     \EndFor 
\EndFor
\State \textbf{end for}
\For{$u, e \in \Sigma^{*}  \: \textit{such that} \: \text{OT}[u][e] \textit{is a hole}$} $\text{OT}[u][e] \gets \Call{MQ}{ue}$
\EndFor
\State \textbf{return} $\Braket{\text{STA,EXP,OT}}$
\end{algorithmic}
\end{algorithm}

\subsection{L'algoritmo}
\subsubsection{Funzionamento}La \textit{ratio} che ispira L* é il teorema \textit{Myhill-Nerode} (sezione \ref{teo:m-n}). Nella tabella di osservazione le righe RED,in realtà un sottoinsieme delle stringhe RED, corrispondono agli stati del DFA ipotesi e le colonne, l'insieme EXP, corrispondono alle stringhe rappresentanti i suffissi che distinguono coppie di stati distinti dell'ipotesi. I singoli stati sono etichettati dalle stringhe che portano dallo stato iniziale allo stato stesso. Lo stato iniziale è etichettato dalla stringa $\epsilon$. Per ogni stato $q$ l'etichetta della colonna (un esperimento) indica lo stato che potrebbe essere raggiunto a partire dallo stato $q$ dopo la lettura della stringa corrispondente all'etichetta dell'esperimento. Due stati sono considerati equivalenti se hanno le righe uguali nella tabella.

L'algoritmo inizia costruendo una tabella di osservazione corrispondente all'automa universale (algoritmo \ref{alg:lstar-initialise})
Una volta resa la tabella completa,chiusa e consistente viene estratta l'ipotesi corrispondente\footnote{La prima ipotesi fatta dal \textit{learner} non è sempre l'automa universale, ma dipende dall'esito delle \ac{MQ}} e viene effettuta un'\ac{EQ} dell'ipotesi al \textit{teacher}. In caso di risposta affermativa cioè di equivalenza il processo termina in quanto il \textit{learner} ha identificato un automa uguale o equivalente al \textit{target} (L* inferisce il DFA minimo, invece il \textit{target} potrebbe non essere un DFA minimo). In caso contrario sarà ritornato un controesempio che sarà usato per modificare la tabella di osservazione e quindi formulare una nuova ipotesi. L'algoritmo \ref{alg:lstar} chiarifica e approfondisce i passaggi summenzionati.

\subsubsection{Correttezza}
\label{sub:cor}
Per verificare che L* è corretto è sufficiente dimostare che termina dato che la terminazione con un  \ac{MAT} assicura l'equivalenza dell'ipotesi col target.
A tal fine si riporta preliminarmente il seguente teorema (\cite{Angluin87}):
\begin{teorema}
\label{teo:angm}
Se una tabella di osservazione è completa,chiusa e consistente, allora l'ipotesi \ac{H} (quella inferita dall'algoritmo \ref{alg:lstar-buildautomaton}) è consistente con la funzione OT. Qualsiasi altra ipotesi consistente con OT ma non equivalente ad \ac{H} deve avere più stati.
\end{teorema}
Nel teorema \ref{teo:angm} con ipotesi consistente con la funzione OT si intende che $\forall u \in \text{STA e } \forall e \in \text{EXP},  ue \in \ac{L} \iff \text{OT}[u][e] = 1$

Il risultato del teorema \ref{teo:angm} è che qualunque DFA consistente con OT o è isomorfico all'ipotesi inferita da L* o contiene almeno uno o più stati. Quindi ogni ipotesi \ac{H} fatta da L* è sempre il minimo DFA consistente con OT.

Un altro risultato che ci torna utile è:
\begin{lemma}
\label{lem:fgv}
Detto n il numero di differenti valori di $row(s) \text{ per } \forall s \in \text{RED}$ in una tabella di osservazione. Qualsiasi \ac{H} consistente con OT deve avere almeno n stati
\end{lemma}
Si indica con n il numero di stati del DFA minimo accettante \ac{L}. E' facile dimostrare che il numero di valori distinti di $row(s) \text{ per } s \in \text{ RED} $ è incrementato monotonicamente fino ad un massimo di n durante l'esecuzione di L*. Infatti sia la chiusura che la consistenza introducono un nuovo stato RED. Se la tabella fosse già chiusa e consistente il controesempio $t$ tornato dal \textit{teacher} comunque garantisce che un nuovo stato RED venga aggiunto alla tabella di osservazione: detto $T_u$ il DFA \textit{target} minimo (ovviamente consistente con OT) --- il controesempio $t$ ci permette di dedurre che \ac{H} e  $T_u$  non sono equivalenti ---  , allora dal teorema \ref{teo:angm} sappiamo che \ac{H} ha al massimo $n-1$ stati. Inoltre L* classificherà il controesempio $t$ allo stesso modo di $T_u$ e quindi il nuovo DFA ipotesi \textit{H'} che  si otterrà  non sarà equivalente ad \ac{H} (per via di t) e inoltre sarà consistente con OT, quindi dal teorema \ref{teo:angm}  si deduce che \textit{H'} deve avere almeno n stati (almeno 1 stato in più di \ac{H}).

Questo dimostra che ad ogni passo del ciclo più esterno di L* almeno uno stato RED distinto deve essere aggiunto sempre alla tabella di osservazione. Quando ci saranno n stati RED distinti L* troverà il DFA \textit{target} minimo consistente con OT infatti il DFA \textit{target} minimo è sempre consistente con OT\footnote{Ma se sia il \textit{target} che \ac{H} sono sempre consistenti con OT come si fa a trovare un controesempio? La risposta è semplice: da OT viene creata un' ipotesi \ac{H} consistente all'OT ma nell'ipotesi possono essere \textit{parsate} anche altre stringhe non contemplate in OT da cui può derivare la non equivalenza con il \textit{target}} e l'ipotesi creata da L* è sempre il DFA minimo per il teorema \ref{teo:angm} e dal lemma \ref{lem:fgv} si ha che l'ipotesi creata da L* deve avere almeno n stati.
Essendo il DFA ipotesi un DFA ipotesi minimo e con n stati e consistente con T non può essere che uguale o isomorfo col DFA \textit{target minimo}.  Quindi L* dovrà costruire nel caso peggiore $n-1$ ipotesi errate prima di trovare l'ipotesi corretta e quindi il numero di \ac{EQ} è al massimo n (perchè l'ipotesi corretta comunque va sottoposta al \textit{teacher})

\subsubsection{Complessità computazionale}\label{subsub:comcom}Come detto in precedenza la complessità computazionale di L* è limitata da un polinomio dipendente dal numero di stati del DFA minimo identificante \ac{L} e dalla lunghezza del controesempio più lungo ritornato dal \textit{teacher}.In \cite{Angluin87} si trova una dimostrazione dettagliata di quanto detto sopra. In questa sede si analizzano dei parametri oggettivi nella valutazione del costo computazionale cioè il numero di \ac{MQ} e di \ac{EQ}. In quest'analisi si terrà conto anche di k un ulteriore parametro che rappresenta la dimensione dell'alfabeto. Il numero di \ac{EQ} sarà limitato da n (sottosezione correttezza \ref{sub:cor}) come dimostrato in precedenza. Il numero di \ac{MQ} è invece limitato dalla dimensione della tabella di osservazione. Il numero di elementi in EXP non può eccedere n , in quanto l'insieme EXP viene incrementato di un elemento quando la tabella di osservazione è inconsistente (algoritmo \ref{alg:lstar-consistent}) e l'inconsistenza può presentarsi al più $n-1$ volte cioè ogni volta che viene aggiunto un nuovo nodo RED distinto (con $n$ nodi RED distinti L* termina quindi se ne aggiungono al più $n-1$) (la dimensione di EXP è al più n e non $n-1$ perchè EXP inizialmente contiene $\lambda$).EXP rappresenta il numero di colonne della tabella di osservazione, adesso si calcola il numero di righe della stessa. Si indica con $m$ la lunghezza del controesempio più lungo ritornato dal \textit{teacher}. Il numero di stati RED non può eccedere $n+m(n-1)$  perchè gli stati RED sono aggiunti quando si scopre che la tabella non è chiusa e quando il \textit{teacher} torna un controesempio.  La non chiusura può accadere al più $n-1$ volte ed ogni volta aggiunge uno stato RED, e ci possono essere massimo $n-1$ controesempi ognuno dei quali può causare l'aggiunta di al più $m$ stati RED (i prefissi aggiunti ad ogni iterazione sono al più pari alla lunghezza del controesempio e nel caso peggiore in cui tutti i controesempi sono lunghi m il numero dei prefissi aggiunti ad ogni iterazione è $m$). Il numero degli stati BLUE è al più $k(n+m(n-1))$ perchè i BLUE, che rappresentano le transizioni, sono ottenuti concatenando tutti i simboli dell'alfabeto agli stati RED. Quindi la dimensione della tabella sarà $\text{righe} * \text{colonne}$ cioè $\text{(RED}+\text{BLUE)}*\text{EXP}$ quindi si ha:
\begin{equation*}
(k+1)(n+m(n-1))\,n = \mathcal{O}(kmn^{2})
\end{equation*}
che è il numero di \ac{MQ} totali.
\subsection{Il teacher} 
\label{sub:tea}Il teacher di L* essendo un \ac{MAT} è chiamato a rispondere a due tipi di query: \ac{MQ} ed \ac{EQ}. Si suppone che esso abbia a disposizione il DFA che identifica \ac{L} quindi è immediato rispondere a una \ac{MQ}. Il \textit{teacher} deve anche vagliare l'equivalenza del target con l'ipotesi fornitagli dal \textit{learner} e in caso di inequivalenza deve tornare un controesempio. A tal fine il \textit{table-filling algorithm} \cite{Nor09} risulta essere un buon algoritmo (il più performante con complessità quasi lineare atto solo a testare l'equivalenza e tornare una witness  quindi senza consentire anche la minimizzazione è \cite{Hop71} )
\subsubsection{Table-filling}Il \textit{table-filliling} \cite{Nor09} è un algoritmo in grado di individuare ricorsivamente tutti gli stati tra loro distinti, alla fine dell'esecuzione le coppie di stati non marcati come tali saranno coppie di stati equivalenti. Per questo motivo l'algoritmo di table-filling è utilizzato anche nella minimizzazione di DFA dove gli stati trovati equivalenti saranno fusi in un unico stato.

Per stati distinti si intende stati per cui esiste almeno una stringa che partendo da quei due stati (e non dallo stato iniziale) giunge in una coppia di stati di arrivo composta da uno stato accettante e da uno stato rigettante.

Inizialmente si distinguono le coppie di stati che non sono equivalenti cioè gli stati che vengono distinti dalla stringa vuota cioè quelle coppie di stati formate da uno stato accettante e da uno rigettante. Al passo successivo si procede esaminando tutte le coppie di stati che momentaneamente l'algoritmo considera equivalenti (che non ha marcato come distinti nei passi precedenti): se per un simbolo dell'alfabeto $s$ da quella coppia di stati di partenza si arriva a una coppia di stati distinti (già marcati dall'algoritmo nei passi precedenti) anche la coppia di stati di partenza va marcata come distinti perchè se gli stati di arrivo sono distinti vuol dire che esiste un suffisso $w$ che li distingue quindi gli stati di partenza saranno distinti dalla stringa $sw$. Questa procedura va ripetuta ed ha termine quando al passo corrente l'algoritmo non ha trovato nessuna nuova coppia di stati distinti.Inoltre come attesta il seguente teorema:
\begin{teorema*}
Se due stati non sono marcati come distinti dall'algoritmo di table-filling, allora questi stati sono equivalenti. \cite{Hop07}
\end{teorema*}
si ha che  quando vengono identificati gli stati equivalenti è possibile passare alla minimizzazione tramite il merge di questi stati.

Per testare l'equivalenza di due DFA si manda in esecuzione il \textit{table-filling} sul DFA costituito dall'unione dei due DFA di cui verificare l'equivalenza. Se al termine dell'esecuzione i due stati iniziali risultano equivalenti i due DFA di partenza saranno equivalenti perchè non esiste nessuna stringa che distingue i due stati iniziali e quindi i due linguaggi dei due DFA sono identici. Per ottimizzare l'esecuzione è possibile interrompere l'esecuzione non appena viene individuato che i due stati iniziali sono distinti.

Per abilitare il \textit{teacher} a ritornare, in caso di inequivalenza, una \textit{witness} è necessario modificare leggermente il \textit{table-filling}. Anzichè limitarsi nel marcare le coppie di stati non equivalenti è necessario anche memorizzare il simbolo dell'alfabeto che ha causato l'inequivalenza. Inoltre bisogna marcare con la stringa vuota o con un marcatore speciale le coppie di stati distinti in fase d'inizializzazione . Quando l'algoritmo termina è possibile creare il controesempio, partendo dalla coppia di stati iniziali, percorrendo la struttura dati usata durante il \textit{table-filling} con l'ausilio delle funzioni di transizione dei due DFA e del marcatore memorizzato fino a quando non viene trovata una coppia di stati contrassegnata con la stringa vuota (cioè uno stato accettante e l'altro no). E' garantito che un controesempio venga sempre individuato ma non vi è la garanzia che ad essere scovato sia quello di lunghezza minima.
\subsubsection{Una versione più efficiente}
Il \textit{table-filling} ha una complessità polinomiale rispetto ad n cioè alla somma del numero degli stati dei DFA di cui si vuole testare l'equivalenza. Si avranno $n*\frac{n-1}{2}$  coppie distinte di stati che verranno tutte considerate ad ogni visita della tabella (la struttura dati mantenuta dall'algoritmo) quindi $\mathcal{O}(n^{2})$. Nel caso peggiore ad ogni iterazione una sola coppia verrà scoperta essere distinta e le coppie sono tutte distinte il numero di iterazioni sono al più  $\mathcal{O}(n^{2})$ . Moltiplicando le due complessità si ottiene $\mathcal{O}(n^{4})$ che è il costo computazionale nel caso peggiore.

E' possibile migliorare la complessità computazionale a $\mathcal{O}(n^{2})$ memorizzando per ogni coppia di stati $(i,j)$ una lista di dipendenza costituita da tutte quelle coppie (x,y) che tramite un singolo simbolo dell'alfabeto $k$ arrivano in $(i,j)$ cioè $\hat{\delta}(x,k)=i$ e $\hat{\delta}(y,k)=j$ .
Si memorizzano in una coda tutte le coppie di stati inizialmente distinte. Si estrae una coppia dalla coda (che quindi è distinta) e tutte le coppie che da essa dipendono sono marcate come distinte nella tabella e sono aggiunte in fondo alla coda. L'algoritmo ripete questi passi finchè la coda è vuota.
%!TEX encoding = UTF-8 Unicode
%!TEX root = ./../main.tex
%!TEX TS-program = xelatex

\chapter{Observation Pack} % 4th chapter title
\label{cap:quattro}
Esistono molte varianti dell'algoritmo L* originariamente presentato da Angluin. L'algoritmo \ac{ObP} presentato da Falk Howar in \cite{Howar12} --presentato per le Mealy Machines ma comunque applicabile ai DFA -- si basa su alcune di queste varianti. Tecnicamente l'\ac{ObP} combina l'idea di usare un \ac{DT} per i linguaggi regolari \cite{Kearns94} con una versione localizzata della tabella di osservazione \cite{Schapire93}. Inoltre viene utilizzata una tecnica più efficiente di gestione del controesempio rispetto ad L*. Quindi l'\ac{ObP}  è in stretta correlazione con L*, e molti dei concetti introdotti nel capitolo \ref{cap:tre} rimangono validi. Anche \ac{ObP}, come L* è un algoritmo di \textit{active learning} nell'ambito dell'\ac{IIR} che garantisce di emettere in output il \ac{DFA} minimo ( o uno ad esso isomorfo ) accettante \ac{L}. Anche in questo caso detto n il numero degli stati del DFA target minimo ed m la lunghezza del più lungo controesempio ritornato dal \textit{teacher} durante l'inferenza, il numero di \ac{MQ} sarà limitato da una funzione polinomiale di n ed m e il numero di \ac{EQ} linearmente da n. Nonostante L* sia il più noto algoritmo di \textit{active learning} l'\ac{ObP} ottiene prestazioni migliori dato che permette di diminuire il numero delle \ac{MQ}. Nell' \ac{ObP} il teacher appartiene ancora alla classe dei \ac{MAT}, in grado di rispondere ad \ac{EQ} e \ac{MQ}.
\section{Fondamenta teoriche}
Si parla di \textit{active learning} perchè il \textit{learner} attivamente può interrogare il \textit{teacher} sull'appartenenza o meno di alcune stringhe ad \ac{L}, in contrapposizione al \textit{passive learning} in cui le stringhe sono date a priori come in \textit{EDSM} ad esempio. Il \textit{learner} da un certo punto di vista si trova ad affrontare un problema di \textit{classificazione} cioè deve assegnare alcune stringhe ad un determinato stato di \ac{H}. L'ipotesi \ac{H}  ottenuta induttivamente sarà consistente con l'etichettatura degli esempi sottoposti al \textit{teacher}  fino al momento della creazione di \ac{H} ma produrrà una generalizzazione dato che in \ac{H} è possibile effettuare il \textit{parsing} di stringhe mai sottoposte al \textit{teacher}. Detto T il \ac{DFA} target,la \textit{classificazione} delle stringhe e la costruzione dell'ipotesi si basa sui risultati del teorema Myhill-Nerode (teorema \ref{teo:m-n}) che consente di:
\begin{enumerate}
\item Trovare un insieme $Sp \subset \Sigma^{*}$ di \textbf{prefissi}, detti \textbf{short prefix o access sequence},   in cui ogni short prefix è una stringa reppresentativa per una classe di equivalenza sulla relazione d'equivalenza $\simeq_{\lambda^{T}}$
 \item Trovare un insieme $V \subset \Sigma^{*}$ di \textbf{suffissi} che è sufficiente a realizzare la relazione di Nerode su \textit{Sp}, cioè tale che $s \not\simeq_{\lambda^{T}} \!\!s'$ implica $\lambda^{T}(sv) \neq \lambda^{T}(s'v) \text{ per } s,s' \in Sp \text{ e qualche } v \in V$ 
\end{enumerate}
Il teorema di Myhill-Nerode asserisce che un linguaggio $L(T) = \ac{L}$ è regolare se e solo se $\simeq_{\lambda^{T}}$ identifica un numero finito di classi d'equivalenza. $\simeq_{\lambda^{T}}$ agisce su due stringhe x e y che sono in relazione se non esiste nessuna stringa z tale che xz e yz, esattamente una delle due appartiene ad \ac{L}: quindi nella classe d'equivalenza ci saranno stringhe non distinguibili da nessuna altra stringa e che quindi rappresentano un unico stato dell'automa T. L'\ac{ObP} trova dei prefissi in cui ogni singolo prefisso è una stringa. contenuta in una specifica classe d'equivalenza (e vi è un prefisso rappresentativo, detto short prefix, per ogni classe d'equivalenza). Essendo  $L(T)=\ac{L}$ un linguaggio regolare, il numero di classi d'equivalenza è finito e sarà certamente possibile trovare un insieme di short prefix Sp . L'esistenza dell'insieme di suffissi V in 2 è garantita dal fatto che se esistono almeno due classi di equivalenza deve esistere almeno un suffisso che distingue le due classi d'equivalenza (che altrimenti sarebbero un'unica classe d'equivalenza). Quindi $\abs{V}$ è limitato dall'indice di $\simeq_{\lambda^{T}}$ cioè dal numero di stati del \ac{DFA} target.
\section{Costruzione dell'ipotesi}
L'\ac{ObP} mantiene due strutture dati rappresentative dell'ipotesi: il \ac{DT} e un insieme di componenti.
\subsection{Tabella di osservazione localizzata}
A differenza di L* in cui vi è un'unica tabella d'osservazione rappresentativa dell'ipotesi, in \ac{ObP} vi è una tabella di osservazione di dimensioni ridotte per ogni stato trovato fino a quel momento denotata come \textbf{componente}:
\begin{definizione*}[Componente] \label{def:obstable}Un componente C è una quadrupla $\Braket{U,u_0,V,OT}$ dove:\\\\
$U \subset \Sigma^{*}$ è un insieme finito di prefissi\\
$u_0 \in U$ è l'unico short prefix del componente\\
$V \subset \Sigma^{*}$ è un insieme di suffissi  $v_1,\dots,v_k$\\
$OT : U \times V \to \{0,1,*\}$ è una funzione così definita:\\\\
\centerline{$
OT[u][v] = 
\begin{cases}
1
& \text{se $uv \in \ac{L}$} \\
0 & \text{se $uv \notin \ac{L}$}\\
* & \text{altrimenti}
\end{cases}
$}   \\\\
Sia $u \in U$ e sia $\abs{V} = n$, si indica con row(u) la riga in OT indicizzata da u cioè $row(u)=OT[u][v_1] \cdot OT[u][v_2] \cdot \dots OT[u][v_n]$. Il componente è individuato dall'access sequence e quindi spesso lo si indica con $C_{u_{0}}$   
\end{definizione*}
Un componente approssima\footnote{Si parla di approssimazione perchè può accadere che alcuni prefissi che attualmente fanno parte dello stesso componente in futuro facciano parte di componenti diverse. Ciò è dovuto all'aggiunta di nuovi suffissi non ancora esaminati fino a quel momento. In ultima analisi ciò è dovuto al fatto che OT ha un dominio ristretto ad $U\times{}V$ anzichè $\Sigma^{*}$} la relazione di Nerode. Ogni componente rappresenta una classe di equivalenza: tutti i prefissi del componente fanno parte della stessa classe di equivalenza cioè   sono equivalenti (secondo OT) in base ai suffissi di quel componente. I valori di output di OT sono ricavati tramite delle \ac{MQ}. L'access sequence non è altro che un prefisso, possibilmente il più breve ma non necessariamente, rappresentativo del componente. Nell'ipotesi \ac{H} ad ogni componente  corrisponde uno stato. I prefissi di un componente sono tutte quelle stringhe che terminano nello stato dell'ipotesi rappresentato da quel componente. L'insieme di tutti i componenti è indicato con $C_I$, invece con Sp($C_I$) o semplicemente Sp si indica l'insieme delle access sequences di tutte le componenti. \textbf{Sp è prefix-closed}.
\subsubsection{Completezza}
La completezza garantisce che non ci siano comportamenti parzialmente (o totalmente) sconosciuti per prefissi presenti all'interno di una componente.
\begin{definizione*}[Componente completo] Un componente è completo se non ha \textit{buchi}. Un \textit{buco} in un componente è una coppia (u,v) tale che $OT[u][v] = *$.
\end{definizione*}
$C_I$ è completo se tutti i componenti sono completi.
L'eventuale incompletezza può essere eliminata mediante \ac{MQ} al \textit{teacher}.
\subsubsection{Chiusura}
La proprietà di chiusura di $C_I$ garantisce che tutti i prefissi di un componente sono equivalenti secondo la relazione di Nerode approssimata.

\begin{definizione*}[Componente chiuso] Un componente $\Braket{U,u_0,V,OT}$ è chiuso se $\text{row}(u) = \text{row}(u_0) \text{ per } \forall u \in U$.
\end{definizione*}
$C_I$ è chiuso se tutti i componenti sono chiusi.\\

La proprietà di chiusura su $C_I$ consente l'applicazione del teorema di Myhill-Nerode sulla relazione $\simeq_{OT}$ . La relazione di Nerode è soltanto approssimata in quanto la funzione OT non è definita su $\Sigma^{*}$ ma su un dominio più ristretto. Tuttavia, se le proprietà di chiusura  e completezza  su $C_I$ sono mantenute il teorema Myhill-Nerode è ancora valido e come visto in \ref{teo:m-n} assicura la costruzione di un \ac{DFA} la cui \textit{funzione di output} è quella su cui si basa la relazione di Nerode utilizzata che nella fattispecie è OT. Un ipotesi \ac{H} consistente con OT può essere costruita come nell'algoritmo \ref{alg:obp-buildautomaton}. L'eventuale non chiusura di un componente può essere eliminata con uno \textit{split} , che consiste nel dividere un componente in due componenti oppure uno stato in due stati (un componente corrisponde ad uno stato).

\begin{algorithm}
\caption{OBP-BUILDAUTOMATON}\label{alg:obp-buildautomaton}
\begin{algorithmic}[1]
\Statex
\Input a closed and complete components set $C_I$
\Output DFA $\Braket{\Sigma,Q,q_\epsilon,F_A,F_R,\delta}$
\State $Q \gets \{q_u :  \forall  u \in \text{Sp}(C_I)\}$
\State $q_\epsilon \gets \text{Component with short prefix } \epsilon $ \LineComment{Stato iniziale corrispondente al componente con short prefix $\epsilon$} 
\State $F_{\mathbb{A}} \gets \{q_u \in Q : \text{OT}[u][\epsilon] = 1\}$
\State $F_{\mathbb{R}} \gets \{q_u \in Q : \text{OT}[u][\epsilon] = 0\}$
\For{$q_u \in Q$}
     \For{$a \in \Sigma$} $\delta(q_u,a) \gets q_w \in Q : \text{row}(ua) = \text{row}(w)$
\LineComment{Da $q_u$ per a si va nello stato $q_w$ se ua è nel componente con short prefix w}     
     \EndFor
\EndFor
\State \textbf{end for}
\State \Return{$\Braket{\Sigma,Q,q_\epsilon,F_A,F_R,\delta}$}
\end{algorithmic}
\end{algorithm}

\subsection{Discrimination tree}
\begin{definizione*}[Discrimination tree]
Un \textit{discrimination tree} è un albero binario con radice definito come \ac{DT} = $\Braket{N,n_0,E,\tau,L}$ dove:\\\\
N è un insieme finito di nodi\\
$n_0 \in N$ è la radice dell'albero\\
$E \subseteq N \times N \times \mathbb{B}$ è l'insieme finito di archi\footnote{Si specifica per ogni arco se il secondo nodo è figlio destro o sinistro del primo nodo (rispettivamente 1 e 0)}\\
$\tau : N \to \Sigma^{*}$ assegna le etichette ai nodi\\
$L \subseteq N$ è l'insieme di foglie 
\end{definizione*}

\begin{figure}[htp]
	\centering
	\includegraphics[ width=0.7\textwidth]{DTipotesi}
	\caption[Correlazione tra ipotesi e discrimination tree]{DFA ipotesi e possibile Discrimination Tree corrispondente}
   \label{fig:hdt}
\end{figure}
Un \ac{DT} è in stretta connessione con l'ipotesi come mostrato in figura \ref{fig:hdt}. C'è una corrispondenza biunivoca tra foglie nel \ac{DT} e stati in \ac{H}. Come si può vedere in figura  \ref{fig:hdt} alcune delle transizioni in \ac{H} sono disegnate in \textit{bold} perchè corrispondono ai prefissi che fungono da short prefix: nella fattispecie si ha $Sp=\{\epsilon,a,ab,b\}$. Le foglie sono etichettate con gli short prefix contenuti in Sp. I nodi interni invece sono etichettati con i suffissi $\in V \text{ di } C_I$ che consentono di \textit{discriminare} le foglie cioè stati diversi del \ac{DFA}. Il \ac{DT} assicura l'esistenza di un discriminatore per ogni coppia di foglie ( e quindi di short prefix e di stati) diverse, il \ac{LCA}. L' \ac{LCA} di due foglie a e b è incontrato nei rispettivi percorsi dai nodi verso la radice e rappresenta ,nei percorsi dalla radice verso a e b, il nodo in cui questi percorsi divergono come evidenziato in figura \ref{fig:lca}.\\Un'operazione fondamentale è il \textit{sift} di una stringa $x \in \Sigma^{*}$ nel \ac{DT} che consente di \textit{affondare} x all'interno dell'albero. Sia q un nodo interno etichettato con v ed A il \ac{DFA} target, il sifting di x  procede,partendo dalla radice, nel sottoalbero sinistro o nel sottoalbero destro di q a seconda del valore di $\lambda^{A}$(xv) (se è 0 si va nel sottoalbero sinistro). Questa procedura è ripetuta finchè una foglia è raggiunta. Ad esempio in figura \ref{fig:hdt} il \textit{sift} della stringa aba ---,che ha access sequence b, e che in \ac{H} termina nello stato $q_2$ --- tramite la valutazione nel \ac{DFA} target A di $\lambda^{A}(aba \cdot \epsilon)$ che da 0 (per questo si va a sinistra) e di $\lambda^{A}(aba \cdot bba)$ che produce 1 (per questo si va a destra)  arriva alla foglia con etichetta b (non casualmente, infatti b e aba sono transizioni che  nell'ipotesi giungono nello stesso stato e quindi sono nella stessa classe di equivalenza).Il \textit{sift} è descritto in dettaglio nell'algoritmo \ref{alg:sift} e nella sottosezione \ref{sub:fun}.
\subsection{Observation Pack}
\begin{definizione*}[Observation Pack]
Un Observation Pack è una tupla $\Braket{C_{I} , \ac{DT}}$
\end{definizione*}
Da cui deriva il nome dell'algoritmo. Un Observation Pack è chiuso e completo se $C_I$ è chiuso e completo.

\begin{figure}[htp]
	\centering
	\includegraphics[ width=0.5\textwidth]{LCA}
	\caption[LCA di due nodi]{LCA di due nodi in un discrimination tree}
   \label{fig:lca}
\end{figure}
\section[Gestione Controesempio]{Gestione del controesempio}

\subsection{Classificazione}
\label{sub:cla}
Un controesempio è una stringa  w $\in \mathcal{L} \oplus L(\ac{H})$ che viene ritornato dal \textit{teacher} quando \ac{H} e il target differiscono. w va sfruttato dal \textit{learner} in qualche modo per produrre una nuova ipotesi \ac{H}. Esistono essenzialmente due modi per farlo: 
\begin{enumerate}
\item \textit{Metodi Suffix-based}. Aggiungono uno o più suffissi del controesempio all'insieme di suffissi dell'algoritmo provocando la non chiusura.
\item \textit{Metodi Prefix-based}. Aggiungono uno o più prefissi di un controesempio all'insieme di prefissi dell'algoritmo causando l'inconsistenza delle osservazioni, cioè una situazione di indeterminismo in \ac{H}.
\end{enumerate}
\ac{ObP} usa la prima strategia.Inoltre un'ulteriore classificazione sui metodi di gestione del controesempio viene effettuata in base a (1) se tutti o solo qualche suffisso (prefisso nel caso dei metodi \textit{prefix-based}) del controesempio sono impiegati e (2) questi suffissi (prefissi) sono applicati a tutti o solo a qualche prefisso (nell'accezione di prefisso rappresentante uno stato dell'ipotesi). Esistono diverse versioni di \ac{ObP} (e in generale anche per i metodi \textit{suffix-based}) in base  a quest'ulteriore classificazione:
\begin{itemize}
\item \textbf{AllGlobally}.\\Si aggiungono tutti i suffissi del controesempio all'insieme di suffissi V di ogni componente in $C_I$.
\item \textbf{OneGlobally}\\Si individua e si aggiunge un singolo suffisso del controesempio e lo si aggiunge all'insieme di suffissi V di ogni componente in $C_I$.
\item \textbf{OneLocally}\\Si individua e si aggiunge un singolo suffisso del controesempio ad un ben preciso componente.
\end{itemize}
In tutti e tre i casi l'aggiunta del suffisso (o dei suffissi) porterà alla non chiusura di almeno un componente e al conseguente \textit{split} che produrrà un nuovo stato nell'ipotesi.  In AllGlobally e OneGlobally è possibile far diventare l'insieme V globale dato che è uguale per ogni componente. Nella strategia OneLocally l'insieme V di suffissi differirà da componente a componente. Su come sia possibile individuare un singolo suffisso dal controesempio e l'esatto componente a cui bisogna aggiungere questo suffisso si rimanda al teorema \ref{teo:dec}.
Una classificazione simile è possibile per i metodi \textit{prefix-based}.
 La politica di gestione del controesempio è una differenza rilevante tra L* e \ac{ObP}. L* gestisce il controesempio in maniera poco sofisticata: usa un metodo \textit{prefix-based} in cui tutti i prefissi del controesempio sono aggiunti alla tabella di osservazione:quindi è una strategia AllGlobally dato che tutti i prefissi del controesempio sono usati in combinazione con un insieme di suffissi globale. Questa strategia causa l'inconsistenza di alcuni prefissi e per risolverla si aggiunge un suffisso che a sua volta causa uno \textit{split} dato che alcuni prefissi (prefissi RED) che prima erano equivalenti adesso non lo sono più e quindi devono corrispondere a due stati diversi. Lo svantaggio principale di questa strategia è che vengono aggiunti dei prefissi improduttivi che fanno aumentare il numero di \ac{MQ}. Invece con una strategia che aggiunge un solo suffisso (o un solo prefisso) è necessario fare delle \ac{MQ} per trovare il suffisso in questione dal controesempio ma il numero complessivo di \ac{MQ} risulterà sempre minore complessivamente ad una strategia AllGlobally come quella usata da L* (qui si fa riferimento alla versione originale di L* in \cite{Angluin87}. Esiste la variante OneLocally di L* in \cite{Kearns94} che determina un unico prefisso del controesempio  che genera inconsistenza, che viene risolta con l'aggiunta di un suffisso che causa uno \textit{split} nella loro versione del discrimination tree). 

\subsection[Decomposizione controesempio]{Decomposizione del controesempio}
Il seguente risultato fondamentale \cite{Schapire93} garantisce che dato qualsiasi controesempio esiste sempre un suffisso che discrimina due prefissi nello stesso componente e ne causa di conseguenza lo \textit{split}:
\begin{teorema}[Decomposizione del controesempio]\label{teo:dec} Sia H un'ipotesi, A il target, e $w \in \Sigma^{+}$ un controesempio ,cioè $\lambda^{A}(w) \neq \lambda^{H}(w)$. Allora esiste una decomposizione $\Braket{u,a,v} \in \Sigma^{*} \times \Sigma \times \Sigma^{*}$ tale che $w = u\!\cdot\! a\!\cdot\! v$ \text{ e } $\lambda^{A}(\lfloor u \rfloor_{H} a \cdot v) \neq \lambda^{A}(\lfloor ua \rfloor_{H} \cdot v)$ .
\end{teorema}  
Innanzitutto bisogna precisare che il controesempio non può essere mai $\epsilon$ perchè $\lambda^A(\epsilon) = \lambda^H(\epsilon)$ che segue da $\epsilon \in Sp$ e dall'invariante (I2) (vedasi sottosezione \ref{sub:corret}) e ciò giustifica l'assunzione $w \in \Sigma^{+}$ che si fa nel teorema \ref{teo:dec}. Si osservi che u e $\lfloor u \rfloor_{H}$ sono stringhe che terminano nello stesso stato di \ac{H}, lo stesso dicasi allora per ua e $\lfloor u \rfloor_{H}\cdot a$. Inoltre ua e $\lfloor ua \rfloor_{H}$ terminano nello stesso stato di \ac{H} da cui segue transitivamente che pure $\lfloor u \rfloor_{H}\cdot a$ e $\lfloor ua \rfloor_{H}$  terminano nello stesso stato dell'ipotesi \ac{H} ( e quindi appartengono anche allo stesso componente dato che c'è corrispondenza biunivoca tra stati dell'ipotesi e componenti). Quanto detto è schematizzato nella figura \ref{fig:con}. Ma il teorema \ref{teo:dec} afferma che $\lambda^{A}(\lfloor u \rfloor_{H} a \cdot v) \neq \lambda^{A}(\lfloor ua \rfloor_{H} \cdot v)$ il che significa che $\lfloor u \rfloor_{H}\cdot a$ e $\lfloor ua \rfloor_{H}$ --- che sono due stringhe che in \ac{H} terminano nello stesso stato --- nel target A terminano in due stati diversi. Quindi si è trovato il suffisso v che discrimina i due prefissi $\lfloor u \rfloor_{H} \cdot a$ e $\lfloor ua \rfloor_{H}$ e in più si sa che i due prefissi sono anche nello stesso componente.
\begin{figure}[htp]
	\centering
	\includegraphics[ width=0.7\textwidth]{DiscriminazioneControesempio}
	\caption[Sfruttare il controesempio]{Sfruttare il controesempio}
   \label{fig:con}
\end{figure}
\subsubsection{Decomposizione del controesempio riformulata}
In \cite{StefCounterexample14} è descritto un framework che permette di riformulare il teorema \ref{teo:dec} in modo da facilitarne la comprensione della correttezza ed anche l'implementazione. Alcune funzioni si trovano nell'appendice \ref{app:due} e qui sono date per scontate.

Sia dato un controesempio $w \in \Sigma^{+}$ tale che $\abs{w}=m$. Dato che $\pi_{H}(w,m) \in Sp$ (si evince dalla definizione di w considerando che il controesempio è lungo m), dall'invariante (I2) (nella sottosezione \ref{sub:corret}) si ha $\lambda^{A}(\pi_{H}(w,m)) = \lambda^{H}(w)$ quindi $\alpha(m) = 1$ . Inoltre dato che $\pi_{H}(w,0) = w$ e $w$ è un controesempio si ha $\lambda^{A}(\pi_{H}(w,0)) \neq \lambda^{H}(w)$, quindi $\alpha(0) = 0$. Allora si può ricondurre il teorema \ref{teo:dec} nel trovare un undice i tale che $\alpha(i) \neq \alpha(i+1)$. Siccome $\alpha(0)=0 \text{ e } \alpha(1)=1$ sicuramente ci sarà un indice i in cui il valore di $\alpha$ passa da 0 ad 1 e ciò dimostra la correttezza e l'esistenza del suffisso. Il teorema \ref{teo:dec} può essere riformulato così:
\begin{teorema}[Decomposizione del controesempio riformulata] \label{teo:decr}Una politica di analisi del controesempio \textit{suffix-based} può essere riformulata come il problema di, data una funzione $\alpha: [0,m+1) \to \mathbb{B}$ con $\alpha(0)=0 \text{ e } \alpha(1)=1$, trovare un indice i, $0 \leq i < m$, soddisfacente $\alpha(i) \neq \alpha(i+1)$ 
\end{teorema}
Una volta trovato un indice i siffatto la decomposizione è u=$w_{[0,i)}$, a=$w_{i}$, v=$w_{[i+1,m)}$

\subsection[Metodi decomposizione controesempio]{Metodi di decomposizione del controesempio} Esistono diverse strategie per la ricerca del suffisso (o per i metodi \textit{prefix-based}) all'interno del controesempio. I parametri da tenere in considerazione nella valutazione di tali strategie sono il numero di \ac{MQ} effettuate e la dimensione del suffisso trovato. Infatti per valutare $\alpha(i)$ è necessaria una \ac{MQ} quindi l'impiego di un'euristica o di una politica che porta velocemente alla scoperta dell'indice i e quindi del suffisso permette di risparmiare il numero di \ac{MQ} da effettuare al \textit{teacher}. La dimensione dei suffissi\footnote{All'interno di un controesempio possono esistere più suffissi discriminatori, cioè molteplici indici i che permettono una decomposizione} è anch esso un parametro molto importante in quanto il suffisso trovato verrà aggiunto all'insieme dei suffissi dei componenti o del componente rispettivamente in OneGlobally e OneLocally, e anche al \ac{DT} (vedasi algoritmo \ref{alg:split}). Quando si navigherà il \ac{DT} , per esempio durante un sift, verranno fatte delle \ac{MQ} in cui parte della stringa è composta dall'etichetta di un nodo interno costituita da un suffisso; quindi il costo della \ac{MQ} crescerà linearmente con la dimensione del suffisso.

La strategia più semplice è quella che effettua una ricerca lineare in ordine discendente cioè partendo da $i = m-1$ che termina quando un valore i tale che $\alpha(i)=0$ è incontrato. Questo metodo assicura di trovare il suffisso più breve ma nel caso peggiore richiede m-1 \ac{MQ}. Anche se un costo lineare alla dimensione del controesempio sembra accettabile negli scenari reali non sempre lo è. In un contesto reale spesso capita di non avere la possibilità di effettuare un \ac{EQ} che va approssimata con un certo numero di \ac{MQ} nell'ambito del \textit{PAC-learning}. In questo scenario il controesempio tornato è una stringa di lunghezza non ottimale che può avere una lunghezza significativa. Esistono dei metodi che permettono di diminuire il numero di \ac{MQ} anche se spesso bisogna rinunciare alla lunghezza minima del controesempio ed accontentarsi di una lunghezza ottima.

\noindent

\begin{minipage}{0.48\textwidth}
\captionsetup{format=ruled,labelfont=bf}
%\flushleft
 \captionof{algorithm}{Binary-Search}\label{alg:bin}
   
    \begin{algorithmic}[0]
    \small
\Statex
\Input A counterexample w with $\abs{w}=m$
\Output Index i : $\alpha(i) \neq \alpha(i+1)$
\State $low \gets 0$
\State $high \gets m$
\While{$high-low>1$}
\State $mid\gets \lfloor \frac {low+high}{2} \rfloor$
\If{$\alpha(mid)=0$}
\State $low \gets mid$
\Else
\State $high \gets mid$
\EndIf
\EndWhile
\State \textbf{\quad\:\:end if}
\State \textbf{end while}
\State \textbf{return} $low$


\end{algorithmic}

  \kern2pt\hrule height.8pt\relax
\end{minipage}%
\hfill
\begin{minipage}{0.50\textwidth}
%\flushright
%\vspace{0pt}
\captionsetup{format=ruled,labelfont=bf}
  \captionof{algorithm}{Exponential-Search}\label{alg:exp}
  
    
    \begin{algorithmic}[0]
    \small
\Statex
\Input A counterexample w with $\abs{w}=m$
\Output Index i : $\alpha(i) \neq \alpha(i+1)$
\State $low\! \gets \!0,\, high\! \gets \!m,\, \text{\textit{ofs}}\! \gets \!1,\, found\! \gets \!\text{\textbf{false}}$
%\State $\text{\textit{ofs}} \gets 1, \:\: found \gets \text{\textbf{false}}$
\While{$high-ofs>0 \text{\textbf{ and }} \neg{}found$}
\If{$\alpha(high-ofs)=0$}
\State $low \gets high - ofs$
\State $found \gets \text{\textbf{true}}$
\Else
\State $high \gets high-ofs$
\State $ofs \gets 2 \cdot ofs$
\EndIf
\EndWhile
\State \textbf{\quad\:\:end if}
\State \textbf{end while}
\State \textbf{return} $\text{\textit{Binary-Search}}(\alpha,low,high)$

\end{algorithmic}
  
  \kern2pt\hrule height.8pt\relax
\end{minipage}





   
\subsubsection{Binary search}
Il metodo descritto nell'algoritmo \ref{alg:bin} è suggerito in \cite{Schapire93}. Esso impiega la ricerca binaria per trovare una decomposizione valida. Il numero di \ac{MQ} necessarie è $\lceil log_{2}(m) \rceil$ sempre, cioè non esiste un caso migliore ma il numero di \ac{MQ} è fisso perchè è necessario testare il valore di $\alpha()$ per due indici i contigui e ciò avviene solo alla fine della ricerca. La dimensione del suffisso trovata può anche essere molto più grande di quella minima.
\subsubsection{Exponential Search}
 Il metodo descritto qui e tutti quelli a seguire sono descritti in \cite{StefCounterexample14}.  La ricerca binaria ha lo svantaggio evidente che può essere tornato un  controesempio relativamente lungo: se il primo valore per \textit{mid} testato è tale che $\alpha(mid)=1$ il suffisso risultante sarà di lunghezza almeno   $\lceil m/2 \rceil$\footnote{La funzione $\alpha()$ non è necessariamente monotona quindi anche se $\alpha(mid)=1$ ci può essere un indice $i>m$ per il quale $\alpha(i)=0$ e quindi esserci un suffisso più breve}. Con \textit{exponential search}  si testano $\alpha(m-2^{0}),\alpha(m-2^{1}),\alpha(m-2^{2})$ eccetera, finchè non si trova un intervallo [l,h) per cui $\alpha(l)=0 \text{ e } \alpha(h)=1$ ed allora si chiamerà il metodo che usa la ricerca binaria descritto nell' algoritmo \ref{alg:bin} sui due indici l ed h. Nel caso peggiore(\textit{exponential search} non riesce a restringere l'intervallo cioè l resta 0 ed h resta m) questo metodo richiede $2\lfloor log_{2}(m) \rfloor$ \ac{MQ} , $\lfloor log_{2}(m)\rfloor$ per \textit{exponential search} e altrettante per la ricerca binaria.Nella maggior parte dei casi l'algoritmo termina molto prima e nel caso migliore ($\alpha(m-1)=0$) si effettua una singola \ac{MQ}. Per come funziona quest algoritmo favorisce il ritrovamento di suffissi più brevi rispetto alla ricerca binaria.\\
  \textit{Exponential Search} è presentato in dettaglio nell'algoritmo \ref{alg:exp}.
 
 \subsubsection{Partition Search}
\textit{Exponential search} può terminare velocemente e individuare suffissi molto brevi ma nel caso che  anche poche posizioni di $\alpha(m-2^{i}) = 1$ può essere svantaggiosa per via della rapida (anche per i piccolo),esponenziale, crescita dell'intervallo. Un approccio più bilanciato è quello di patizionare $\alpha$ in $\lceil log_{2}(m)\rceil$ intervalli, ognuno di lunghezza $s = \lfloor  \frac{m}{log_{2}m} \rfloor$. Poi i valori testati saranno $\alpha(m-s), \alpha(m-2s)$ eccetera, finchè un intervallo [l,h) soddisfacente $\alpha(l)=0 \text{ e } \alpha(h)=1$ è trovato. Quest intervallo sarà poi sottoposto al metodo di ricerca binaria per trovare un indice i in esso. In \textit{exponential search} a causa del passo esponenziale questo intervallo poteva risultare molto grande, in \textit{partition search} è di dimensione s. Quindi il numero di \ac{MQ} da effettuare nel caso peggiore con  \textit{partition search} è $log_{2}(s) = log_{2}(\lfloor  \frac{m}{log_{2}m} \rfloor)$ per via della ricerca binaria da eseguire sull'intervallo trovato più $\lfloor log_{2}(m)\rfloor$ (il numero di partizioni) \ac{MQ} per trovare l'intervallo. Quindi sono necessarie $\mathcal{O}(log_{2}(m))$ \ac{MQ}. \textit{Partition Search} è presentato in dettaglio nell'algoritmo \ref{alg:par}. Si osservi come il costo della ricerca binaria sia fisso (cioè è presente anche nel caso migliore) e dipendente da m. Quindi ci si aspetta che questo metodo funzioni meglio per controesempi non troppo grandi (m piccolo).

\subsubsection{Eager Search}
\textit{Eager Search} è una variante del metodo di ricerca binaria. Quest ultimo richiede sempre --- cioè non c'è un caso migliore o medio ma il numero di \ac{MQ} è sempre lo stesso --- $log_{2}(m)$ \ac{MQ}. Come accennato in precedenza il motivo è che il solo valore $\alpha(i)$ da solo non è sufficiente ma è necessaro testare anche $\alpha(i+1)$\footnote{perchè dal teorema \ref{teo:decr} si deve trovare un indice i per cui $\alpha(i) \neq \alpha(i+1)$}. La soluzione proposta in \textit{Eager Search} è di testare ogni volta il valore di $\alpha$ sia per i che per i+1 e vagliare se differiscono o detto in maniera più succinta che il
valore di $\beta$(vedasi \ref{equ:beta}) sia uguale ad 1. Nel caso peggiore siccome ogni valutazione di $\beta$ richiede 2 \ac{MQ} e il numero di valutazioni da fare è lo stesso della ricerca binaria (nel caso peggiore) sono necessarie $2log_{2}(m)$ \ac{MQ}. Tuttavia nel caso migliore solo 2 \ac{MQ} sono sufficienti e la ricerca può terminare molto prima di quella binaria. Questa strategia è affetta dallo stesso problema della ricerca binaria per quanto riguarda la dimensione dei suffissi tuttavia è possibile utilizzare \textit{Eager Search} in luogo della ricerca binaria sia in \textit{exponential search} che in \textit{partition search}. 

\noindent

\begin{minipage}{0.48\textwidth}
\captionsetup{format=ruled,labelfont=bf}
%\flushleft
 \captionof{algorithm}{Partition-Search}\label{alg:par}
   
    \begin{algorithmic}[0]
    \small
\Statex
\Input A counterexample w with $\abs{w}=m$
\Output Index i : $\alpha(i) \neq \alpha(i+1)$
\State $step \gets \lfloor \frac{m}{log_{2}(m)} \rfloor low \gets 0,\, high \gets m$
\State $found \gets \text{\textbf{false}}$
\While{$high\!-\!step\!>\!low \text{\textbf{ and }} \neg{}found$}
\If{$\alpha(high-step)=0$}
\State $low \gets high-step$
\State $found \gets \text{\textbf{true}}$
\State \text{\textbf{break}}
\Else
\State $high \gets high-step$
\EndIf
\EndWhile
\State \textbf{\quad\:\:end if}
\State \textbf{end while}
\State \textbf{return} $\text{\textit{Binary-Search}}(\alpha,low,high)$


\end{algorithmic}

  \kern2pt\hrule height.8pt\relax
\end{minipage}%
\hfill
\begin{minipage}{0.50\textwidth}
%\flushright
%\vspace{0pt}
\captionsetup{format=ruled,labelfont=bf}
  \captionof{algorithm}{Eager-Search}\label{alg:eag}
  
    
    \begin{algorithmic}[0]
     \small
\Statex
\Input A counterexample w with $\abs{w}=m$
\Output Index i : $\beta(i)=1$
\State $low \gets 0,\: high \gets m-1$
\While{$high > low$}
\State $mid\gets \lfloor \frac {low+high}{2} \rfloor$
\If{$\beta(mid)=1$}
\State \textbf{return} $mid$
\ElsIf{$\beta(mid)=0$}
\State $low \gets mid+1$
\Else
\State $high \gets mid-1$
\EndIf
\EndWhile
\State \textbf{\quad\:\:end if}
\State \textbf{end while}
\State \textbf{return} $low$
\end{algorithmic}
  
  \kern2pt\hrule height.8pt\relax
\end{minipage}
\section{L'algoritmo}
\label{sec:alobp}
\subsection{Funzionamento}
\label{sub:fun}
\ac{ObP} nella fase d'inizializzazione crea il componente corrispondente allo stato iniziale $C_{\epsilon} = \Braket{\Sigma \cup \{\epsilon\},\epsilon,\epsilon,\emptyset}$ ed il \ac{DT} è costituito solo dalla radice quindi $DT = \Braket{\{n_{\epsilon}\},n_{\epsilon},\emptyset,\tau(n_{\epsilon})=\epsilon,\{n_{\epsilon}\}}$ come si può vedere nell'algoritmo \ref{alg:obpp}. Dopodichè si chiama la funzione closePack() (vedasi algoritmo \ref{alg:clp}) che ha lo scopo di rendere chiusi e completi $C_{I}$ e modificare il \ac{DT} in modo da rappresentare la stessa ipotesi rappresentata da $C_{I}$. Si completano i $C_{I}$ e poi si ricerca un componente $C_{u}$ (questo in generale e non solo nella fase di inizializzazione dove l'unico componente è $C_{\epsilon}$) e un prefisso $u' \in C_{u}$ per cui $\text{row}(u) \neq \text{row}(u')$. Si seleziona il suffisso v per cui $OT[u][v] \neq OT[u'][v]$ e si divide $C_u$ chiamando la funzione \textit{split} (algoritmo \ref{alg:split}) che genererà un nuovo componente $C_{u'}$ e lo ritornerà a closePack(). In split() alcuni dei prefissi $U \in C_{u}$,che indichiamo con $x \in U$, ed esattamente quelli per cui $OT[x][v] \neq OT[u][v]$ vengono fatti migrare in $C_{u'}$ che nel frattempo è stato creato(cioè eliminati dal componente $C_{u}$ ed immessi in $C_{u'}$). Si procede poi a modificare di conseguenza anche il \ac{DT}: si individua il nodo foglia con etichetta u e lo si splitta nel senso che questo nodo foglia diventa un nodo interno con etichetta v (il suffisso, cioè il discriminatore) ed i suoi figli saranno due nodi foglia uno con etichetta u' e l'altro con etichetta u come si può apprezzare in figura\footnote{Per coerenza con il discorso fatto si consideri $v=v$, $b=u$,$b'=u'$} \ref{sub:split} (si stabilisce qual è il figlio sinistro e quale il destro in base al risultato della \ac{MQ} $\lambda^{A}(uv)$ nel target A, se questa fa zero u è il figlio sinistro del nuovo nodo con etichetta v altrimenti è il figlio destro). Si puntualizza che lo \textit{split} non necessita di \ac{MQ} aggiuntive e che l'insieme V del nuovo componente creato con lo \textit{splitting} è lo stesso dell'insieme V del componente da cui ha origine. A questo punto split() ritorna il componente in questione a closePack() che memorizza il componente trovato $C_{u'}$ in W\footnote{ (solo quando closePack() viene chiamato per la prima volta in assoluto ,immediatamente dopo la fase di inizializzazione, a W deve essere aggiunto anche $C_{\epsilon}$ (per esempio una coda)}. Finchè W non è vuoto si deve estrarre ed eliminare da W un componente , nella fattispecie $C_{u'}$, e concatenare l'access sequence del componente u' ad ogni simbolo di $\Sigma$ e per ogni stringa ottenuta effettuare il \textit{sift} (algortimo \ref{alg:sift}). Questo serve per assicurare in ogni caso la possibilità di costruire una nuova ipotesi \ac{H} in modo che per ogni nuovo componente aggiunto per ogni simbolo dell'alfabeto sia ben definito lo stato che viene raggiunto. Adesso si descrive in dettaglio come avviene il \text{sift} di una stringa x nel \ac{DT}: si parte dalla radice e si effettua la query nel target A $\lambda^{A}(x\cdot{}\tau(n_{0}))$ ,se da esito 0 si procede verso il sottoalbero sinistro altrimenti verso quello destro. Si continua in questa maniera finchè due situazioni possono presentarsi:
\begin{itemize}
\item Un nodo foglia $n_{x'}$ è incontrato. Significa che la stringa x in \ac{H} porta nello stato rappresentato dallo foglia in cui si è arrivati: detto $x'$ lo short prefix di questa foglia si ha che $x \in [x']_{\simeq_{OT}}$. Quindi si dovrà anche provvedere ad aggiungere x all'insieme U del componente $C_{x'}$. L'assegnazione della stringa come nuovo prefisso a un componente non è necessariamente definitiva perchè successivamente può accadere che la scoperta di un nuovo suffisso renda non chiuso un componente e per assicurare la chiusura si debba effettuare uno \text{split} che divide il componente in due parti e causando così possibilmente anche la migrazione di alcuni prefissi nel nuovo componente.
\item
Un nodo interno $n_{z}$ con un solo figlio e con etichetta $z$ è incontrato ma  $\lambda^{A}(xz)$ suggerisce di andare verso il percorso dove $n_{z}$ non ha figli. In questo caso deve avvenire la creazione di una nuova foglia $n_{x}$ con etichetta x come figlio del nodo $n_{z}$. Conseguentemente deve essere aggiunto $C_x$ a $C_I$. L'insieme di suffissi V di $C_x$ sarà inizializzato in OneLocally con i suffissi trovati lungo il percorso nel \ac{DT} per arrivare alla nuova foglia e con OneGlobally e AllGlobally  con i suffissi globali dell'algoritmo.\end{itemize}
Nel caso in cui  sift() torna un nuovo componente a closePack() quest ultimo va aggiunto in W. Si ripete il procedimento finchè W non è vuota. Inoltre l'algoritmo deve rendere $C_{I}$  completo (la non completezza è generata dal \textit{sifting} e anche dall'eventuale non chiusura tranne alla prima iterazione dove è causata dal suffisso trovato nella decomposizione del controesempio). Si procede poi alla generazione dell'ipotesi \ac{H} come visto nell'algoritmo \ref{alg:obp-buildautomaton} e si effettua un \ac{EQ} cui il \textit{teacher} può rispondere dato che si assume che appartenga alla classe dei \ac{MAT}. Se il target A e l'ipotesi \ac{H} sono equivalenti l'algoritmo termina garantendo che \ac{H} sia il DFA minimo e l'equivalenza di A e \ac{H} altrimenti verrà tornato un controesempio che verrà sfruttato per trovare un suffisso discriminante due prefissi appartenenti allo stesso componente (e con uno dei prefissi ,diciamo x, $: x \in Sp$). Se si usa OneLocally si aggiunge il suffisso all'insieme V di $C_{x}$, con OneGlobally si aggiunge il suffisso all'insieme V che sarà globale per tutte le componenti. Con la strategia AllGlobally non è necessario effettuare la decomposizione del controesempio in quanto si aggiungeranno tutti i suffissi del controesempio all'insieme di suffissi V globale e comune a tutte le componenti ma è chiaramente una strategia controproducente. L'aggiunta del suffisso v garantisce la non chiusura ad ogni  `` generazione  '' e quindi l'aggiunta di almeno un nuovo componente, e quindi stato, mediante lo \textit{split}.

\begin{algorithm}
\caption{OBP-SIFT}\label{alg:sift}
\begin{algorithmic}[1]
\Statex
\Input a $DT=\Braket{N,n_0,E,\tau,L}$, $C_I$ , new prefix $u \in \Sigma^{*}$   
\Output A new component or  ``OK''
\State $n=n_0$
     \While{$n \notin L$}
     \State $v \gets \tau(n)$
     \State $o \gets MQ(uv)$ \Comment{$MQ(uv) = \lambda^{A}(uv)$}
     \If {$ \exists (n,n',o) \in E$} \Comment{Se n ha un figlio nella direzione indicata da o}
     \State $n \gets n'$
     \Else \Comment{Il nodo n non ha figli nella direzione indicata da o}
     \State Create new node $n_u$
     \State $N \gets N \cup \{n_u\},\quad E \gets E \cup \{(n,n_u,o)\}$
     \State $\tau(n_u)=u$
     \State $\text{Create component }  C_u$ \LineComment{Aggiungi a $C_u$ lo short prefix u e i suffissi secondo la strategia usata}
     \State \textbf{return} $C_u$
     \EndIf  
     \EndWhile
      \State \textbf{end while}
     \LineComment{Se si è qui significa che si è arrivati a una foglia}
     \State $u'=\tau(n)$ \Comment{Si prende lo short prefix del nodo n}
     \State $\text{add u to } C_{u'}$ 
     \LineComment{aggiungere il prefisso al componente solo se il prefisso non è già presente}
    \State \textbf{return} ``OK'
\end{algorithmic}
\end{algorithm}

Inoltre a differenza di L* in \ac{ObP} è possibile sfruttare più volte lo stesso controesempio. In L* tutti i prefissi del controesempio  compreso il controesempio stesso vengono aggiunti alla tabella di osservazione e la nuova ipotesi generata è consistente con essi e quindi alla successiva `` generazione  '' il controesempio precedente non è più riutilizzabile perchè difatti non è più in \ac{L} $\oplus$ L(\ac{H}). In \ac{ObP} invece viene aggiunto un solo suffisso del controesempio e quindi potrebbe accadere che alla successiva `` generazione  '' sia ancora un controesempio utilizzabile per trovare un altro suffisso che produca la non chiusura. Ciò può fare risparmiare molte \ac{EQ} al costo di una \ac{MQ} aggiuntiva per ogni `` generazione  '' (perchè è necessario testare se il controesempio è ancora tale).

\begin{algorithm}
\caption{OBP-SPLIT}\label{alg:split}
\begin{algorithmic}[1]
\Statex
\Input a $DT=\Braket{N,n_0,E,\tau,L}$, $C_I$ , a component $C_{u_{0}}=\Braket{U,u_0,V,OT}$, a prefix $u \in U$ , a suffix $v \in V : OT[u_0][v]) \neq OT[u][v]$
\Output A new component
\State $\text{Create component } C_u=\Braket{\emptyset,u,V,\emptyset}$
     \For{$u' \in U$}
     \If {$OT[u_0][v] \neq OT[u'][v]$}
     \State $\text{Transfer u' from } C_{u_{0}} \text{ to } C_{u} $ \Comment{Trasferire significa eliminare da $C_{u_{0}}$}
     \EndIf
     \EndFor
     \State \textbf{end for}
      \LineComment{Adesso a seguire le modifiche da apportare al discrimination tree}
      \State Let $n \in L$ where $\tau(n)=u_0$ \Comment{Seleziona la foglia con etichetta $u_0$}
      \State $\tau(n) = v$ \Comment{Modifica l'etichetta del nodo con quella del suffisso discriminante}
      \State Create new node $n_u$
      \State Create new node $n_{u_{0}}$
      \State $N \gets N \cup \{n_u,n_{u_{0}}\}$
      \State $\tau(n_u) = u$
      \State $\tau(n_{u_{0}}) = u_0$
      \State $E \gets E \cup \{(n,n_u,OT[u][v])\} \cup \{(n,n_{u_{0}},OT[u_0][v])\}$
    \State \textbf{return} $C_u$
\end{algorithmic}
\end{algorithm}



\begin{algorithm}
\caption{OBP-CLOSEPACK}\label{alg:clp}
\begin{algorithmic}[1]
\Statex
\Input a observation pack $\Braket{DT,C_{I}}$
\Output ipotesi H
\State $W \gets \emptyset $ \text{ (only in first call ever $W \gets \{C_{\epsilon}\}$)} 
     \While{$C_{I} \text{ is unclosed or incomplete }$} \label{lin:clpunclosed}
     \State Complete $C_I$
     \State Let $C_{u_{0}} = \Braket{U,u_0,V,OT} \text{ with } u \in U, v \in V : OT[u_0][v] \neq OT[u][v]$ \label{lin:clpsuffix}
     \State $C_u = $ \Call{OBP-SPLIT}{$DT,C_I,C_{u_{0}},u,v$}
     \State $W \gets W \cup \{C_u\}$
     \While{$W \neq \emptyset$}
     \State $C_u \gets poll(W)$ \LineComment{Estrai un componente secondo una qualche politica ad esempio FIFO}
     \For{$a \in \Sigma$}
     \State $C = $ \Call{OBP-SIFT}{$DT,C_I,ua$}
     \If{$C \neq$ ``OK''}
     \State $W \gets W \cup C$
     \EndIf
    
     \EndFor
      \State \textbf{end for}
     
     \EndWhile
     \State \textbf{end while}
    
    \EndWhile
    \State \textbf{end while} 
    
    \State $H= $ \Call{OBP-BUILDAUTOMATON}{$C_{I}$}
    \State \textbf{return } H
     
\end{algorithmic}
\end{algorithm}

\begin{algorithm}
\caption{OBSERVATION PACK}\label{alg:obpp}
\begin{algorithmic}[1]
\Statex
\Input alfabeto $\Sigma$ 
\Output minimal DFA H : L(H)=\ac{L}  
\State $C_{\epsilon}=\Braket{\Sigma \cup \{\epsilon\},\epsilon,\epsilon,\emptyset}$
\State $C_I = C_{\epsilon}$ 
    \State $DT = \Braket{\{n_{\epsilon}\},n_{\epsilon},\emptyset,\tau(n_{\epsilon})=\epsilon,\{n_{\epsilon}\}}$
    \Loop
    \State H = \Call{OBP-CLOSEPACK}{$DT,C_I$}
    \State w = \Call{EQ}{H}
    \If{ w = ``OK''}
    \State \textbf{return} H 
    \EndIf
    \State \textbf{end if}
     \If{AllGlobally}
    \State Add all suffixes of w to V of all components in $C_I$
    \State \textbf{continue}
    \EndIf
    \State \textbf{end if}
    \State Split w with a decomposition method in uav : $\lambda^{A}(\underbrace{\lfloor u \rfloor_{H}a}_{p} v) \neq \lambda^{A}(\underbrace{\lfloor ua
     \rfloor_{H}}_{u'}v)$
     \Statex \quad \: tale che $p \in C_{u'}$
    \If{OneGlobally}
    \State Add v to V of all components $\in C_I$ \label{lin:addsuffixes}
    \EndIf
    \State \textbf{end if}
    \If{OneLocally}
    \State Add v to V of $ C_{u'}$
    \EndIf
    \State \textbf{end if}
    \EndLoop
    \State \textbf{end loop}
     
\end{algorithmic}
\end{algorithm}

\subsection{Correttezza}
\label{sub:corret}
La correttezza dell' \ac{ObP} scaturisce dal mantenimento di tre \textit{\textbf{invarianti d'apprendimento}} :
\begin{enumerate}[label=\textbf{{(I}\arabic*{)}}]
  \item $u \neq u' \in Sp$ corrispondono a stati differenti nel \ac{DFA} target A cioè $A[u] \neq A[u']$.
  \item Ogni stato q in \ac{H} è accettante se e solo se ``parsando'' $\lfloor q \rfloor_{H}$ nel target A si termina pure in uno stato accettante. In simboli: $\forall q \in Q^{H} : q \in  F_{\mathbb{A}}^{H} \Leftrightarrow f_{Sp}(q) \in F_{\mathbb{A}}^{A}$
  \item Transizioni in \ac{H} puntano allo stato corretto nel target, se quest ultimo è già stato scoperto dal \textit{learner}: $\forall q \in Q^{H}, a \in \Sigma \text{ si ha } \delta^{A}(f_{Sp}(q),a) \in A[Sp] \implies f_{Sp}(\delta^{H}(q,a)) = \delta^{A}(f_{Sp}(q),a)$
\end{enumerate}
E' evidente che la scoperta di nuovi short prefix in Sp mantenendo la condizione (I1)  porterà alla scoperta di tutti gli stati nel target A (dato che sono finiti per il teorema di Myhill-Nerode descritto in \ref{teo:m-n}). Le condizioni (I2) ed (I3) garantiscono che $f_{Sp}$ è un isomorfismo .In L* vi è una violazione della condizione (I1) perchè può accadere che più di uno short prefix (più di uno stato RED) rappresentino lo stesso stato nel target. Il requisito di consistenza tuttavia garantisce che uno qualsiasi di questi può essere scelto come rappresentativo dello stato target.\\

Si dimostra che le tre \textit{invarianti} sono possedute dall'algoritmo \ac{ObP}:
\begin{enumerate}[label=\textbf{{(I}\arabic*{)}}]
\item Siano u e $u' \in Sp \text{ e } u \neq u' $,  allora essendo in due componenti diversi esiste un suffisso v che distingue u e $u'$ cioè tale che $\lambda^{A}(uv)=OT[u][v] \neq \lambda^{A}(u'v)=OT[u'][v]$. Quindi si ha anche $\lambda_{A[u]}^{A}(v) \neq \lambda_{A[u']}^{A}(v)$ e quindi $A[u] \neq A[u']$
\item  Sia $u=\lfloor q \rfloor_{H} $ l'access sequence di uno stato $q \in Q^H$.  Si ha allora che $q \in F_{\mathbb{A}}^{H} \iff$\!\!\footnote{Nel caso della strategia OneGlobally è lampante capire che in ogni componente ci sarà sempre il suffisso $\epsilon$ essendo questo già presente in fase d'inizializzazione. Utilizzando OneLocally accade lo stesso perchè i suffissi da aggiungere ad un nuovo stato (rappresentato da una access sequence e da un componente) sono quelli incontrati durante il \textit{sifting} dell'access sequence nel discrimination tree e quindi si passa sempre dalla radice che contiene il suffisso $\epsilon$} $OT[u][\epsilon]=1$.  Dato che $OT[u][\epsilon]=\lambda^{A}(u\cdot\epsilon)=\lambda_{A[u]}^{A}(\epsilon)$ si può concludere che $q \in  F_{\mathbb{A}}^{H} : u=\lfloor q \rfloor_{H} \Leftrightarrow A[u]$\footnote{è uguale a $f_{Sp}(q)$} $\in F_{\mathbb{A}}^{A}$
\item Sia dato uno stato $q \in Q^{H} $ tale che $u = \lfloor q \rfloor_{H}$, il successore di q per il simbolo $a \in\Sigma$ è determinato con il \textit{sifting} di ua nel \ac{DT}. Sia lo stato con access sequence $u'$ il risultato di questa operazione di \textit{sifting}, si definisce allora $\delta^{H}(q,a) = H[u']$. Se però nel target A questa transizione è errata cioè $\delta^{A}(A[u],a) \neq A[u']$, si sa per certo che il successore reale sul simbolo $a$ cioè  $A[ua]$ non è ancora stato scoperto, quindi (I3) è preservata: si noti che \textit{siftando} ua nell'albero, arrivando ad $u'$ si escludono definitivamente  gli altri  stati scoperti fino a quel punto $A[Sp\textbackslash \{u'\}]$ come possibili successori sul simbolo $a$ di $A[u]$. 
\end{enumerate}
Ciò dimostra la correttezza dell' \ac{ObP} ma tramite quanto detto sopra si dimostra solo che \ac{H} ed A sono isomorfi senza assumere la canonicità per \ac{H}. Infatti in generale  A non è un \ac{DFA} minimo e secondo quanto detto finora non si può dedurre che \ac{H} sia minimo ma solo isomorfo  ad A. Inoltre a causa del maccanismo di gestione del controesempio non vi è la garanzia che l'insieme di suffissi sia \textit{suffix-closed} dato che viene aggiunto solo un suffisso del controesempio. Quanto detto causa anche la generazione di ipotesi intermedie non minime. Tuttavia applicando l'euristica di utilizzare più volte lo stesso controesempio finchè da esso non è più possibile estrarre un suffisso discriminante è garantito che l'ipotesi finale è un \ac{DFA} canonico \cite[p. 25]{Stef11}. Esiste un'ottimizzazione che tramite  il concetto di \textit{semantic suffix closdness} ,sempre in \cite{Stef11} , assicura che anche le ipotesi intermedie prodotte siano canoniche ed il meccanismo che permette di scoprire dei nuovi suffissi da quelli esistenti in modo da ottenere la \textit{semantic suffix closdness} permette di risparmiare anche delle \ac{EQ} e \ac{MQ} perchè permette di scoprire nuovi suffissi discriminanti e nuovi stati senza costi aggiuntivi in termini di \ac{EQ} ed \ac{MQ} (vi è però il tempo di esecuzione aggiuntivo dell'algoritmo per garantire la \textit{semantic suffix closdness} ).


\subsection{Complessità computazionale}
Il tempo di esecuzione degli algoritmi di \textit{active learning} è quasi sempre un polinomio di grado piccolo dipendente dalla dimensione dell'alfabeto k, il numero di stati  del \ac{DFA} canonico  equivalente ad \ac{L} che qui si indica con $n$ , e la lunghezza del controesempio $m$ con cui si indica la lunghezza del più lungo controesempio tornato dal \textit{teacher}. Il tempo di esecuzione è quasi sempre dominato dal tempo speso nell'effettuare  \ac{MQ} ed \ac{EQ} \footnote{Talvolta si conta anche il numero di simboli contenuto in tutte le stringhe per cui si effettua una \ac{MQ} per pesare il costo di una \ac{MQ} dato che è lineare con la lunghezza} e quindi contare quest ultime può essere sufficiente per analizzare le prestazioni dell'algoritmo. In quest'ottica in \cite[p. 20]{Howar12} si trova un'analisi dell' \ac{ObP} nelle sue varie forme per le  Mealy machines confrontando i risultati con L*. Questi risultati vengono qui applicati ai \ac{DFA} e riportati nella tabella \ref{tab:mqa}. Dato che per tutti gli algoritmi il numero di \ac{EQ} è limitato da $\mathcal{O}(n)$ l'analisi delle \ac{EQ} viene omessa. Infine si noti come l'analisi venga effettuata nel caso peggiore.

Per tutti gli approcci il numero di suffissi necessari per distinguere tutti gli stati è limitato da $n$ dato che ogni aggiunta di un suffisso determina almeno uno \textit{split} e l'individuazione di almeno un nuovo stato (stati che sono al massimo n). Il numero di prefissi in tutte le componenti per \ac{ObP} è al più $n \cdot k$ perchè al più si hanno n componenti (n stati), se da ogni stato si hanno k transizioni il numero di transizioni totali sarà $n \cdot k$ (i prefissi di uno stato sono le transizioni che finiscono in quello stato). La dimensione della tabella di osservazione o meglio la somma delle dimensioni di tutte le tabelle di osservazione di ogni componente è allora $nk \cdot n$ (numero prefissi $\cdot$ numero suffissi). Per AllGlobally si aggiungono tutti gli $m$ suffissi di ogni controesempio, e lo si fa n volte quindi si avranno $n \cdot m$ suffissi con una tabella di dimensione $nk \cdot nm$. Per OneLocally e OneGlobally vi è da aggiungere il numero di query fatte durante la ricerca del suffisso che nel caso della ricerca binaria è $log_{2}(m)$ effettuate n volte. Per la dimensione della tabella di L* si faccia riferimento alla sottosezione \ref{subsub:comcom}.

\begin{table}[htp]
\centering
\[ 
\begin{array}{lcc} 
\toprule
\text{Algoritmo} & \text{Dimensione tabella} & \text{Membership queries}  \\
\midrule  
\text{L*}  &  (k + 1)\cdot(n + m(n − 1))\cdot n & \mathcal{O}(n^{2}km) \\
\text{AllGlobally}  &  nk\cdot nm & \mathcal{O}(n^{2}km) \\
\text{OneLocally}  &  nk\cdot n & \mathcal{O}(n^{2}k) \\
\text{OneGlobally}  &  nk\cdot n & \mathcal{O}(n^{2}k) \\

\bottomrule
\end{array}
\]
 \caption[Complessità membership queries ObP]{Complessità delle membership queries per le differenti varianti di ObP}
\label{tab:mqa}
\end{table}  

Tuttavia i risultati evidenziati nella tabella \ref{tab:mqa} ed anche quelli omessi per le \ac{EQ} costituiscono solo un limite superiore, il lavoro sperimentale in \cite[p. 22-23]{Howar12} mette meglio in evidenza le differenze tra i vari algoritmi. OneLocally è l'algoritmo con il minore numero di \ac{MQ} dato che il suffisso discriminante è aggiunto solo ad un componente e quindi sarà necessario completare solo la tebella di osservazione di questo componente ,e nonostante OneLocally e OneGlobally abbiano lo stesso ordine di grandezza per il numero di \ac{MQ} fatte al crescere dei parametri la differenza in termini di \ac{MQ} è notevole. In OneLocally tuttavia come detto in \cite[p. 23]{Howar12} il numero delle \ac{EQ} aumenta drasticamente rispetto a OneGlobally anche se in numero  pur sempre limitato dal numero di stati del \ac{DFA} minimo equivalente ad \ac{L}. L* effettua un numero di \ac{EQ} quasi sempre minore (seppur di poco) di OneGlobally ma un numero di \ac{MQ} sempre molto maggiore sia a OneLocally che a OneGlobally. AllGlobally consente di effettuare un numero di \ac{EQ} quasi sempre leggermente inferiore anche ad L* ma il numero di \ac{MQ} è sempre molto elevato anche rispetto ad L*.
\subsection{Discrimination tree vs Tabella di Osservazione localizzata}
 In figura \ref{sub:tol} vi è la rappresentazione della tabella di osservazione per
 \begin{figure}[htp]
\centering
\subfloat[\emph{Tab. Osservazione L*}][\label{sub:tol}]
{\includegraphics[width=.45\textwidth,height=6cm,keepaspectratio]{OBTLSTAR}} \quad
\subfloat[\emph{DFA ipotesi}.][\label{sub:dfah}]
{\includegraphics[width=.45\textwidth]{DFALstar}}
\caption{DFA ipotesi e corrispondente tabella di osservazione in L*}
\label{fig:dfal}
\end{figure} 
l'algoritmo L* per il \ac{DFA} in figura \ref{sub:dfah}. I prefissi RED distinti tra di loro $\{\epsilon,a,b\}$ (cioè con row() distinte tra di loro) sono gli short prefix delle classi di equivalenza che hanno una corrispondenza con gli stati del \ac{DFA}. I prefissi BLUE sono invece delle stringhe appartenenti ad una delle classi di equivalenza (un prefisso BLUE x termina nello stato rappresentato da uno stato RED r ed appartiene alla sua classe di equivalenza se row(r)=row(x) ). Il \ac{DT} è una rappresentazione della tabella di osservazione scevra da ridondanze come si può apprezzare in figura \ref{sub:dis}. Infatti a parte lo short prefix superfluo bb,  nella tabella di osservazione di L* non tutti i suffissi sono necessari per distinguere le varie classi di equivalenza (gli stati RED). Ad esempio il singolo suffisso  $\epsilon$ da solo è sufficiente per distinguere $[\epsilon]$ dalle altre classi di equivalenza. Lo scopo di un \ac{DT} è di eliminare queste ridondanze ed organizzare le osservazioni delle passate \ac{MQ} in una maniera efficiente in modo da consentire efficientemente il \textit{sifting} di una stringa.
 \begin{figure}[htp]
\centering
\subfloat[\emph{Discrimination Tree}][\label{sub:dis}]
{\includegraphics[width=.45\textwidth,height=6cm,keepaspectratio]{DTLSTAR}} \quad
\subfloat[\emph{Split}.][\label{sub:split}]
{\includegraphics[width=.45\textwidth]{Split}}
\caption{Discrimination Tree e split di uno stato}
\label{fig:diss}
\end{figure} 
Nella tabella \ref{tab:com} vi è l'insieme di componenti di \ac{ObP} corrispondente alla tabella di osservazione per OneGlobally anche se è solo un esempio e molti dei prefissi presenti in L* potrebbero non essere presenti in \ac{ObP}. 
\begin{table}[htp]
\centering 
\begin{tabular}{|l|c|c|} 
\hline
$C_{\epsilon}$ & $\epsilon$ & $a$  \\
 \hline  
$\epsilon$  &  1 & 0 \\
\hline 
aa  &  1 & 0 \\
bb  &  1 & 0 \\
\hline
\end{tabular}
\quad
\begin{tabular}{|l|c|c|} 
\hline
$C_{a}$ & $\epsilon$ & $a$  \\
\hline
$a$ & 0 & 1  \\
 \hline  
$bba$  &  0 & 1 \\
\hline
\end{tabular}
\quad
\begin{tabular}{|l|c|c|} 
\hline
$C_{b}$ & $\epsilon$ & $a$  \\
\hline
$b$ & 0 & 0  \\
 \hline  
$ab$  &  0 & 0 \\
$ba$  &  0 & 0 \\
$bbb$  &  0 & 0 \\
\hline
 \end{tabular}
 \caption[Insieme di componenti]{Insieme di componenti}
\label{tab:com}
\end{table}  
 
Sia il \ac{DT} che $C_I$ sono rappresentativi dell'ipotesi e quindi vi deve essere in ogni momento coerenza anche tra \ac{DT} e $C_I$ oltre che con l'ipotesi. Infatti per ogni foglia nell'albero c'è un corrispondente componente  in $C_I$, l'etichetta della foglia corrisponde all'access sequence di quel componente. Entrambe le strutture dati sono rappresentative dell'ipotesi.
A questo punto ci si potrebbe chiedere perchè utilizzare sia un \ac{DT} che $C_I$ per rappresentare l'ipotesi. Le considerazioni da fare differiscono tra OneLocally e OneGlobally. Quando si fa il sift di una stringa che termina in un nodo foglia si aggiungerà quella stringa al componente corrispondente a quel nodo foglia come detto nella sottosezione \ref{sub:fun} .Tuttavia in OneGlobally l'insieme V di suffissi è globale e non coincide con i suffissi (che costituiscono un sottoinsieme di tutti i suffissi globali V),etichette dei nodi interni, incontrati durante la navigazione del \ac{DT} per arrivare alla foglia summenzionata. Quindi vi è l'evenienza che tramite il \textit{sift} di una stringa venga introdotta un'ulteriore non chiusura (ulteriore perchè sicuramente la non chiusura viene prodotta in prima istanza dal suffisso estrapolato dal controesempio) e conseguente \textit{split}. Se non avessimo l'insieme di componenti sarebbe arduo esaminare la non chiusura nello scenario appena descritto da cui l'esigenza di $C_I$. Si noti inoltre che quanto appena descritto è il motivo per cui il numero di \ac{EQ} non coincide ma è minore del numero degli stati del \ac{DFA}  minimo equivalente ad \ac{L}; infatti ad ogni `` generazione  '' si possono avere più di uno \textit{split} (ogni \textit{split} consente di determinare un nuovo stato ed avvicinarci più velocemente alla soluzione) proprio per il motivo appena descritto. In OneLocally invece i suffissi di un componente $C_{x}$ coincidono esattamente con i suffissi incontrati durante il \textit{sifting} di x nel \ac{DT}. Quando si effettua il sifting di un'altra stringa che tramite il \ac{DT} si scopre essere in $[x]_{\simeq_{OT}}$, dato che giunge nel nodo foglia che ha $\tau(n_{x}) = x$, i suoi suffissi sono esattamente quelli incontrati durante la navigazione del \ac{DT} e quindi non può essere introdotta non chiusura. Quindi in OneLocally si ha esattamente uno \textit{split} ad ogni `` generazione  '' ed il numero di \ac{EQ} coincide con il numero di stati del DFA target minimizzato. In realtà non è così per via dell'ottimizzazione descritta in \ref{sub:fun} che consente di riutilizzare più volte lo stesso controesempio e risparmiare delle \ac{EQ}, ma comunque il numero di \ac{EQ} in OneLocally sarà maggiore del numero di \ac{EQ} in OneGlobally. Va detto che il controllo sulla non chiusura che si effettua in CLOSEPACK alla riga \algref{alg:clp}{lin:clpunclosed} può essere eliminato in OneLocally(il controllo va fatto solo la prima volta in assoluto che CLOSEPACK è chiamato, per le restanti generazioni la non chiusura si verificherà solo per via del suffisso del controesempio). Riassumendo in OneLocally si potrebbe fare a meno dell'insieme di componenti ed utilizzare esclusivamente il \ac{DT}. Tuttavia utilizzare anche l'insieme di componenti in ogni caso comporta dei vantaggi prestazionali per alcune operazioni da fare in\ac{ObP}. La costruzione di \ac{H} prevede di trovare lo short prefix associato ad una dato prefisso (ottenuto concatenando un altro short prefix con un simbolo di $\Sigma$), quest'operazione è più efficiente tramite $C_{I}$ dato che nel \ac{DT} comporta il \textit{sifting} del prefisso e quindi un numero di \ac{MQ} nel caso medio logaritmico  da moltiplicare per la dimensione media dei suffissi incontrati. E ancora nella decomposizione del controesempio vi è l'esigenza di trovare dato un prefisso, lo si chiami p,il corrispondente short prefix di p ma può accadere che p non sia mai stato visto o che facendone il \textit{sift} nel \ac{DT} non termini in una foglia già esistente ma invece provochi la creazione di un nuovo nodo foglia. Quindi non riusciamo a trovare lo short prefix associato a p senza l'ausilio di $C_I$, anche se in realtà è pur sempre possibile effettuare il  `` parsing  '' della stringa x in \ac{H} e tornare lo short prefix associato allo stato cui si arriva in \ac{H}. In realtà lo sforzo computazionale richiesto per mantenere un'ipotesi anche tramite l'insieme di componenti potrebbe essere maggiore dei vantaggi prestazionali ottenuti rispetto ad un'implementazione che usa esclusivamente il \ac{DT} per rappresentare \ac{H} nel caso di OneLocally. Si ribadisce invece che nel caso di OneGlobally  (e anche AllGlobally) ,per i motivi suddetti, $C_I$ è necessario per testare la chiusura e garantire un corretto funzionamento.
\subsection{Teacher} Tutte le considerazioni effettuate nella sottosezione \label{sub:tea} inerenti il \textit{teacher} di L* rimangono valide anche per l'\ac{ObP} e ad esse si rimanda.  
 
\section{Scelte Progettuali} Si è implementato in C++11 l'algoritmo \ac{ObP} utilizzando la versione OneGlobally (si è implementato anche OneLocally ma la versione di \ac{ObP} inclusa nella libreria Gi-learning è OneGlobally) ed interfacciandosi con la libreria preesistente Gi-learning. L'esposizione e lo pseudocodice riportato nella sezione\ref{sec:alobp} per l'\ac{ObP} fanno riferimento a Howar \cite{Howar12} tranne qualche piccola modifica. Diversamente alcune delle scelte   progettuali per il codice  si discostano in parte da quanto esposto in Howar. Qui saranno illustrate le differenze principali e le ragioni da cui tali scelte sono scaturite.

In prima analisi viene effettuata una preliminare minimizzazione del \ac{DFA} target al fine di permettere un'eventuale\footnote{perchè il \ac{DFA} può essere già minimo} velocizzazione del \textit{testing} dell'equivalenza tra \ac{H} ed il \ac{DFA} target, dato che le \textit{performances} dell'algoritmo d'equivalenza \textit{table-filling} dipendono in maniera quadratica dal numero di stati.

E' stata inserita l'ottimizzazione che permette di sfruttare più volte lo stesso controesempio finchè non lo è più. Come detto ciò consente di ridurre il numero di \ac{EQ}. L'ottimizzazione è presente anche nell'implementazionde dell'\ac{ObP} nella \textbf{LearnLib}\footnote{\url{www.learnlib.de} in cui si può trovare sotto la nomenclatura \textit{Discrimination Tree} un'implementazione in Java della'algoritmo \ac{ObP} di Howar}.\\
Non è stata implementata invece l'ottimizzazione \textit{semantic suffix closdness} (vedasi \cite{Stef11}) che consente di ottenere delle ipotesi intermedie minime (che consentono di velocizzare le \ac{EQ}) e di ridurre eventualmente il numero  di \ac{EQ}. L'ottimizzazione è invece presente nell'implementazione  dell'\ac{ObP} nella \textbf{LearnLib}.\\
Si è scelto di completare le componenti nel momento stesso che l'incompletezza è introdotta dato che i punti dove avviene sono ben circostanziati:
\begin{itemize} 
\item dopo l'individuazione del suffisso discriminante, nell'algoritmo e alla riga \algref{alg:obpp}{lin:addsuffixes}, e la sua aggiunta all'insieme globale di suffissi che avviene in una nuova funzione chiamata UPDATE-FROM-COUNTEREXAMPLE. Inoltre in UPDATE-FROM-COUNTEREXAMPLE  dato che --- in corrispondenza dei prefissi discriminati dal suffisso tornato dal metodo di decomposizione del controesempio --- si conosce già l'esito della \ac{MQ} ,perchè già effettuate nel metodo di decomposizione del controesempio, si risparmiano 2 \ac{MQ}. Detto $n$ il numero di stati del \ac{DFA} target minimo ciò consente di risparmiare fino a un massimo di $2\cdot n$ \ac{MQ}.
 \item in OBP-SIFT(), l'algoritmo \ref{alg:sift}. Sia se durante il \textit{sifting} il prefisso termini in un componente esistente oppure in uno nuovo è necessarrio effettuare delle \ac{MQ} per completare il componente esistente oppure il nuovo componente rispettivamente sull'insieme di suffissi globale.  In questo modo durante il \textit{sifting} sarà possibile evitare di eseguire nuovamente le \ac{MQ} effettuate durante la navigazione del \ac{DT} con quel dato prefisso per tutti i suffissi incontrati durante il \textit{sifting} nel \ac{DT}. Resta inteso che utilizzando la strategia OneGlobally non tutti i suffissi contenuti nell'insieme di suffissi globale saranno incontrati e per i restanti alla fine sarà necessario effettuare una \ac{MQ} esplicita per assicurare la chiusura del componente. 
\end{itemize}
OBT-SPLIT non introduce incompletezza.\\ La summenzionata gestione della completezza oltre a consentire di risparmiare delle \ac{MQ}  implica anche una migliore efficienza dell'algoritmo dato che alla riga \algref{alg:clp}{lin:clpunclosed}  l'algoritmo originale OBT-CLOSEPACK deve controllare se $C_I$ è completo e ciò comporta un controllo su tutte le componenti, controllo non più necessario.\\
Un'altra modifica significativa rispetto all'algoritmo OBT-CLOSEPACK originale (algoritmo \ref{alg:clp}) sta nell'individuazione del suffisso che determina la non chiusura, alla riga \ref{lin:clpsuffix}. Infatti il metodo di decomposizione del controesempio consente di determinare quali sono i due prefissi ed il suffisso che determina la non chiusura. Quindi ogni volta che viene chiamato OBT-CLOSEPACK non è necessario spendere del tempo di esecuzione nel cercare dove non si verifica la non chiusura. Questo è vero ad ogni prima chiamata di OBT-CLOSEPACK, può poi accadere che venga generata ulteriore non chiusura all'interno dell'algoritmo OBT-CLOSEPACK stesso che in questo caso va individuata. C'è da precisare che quanto detto non è valido per la prima chiamata in assoluto di OBT-CLOSEPACK dato che la prima chiamata in assoluto non è preceduta da una chiamata al metodo di gestione del controesempio (che come detto permette di individuare esattamente in quale componente e per quale coppia di prefissi e quale suffisso accade la non chiusura). \\\\
Inoltre esistono adesso due versioni del metodo di decomposizione del controesempio BINARY-SEARCH, di cui una è uguale a quella descritta nell' algoritmo  \ref{alg:bin},che è usata dagli altri metodi di decomposizione del controesempio, e l'altra  invece consente di scovare la decomposizione del controesempio \textit{on the fly} cioè durante l'esecuzione dell'algoritmo stesso (cioè all'interno dell'algoritmo di decomposizione del controesempio e quindi non verrà tornato un indice ma la decomposizione stessa).
Nell'appendice,ed esattamente nella sezione \ref{sec:implobp} sono inseriti i dettagli implementativi più significativi.    
%!TEX encoding = UTF-8 Unicode
%!TEX root = ./../main.tex
%!TEX TS-program = xelatex

\chapter{Observation Pack Approssimato} % chapter 6 title
\label{cap:sei}
%!TEX encoding = UTF-8 Unicode
%!TEX root = ./../main.tex
%!TEX TS-program = xelatex

\chapter{Risultati sperimentali} % 5th chapter title
\label{cap:sette}

%%%%%%%%%%%
%    CONCUSIONI    %
%%%%%%%%%%%
\cleardoublepage
\phantomsection
\addcontentsline{toc}{chapter}{Conclusioni}
\chapter*{Conclusioni}
\label{cap:con}

Inserire il discorso che è possibile provare ObPA nel caso si approssimano solo le eq. query e le MQ sono esatte. CHe lo scenario cui si è ricondotti qui è molto simile agli algoritmi di passive learning. Gli algoritmi di active learning invece sono più efficienti proprio perchè hanno piu informazioni e qui riducendo a zero l'informazione aggiuntiva riconducendoci a uno scenario operativo degli algoritmi di passive learning ma sempre inseriti in un algoritmo di active learning si hanno riultati non ottimi. Diminuendo solo parzialmente l'informazione (appr. solo l'eq. come fanno molti lavori in letteratura sicuramente si otterrebbero risultati migliori).

Dato che il W-Method ha una complessità esponenziale (Vedi CHow) esistono metodi simili più efficienti come il Wp-Method e HSI method che si può pensare di implementare per provare a risolvere i problemi col W-Method


\appendix

%%%%%%%%%%
% APPENDICE 1 %
%%%%%%%%%%
%!TEX encoding = UTF-8 Unicode
%!TEX root = ./../main.tex
%!TEX TS-program = xelatex

\chapter{Preliminari} % Main chapter title
\label{app:uno}
Lo scopo di quest'appendice è di stabilire una comune sintassi e semantica per concetti che sono rilevanti in tutta la tesi. Le definizioni e le notazioni qui introdotte sono essenziali per la maggior parte dei capitoli e quindi andrebbero lette.

L'appendice è divisa concettualmente in tre parti. Nella prima parte si introdurranno questioni  puramente matematiche mentre nella seconda parte si definiranno grammatiche e linguaggi, infine nell'ultima parte si tratteranno in dettaglio gli automi a stati finiti. 

\section{Notazione matematica}
L'obiettivo di questa sezione è di introdurre i concetti matematici propedeutici per questa tesi. Senza dubbio una conoscenza matematica di base è necessaria e chiaramente non può essere introdotto ogni singolo elementare concetto.
\subsection{Insiemi}
Con $\mathbb{N}$ si indica l'insieme di numeri naturali interi non negativi incluso 0 (cioè $\mathbb{N} = {0,1,2,\dots}$). L'insieme di interi positivi è denotato da $\mathbb{N}^{+}$ . Si definisce con $\mathbb{B} = \{0,1\}$ l'insieme di valori booleani dove 0 è associato al valore logico \textit{falso} ed 1 al valore logico \textit{vero}.

Dato un generico insieme $X, \abs{X}$ denota la sua cardinalità, cioè il numero di elementi che contiene.
\subsection{Relazione d'equivalenza}
Una relazione binaria riflessiva ,simmetrica e transitiva  $\approx \, \subseteq X\!\times{}\!X$ su un insieme $X$ è detta una \textbf{\textit{relazione d'equivalenza}}. Dato un insieme $X$ ed un elemento $x \in X$ si denota con $[x]_\approx = \{x' \in X\ | \: x \approx x'\}$ la \textit{classe di equivalenza} di x(rispetto alla relazione d'equivalenza $\approx$).

Una relazione d'equivalenza $\approx$ su un insieme \textit{X} si dice che \textit{satura} un sottoinsieme $X' \subseteq X$ se e solo se $X'$ è l'unione di alcune delle classi d'equivalenza di $\approx$. In simboli si ha:
\begin{equation*}
X' = \bigcup_{x \in X'}^{}{\!\![x]_\approx} \text{ ,}
\end{equation*}
e ogni classe di equivalenza $[x]_\approx \text{ di } \approx$ o è un sottoinsieme o è disgiunta da $X'$.\\
Il \textbf{\textit{quoziente}} (o insieme quoziente) di \textit{X} rispetto a una relazione d'equivalenza $\approx$ è definito come l'insieme di tutte le classi d'equivalenza, ed è indicato da $X\!/\!\!\approx  \:=  \{[x]_\approx \: | \: x \in X \}$. L'\textbf{\textit{indice di una relazione d'equivalenza}} $\approx$ è definito come il numero di classi d'equivalenza, cioè è uguale a $\abs{X\!/\!\!\approx}$ .
 Una \textit{partizione} di un insieme $X$ è un insieme $P$ i cui elementi, detti \textit{blocchi} ed indicati con $C$, sono sottoinsiemi (disgiunti e non vuoti) dell'insieme $X$ tali che: 
\begin{enumerate}
\item se $C \in P \text{ allora } C \ne \emptyset$
\item se $C_1,C_2 \in P \text{ e } C_1 \ne C_2 \text{ allora } C_1 \cap C_2  = \emptyset   $
\item se $a \in X \text{ allora esiste } C \in P \text{ tale che } a \in C$ (è un altro modo di dire che l'unione di tutti gli insiemi C deve formare X)
\end{enumerate}  
Il \textit{quoziente} di \textit{X} forma una \textit{partizione} di \textit{X}

\section{Linguaggi e grammatiche}
\subsection{Alfabeto, stringhe e linguaggi}
\subsubsection{Alfabeto}
Di definisce L'\textbf{\textit{alfabeto}} $\Sigma$ un qualsiasi insieme finito e non vuoto di simboli.

\subsubsection{Stringhe}
Una \textbf{\textit{stringa}} è definita come una sequenza di simboli presi da un alfabeto. Cioè una stringa \textit{s} definita su $\Sigma$ è una sequenza $ s = a_1\dots{}a_n$ tale che $a_i \in \Sigma$. 

$\abs{s}$ denota la lunghezza  della stringa s.\\
La \textit{stringa vuota} è indicata con $\epsilon$ e $\abs{\epsilon} = 0$. \\
Con $\Sigma^{*}$ si denota l'insieme di tutte le possibili stringhe ottenibili sull'alfabeto $\Sigma$. Inoltre con $\Sigma^{+}$ si denota l'insieme $\Sigma^{*} - \{\epsilon\}$ \\
I singoli simboli costituenti una stringa $w \in \Sigma^{*}$ sono indicati con $w_i$ con $0 \leq i \le \abs{w}$ quindi $w = w_0 w_1 \dots w_{\abs{w}-1}$ . Per qualche intero nell'intervallo $I \subseteq [0,\abs{w}], w_I$ è la stringa risultante prendendo solo le posizioni in w corrispondenti agli indici in $I$ . Quindi $w_{[0,k)}$ è il prefisso di w di lunghezza k, e $w_{[k,\abs{w})}$ è il suffisso di w che inizia all'indice k(compreso). Si osservi che $w_{[0,0)} = w_{[\abs{w},\abs{w})} = \epsilon$ e $w_{[0,\abs{w})} = w$ .\\
Un \textbf{\textit{prefisso}}  di una stringa $w \in \Sigma^{*}$ è una stringa $u \in \Sigma^{*}$ tale che esiste una stringa $v \in \Sigma^{*}$ soddisfacente $w=uv$. L'insieme di tutti i prefissi di una stringa $w$ si indica con Pref(w)\\
Un \textbf{\textit{suffisso}}  di una stringa $w \in \Sigma^{*}$ è una stringa $v \in \Sigma^{*}$ tale che esiste una stringa $u \in \Sigma^{*}$ soddisfacente $w=uv$. L'insieme di tutti i suffissi di una stringa $w$ si indica con Suff(w)\\
Una \textbf{sottosequenza} di una stringa è una qualsiasi stringa ottenuta rimuovendo dalla stringa di partenza zero o più simboli non necessariamente consecutivi.
\subsubsection{Linguaggi}
Un linguaggio \textit{L} su un alfabeto $\Sigma$ è un qualsiasi sottoinsieme di stringhe di $\Sigma^{*}$\\ 
Si definisce l'insieme dei prefissi di un linguaggio \textit{L}:
\begin{equation*}
 \text{Pref}(L) = \bigcup_{w \in L}^{}{\text{Pref}(w)}
 \end{equation*}
e l'insieme dei suffissi di \textit{L}:
\begin{equation*}
 \text{Suff}(L) = \bigcup_{w \in L}^{}{\text{Suff}(w)}
 \end{equation*}
 Un linguaggio \textit{L} è detto essere \textbf{\textit{prefix-closed}} se e solo se $\text{Pref}(L) = L$ . Informalmente questa proprietà di un linguaggio \textit{L} viene sfruttata per indicare che qualunque stringa appartenente ad \textit{L} si prende, qualunque suo prefisso deve ancora appartene ad \textit{L}. Analogamente un linguaggio \textit{L} è \textbf{\textit{suffix-closed}} se e solo se $\text{Suff}(L) = L$ . \\
 
 La \textbf{\textit{differenza simmetrica}} di due linguaggi $L_1 \text{ e } L_2$ denotata con $L_1 \oplus L_2$ è tale che:
 \begin{equation*}
 L_1 \oplus L_2 = \{x \in \Sigma^{*} : (x \in L_1 \land x \notin  L_2) \lor (x \notin L_1 \land x \in L_2)
 \end{equation*}
 Un linguaggio può essere identificato mediante due tipi di descrizioni:
 \begin{enumerate}
 \item \textbf{Descrizione generativa}\\Consiste nell'utilizzare un formalismo denominato \textbf{grammatica generativa} ,introdotto da Noam Chomsky, che consiste in una serie di simboli e regole mediante le quali è possibile generare tutte e sole le stringhe del linguaggio.
 \item \textbf{Descrizione descrittiva-identificativa}\\Il linguaggio è identificato o tramite un'enumerazione delle stringhe che vi appartengono o tramite una descrizione che cattura le caratteristiche delle sentenze costituenti il linguaggio,ad esempio le espressioni regolari. Un altro sistema formale identificativo sono gli \textbf{automi} .
 \end{enumerate}
 I linguaggi possono essere classificati in base ai due tipi di descrizioni.  Infatti è possibile delineare una tassonomia di grammatiche cui si farà corrispondere una classe di linguaggi. Ad ogni classe di grammatiche corrisponderà una classe di linguaggi (tutti quelli che quella classe di grammatiche è in grado di generare). 
E' possibile effettuare un'analoga corrispondenza tra classi di automi e classi di linguaggi, e quindi anche tra la gerarchia di automi e quella di grammatiche. Si rimanda alla sottosezione \ref{sub:gra} per una formalizzazione di questa gerarchia. 
 
 
 \subsection{Grammatiche}
 \label{sub:gra}
 \begin{definizione*}[Grammatica generativa di Chomsky] Una \textit{\textbf{grammatica generativa di Chomsky}} è una quadrupla:\\
 
 \centerline{$G = (\Sigma, V, S, P)$}
 

 dove:\\
 $\Sigma$ è l'alfabeto, detto insieme di simboli terminali\\
 V è l'insieme di simboli non terminali\\
 S è il simbolo iniziale ed appartiene a V\\
 P è l'insieme delle produzioni costituiti da una testa $\Psi$ (il lato sinistro della produzione) e da una coda $\Omega$ aventi in generale questa forma:\\

 \centerline{$\Psi \rightarrow \Omega \text{ con } \Psi \in (\Sigma \cup V)^{*}V(\Sigma \cup V)^{*} \text{ e } \Omega \in (\Sigma \cup V)^{*}$}
 \end{definizione*}
 Si effettua una classificazione delle classi di grammatiche (e dei linguaggi ad esse associate) imponendo delle restrizioni sulle regole di produzione \cite{Chomsky59}:
 \begin{itemize}
 \item \textbf{Grammatiche di tipo zero - Unresticted} E' la classe di grammatiche più in alto nella gerarchia e per la quale non vi sono regole di restrizione da applicare alle produzioni. Sono in grado di generare la classe di \textbf{linguaggi ricorsivamente enumerabili}. Mediante l'approccio identificativo l'automa che riconosce ed accetta questi linguaggi è la \textbf{Macchina di Turing}
 \item \textbf{Grammatiche di tipo uno - Context Sensitive} Le regole di produzione sono così definite:\\
 \centerline{$\alpha_1 A \alpha_2 \rightarrow \alpha_1 \Omega \alpha_2 \text{ con } \alpha_1 ,\alpha_2 ,\Omega \in (\Sigma \cup V)^{*} \text{ e } A \in V$}
 
  La classe di linguaggi che queste grammatiche sono in grado di generare è detta \textbf{context-sensitive}. Gli automi in grado di riconoscere ed identificare questi linguaggi sono detti \textbf{Linear Bounded Automata}
 \item \textbf{Grammatiche di tipo due - Context Free} Le regole di produzione sono così definite:\\
 \centerline{$A \rightarrow \Omega \text{ con } \Omega \in (\Sigma \cup V)^{*} \text{ e } A \in V$}
 
  La classe di linguaggi che queste grammatiche sono in grado di generare è detta \textbf{context-free}. Gli automi in grado di riconoscere ed identificare questi linguaggi sono detti \textbf{Push Down Automata}
 \item \textbf{Grammatiche di tipo tre - Regular} Le regole di produzione sono così definite:\\
 \centerline{$A \rightarrow \alpha B \text{ oppure } A \rightarrow B\alpha  \text{ con } A \in V, B \in (V \cup \{\epsilon\}) \text{ e } \alpha \in \Sigma^{+}$}
 
 I linguaggi che queste grammatiche generano sono detti \textbf{linguaggi regolari}. Gli automi in grado di riconoscere ed identificare questi linguaggi sono detti \textbf{\ac{FSA}}.    
 \end{itemize}
 
Le diverse classi di linguaggi, e quindi anche di grammatiche ed automi, si includono propriamente in maniera gerarchica come in figura \ref{fig:linger} .
\begin{figure}[htp]
	\centering
	\includegraphics[ width=0.7\textwidth]{LinguaggiGerarchia}
	\caption[Gerarchia di linguaggi]{Gerarchia di linguaggi secondo Chomsky}
   \label{fig:linger}
\end{figure}
Ad una grammatica si può associare un unico linguaggio. Adesso si definirà come avvenie tale associazione:
\begin{definizione*}
Il linguaggio generato da una grammatica $\mathcal{G}$ è l'insieme di tutte le stringhe che possono essere derivate a partire dal simbolo iniziale S:\\

\centerline{$L(\mathcal{G}) = \{x \in \Sigma^{*} :S\xRightarrow[]{\mathcal{G}} x\}$}
\end{definizione*}
E' rilevante notare che come detto una grammatica genera un unico linguaggio ma un linguaggio può essere generato da molteplici grammatiche.\\
Inoltre la tassonomia di Chomsky non è esaustiva di tutti i linguaggi possibili, infatti esistono dei linguaggi che non sono ricorsivamente enumerabili cioè non c'è nessuna macchina di Turing che li riconosce.\\
Infine esistono altre classi di linguaggi che non sono incluse nella classificazione appena esposta e che saranno adoperate nel corso della trattazione:
\begin{definizione*}[Linguaggio Ricorsivo] Un linguaggio $L$ è detto \textbf{ricorsivo} se è decidibile cioè esiste una macchina di Turing $M$ che accetta ogni stringa in $x \in L$ e rigetta ogni stringa $x \not\in L$
\end{definizione*}
I linguaggi regolari e context-free sono ricorsivi. Esistono invece linguaggi context-sensitive non ricorsivi, cioè non ne sono un sottoinsieme \cite[p. 124]{Levelt08}. Tutti  i linguaggi ricorsivi sono ricorsivamente enumerabili (cioè sono linguaggi semidecidibili per i quali una stringa non appartenente al linguaggio può essere sia rigettata che andare in ciclo infinito rispetto a una macchina di Turing) ma non è sempre vero il viceversa.
\begin{definizione*}[Linguaggio primitivo ricorsivo]
Un linguaggio è \textbf{primitivo ricorsivo} quando la sua funzione caratteristica è primitiva ricorsiva.
\end{definizione*}
Il concetto di funzione caratteristica è spiegato nella sezione \ref{sec:FSA}. Invece una funzione primitiva ricorsiva è una funzione definita come una delle funzioni di base o combinando le funzioni di base con operazioni come composizione e ricorsione. Per una definizione formale si rimanda a \cite{Rob47}. I linguaggi primitivi ricorsivi sono un sottoinsieme dei linguaggi ricorsivi.
\begin{definizione*}[Linguaggi superfiniti]
La classe dei linguaggi superfiniti è tale se contiene tutti i linguaggi finiti ed almeno un linguaggio infinito.
\end{definizione*}
 
\section{Automi a stati finiti}
\label{sec:FSA}
Nella sottosezione \ref{sub:gra} sono stati introdotti gli automi e una loro classificazione. Qui ci si concentrerà sullo studio dei \ac{FSA} in relazione ai linguaggi regolari, la classe dei linguaggi a cui questa tesi è rivolta. Come visto gli \ac{FSA} sono un caso speciale di \textit{macchina di Turing} e più nello specifico un caso speciale di \ac{FSM} che in questa sede non interessa definire. Qui basterà dire che   una \ac{FSM} è un \textit{transiction system} costituita da un insieme finito di stati dove ogni transizione è innescata da un'azione tra un insieme finito di azioni (di solito denotato da $\Sigma$). Esistono diversi tipi di \ac{FSM} come le \textit{Mealy Machines} e gli \ac{FSA}. Tra gli \ac{FSA} si annoverano gli \ac{NFA} strettamente correlati ai \ac{DFA} su cui si focalizzerà l'attenzione.
\begin{definizione}[Automa a stati finiti deterministico]
\label{def:dfa}
Un \ac{DFA} A è una quintupla:\\

\centerline{$A = \Braket{\Sigma,Q^{A},q_{\epsilon}^{A},\delta^{A}, \mathbb{F}_{\mathbb{A}}^{A}}$}
\end{definizione}
dove:\\
$\Sigma$ è un alfabeto\\
$Q^{A}$ è un insieme finito di stati\\
$q_{\epsilon}^{A} \in Q^{A}$ talvolta indicato come $q_{\lambda}^{A}$ è lo stato iniziale\\
$\delta^{A} : Q^{A} \times \Sigma \to Q^{A} $ è la funzione di transizione\\
$\mathbb{F}_{\mathbb{A}}^{A} \subseteq Q^{A}$ è l'insieme degli stati accettanti\\

Inoltre nel corso della trattazione seguendo \cite[p. 72]{DeLaHiguera10} in alcuni casi è conveniente utilizzare anche un'altra definizione per i \ac{DFA} uguale a quella appena data ma comprendente anche un nuovo insieme $\mathbb{F}_{\mathbb{R}}^{A} \subseteq Q^{A}$ che è l'insieme degli stati rigettanti. Quando si parlerà di \ac{DFA} si farà sempre riferimento alla prima definizione,quella classica,  a meno che non è specificato o l'utilizzo della seconda definizione risulta tacitamente evidente dall'utilizzo dell'insieme $\mathbb{F}_{\mathbb{R}}^{A}$. Si indica con $\norma{A} = \abs{Q^{A}}$. 
Inoltre in molti frangenti è conveniente utilizzare (questo è un discorso che esula dalla definizione di un \ac{DFA}) una versione estesa della funzione di transizione a una stringa anzichè ad un solo simbolo dell'alfabeto. Si definisce allora $\hat{\delta}^{A} : Q^{A} \times \Sigma^{*} \to Q^{A}$ definendo induttivamente $\hat{\delta}^{A}(q,\epsilon) = q \text{ e } \hat{\delta}^{A}(q,aw) = \hat{\delta}^{A}(\delta^{A}(q,a),w) \text{ per } q \in Q^{A} \text{ e } aw \in \Sigma^{+} \text{ e } w \in \Sigma^{*}$. Per la funzione di transizione estesa nel corso della tesi sarà anche usata interscambiabilmente la definizione $A[w] = \hat{\delta}^{A}(q_{\epsilon}^{A},w)$ . Inoltre si estende quest'ultima notazione agli insiemi di stringhe: $W \subseteq \Sigma^{*} \text{ | } A[W] = \{A[w] : w \in W\}$ .\\
 
Una stringa \textit{x} è detta essere accettata da un \ac{DFA} \textit{A} se e solo se $\hat{\delta}^{A}(q_\epsilon,x) = q'$ tale che $q' \in \mathbb{F}_\mathbb{A}^{A}$ che significa che usando la funzione di transizione estesa $\hat{\delta}^{A}$ a partire dallo stato iniziale è possibile arrivare ad uno stato accettante.   Il linguaggio individuato da un \ac{DFA} A è allora:\\
\centerline{$L(A) = \{x \in \Sigma^{*} : \hat{\delta}^{A}(q_\epsilon,x) \in \mathbb{F}_\mathbb{A}^{A}\}$}

Una relazione analoga a quanto visto tra linguaggi e grammatiche sussiste tra linguaggi e \ac{DFA}: un \ac{DFA} induce un solo linguaggio regolare, ma ad un linguaggio regolare corrispondono più \ac{DFA}.

In molti contesti è utile rifersi alla \textit{funzione di output}  di un \ac{DFA} A, $\lambda^{A}$ come:\\\\
\centerline{$
\lambda^{A} : \Sigma^{*} \to \mathbb{B}, \quad \forall w \in \Sigma^{*}\quad\lambda^{A}(w) = \begin{cases}
1
& \text{se $w \in L(A)$} \\
0 & \text{altrimenti}\\
\end{cases}
$}\\\\
La \textit{funzione di output} è la \textit{funzione caratteristica} di L(A).Inoltre $\lambda^{A}$ appena definita sopra può essere vista come un caso particolare, per $q=q_\epsilon^{A}$, di $\lambda_{q}^{A}(w)$ che assume valore  1 se $ \hat{\delta}^A(q,w) \in \mathbb{F}_{\mathbb{A}}^{A}$ . Ancora, due \ac{DFA} ,A e A' , sono \textbf{equivalenti} denotato da $A \cong A'$, se loro hanno la stessa \text{funzione di output}, cioè se $\lambda^{A} = \lambda^{A'}$, cioè se indivividuano lo stesso linguaggio. Il concetto di equivalenza è rilevante anche tra gli stati dello stesso \ac{DFA} ed informalmente due stati sono equivalenti se non esiste nessuna stringa che li distingue, cioè che partendo da quei due stati porta a stati di arrivo che sono uno accettante e l'altro no.
\begin{definizione*}[Stati equivalenti]
Detto A essere un \ac{DFA}, e q e p $\in Q^{A}$ stati di A. q e p sono \textit{equivalenti} ,denotato da $q \equiv p$, se $\lambda_{q}^{A} = \lambda_{p}^{A}$
\end{definizione*}
Una stretta correlazione tra l'equivalenza di stati e l'equivalenza di \ac{DFA} è il seguente risultato: $A \cong A' \Leftrightarrow q_{\epsilon}^{A} \equiv q_{\epsilon}^{A'}$.
Un altro concetto importante è quello di \textit{isomorfismo} tra due DFA.
\begin{definizione*}[Isomorfismo di \ac{DFA}]
Detti A ed $\text{A}'$ due \ac{DFA} definiti su $\Sigma$. A ed $\text{A}'$ sono detti isomorfici se esiste un isomorfismo $f : Q^{A} \to Q^{A'}$, cioè una funzione soddisfacente le seguenti condizioni:
\begin{enumerate}
\item $f(q_{\epsilon}^{A}) = q_{\epsilon}^{A'}$
\item $\forall q \in Q^{A} : q \in F_{\mathbb{A}}^{A} \Leftrightarrow f(q) \in F_{\mathbb{A}}^{A'}$
\item $\forall q \in Q^{A}, a \in \Sigma : f(\delta^{A}(q,a)) = \delta^{A'}(f(q),a)$
\end{enumerate} 
\end{definizione*}
L'isomorfismo è un requisito più forte dell'equivalenza: \ac{DFA} isomorfi sono pure equivalenti, ma in generale non è vero il contrario. Quindi dato un linguaggio L esistono più \ac{DFA} in grado di riconoscerlo cioè con la stessa \textit{funzione di output} $\lambda$. Tra questi di particolare interesse sono quelli con il minor numero di stati. Un \ac{DFA} A è detto \textbf{minimo} se qualunque altro \ac{DFA} $\text{A}'$ tale che $A \cong A'$ (con la stessa \textit{funzione di output}) soddisfa $\abs{Q^{A'}}\geq\abs{Q^{A}}$. Ovviamente in un \ac{DFA} minimo nè stati irragiungibili ne stati equivalenti possono essere presenti perchè potrebbero essere eliminati senza cambiare la funzione di output $\lambda$. Inoltre il DFA minimo è sempre unico a meno di una possibile rinomina degli stati. Per motivi storici esiste anche la definizione di \ac{DFA} \textbf{canonico} che è intercambiabile con quella di \ac{DFA} minimo ma entrambe indicano lo stesso ente matematico e sono del tutto equivalenti. Formalizzando
\begin{definizione*}[DFA minimo/canonico]
Detto A essere un \ac{DFA} su $\Sigma$. A è detto canonico se le seguenti condizioni sono verificate:
\begin{enumerate}
\item Tutti gli stati sono raggiungibili: $A[\Sigma^{*}] = Q^{A}$
\item Tutti gli stati sono a coppie separabili\footnote{due stati sono separabili o distinti se non sono equivalenti}: $\forall q \ne p \in Q^{A} : \exists w \in  \Sigma^{*} : \lambda_{q}^{A}(w) \ne \lambda_{p}^{A}(w)$
\end{enumerate}
\end{definizione*}
Per ogni DFA esiste sempre ,ed è unico (a meno delle etichette degli stati) un \ac{DFA} equivalente  che è canonico.
\subsection{FSA particolari}
Alcuni \ac{DFA} ed \ac{NFA} sono particolarmente significativi e ricorreranno spesso nell'ambito di questa tesi. Inoltre è necessario conoscere per il proseguio della trattazione qual è la differenza principale tra \ac{DFA} ed \ac{NFA}. Un \ac{NFA} è detto non deterministico perchè la sua funzione di transizione può avere più di una transizione per un dato simbolo dell'alfabeto (ed inoltre vi possono essere transizioni anche in corrispondenza di $\epsilon$). Inoltre in un \ac{NFA} non vi è necessariamente per ogni stato una transizione in corrispondenza di ogni simbolo dell'alfabeto.
\subsubsection{Maximal Canonical Automaton}
\begin{definizione}[Maximal Canonical Automaton]
Detto $I_+ = \{x_1,\cdots ,x_N\}$ un insieme di esempi positivi, si definisce \textit{\textbf{Maximal Canonical Automaton rispetto ad $I_+$}} e si denota con \textit{\textbf{MCA($I_+$)}} un \ac{NFA} costituito da una quintupla $\Braket{\Sigma , Q , q_\epsilon , \delta , \mathbb{F}_\mathbb{A}}$ dove:\\\\
$\Sigma$ è l'alfabeto su cui è definito $I_+$\\
$Q = \{q_{u}^{i} : u \in \text{Pref}(x_i) \land u \ne \epsilon \} \cup \{q_\epsilon\}, 1 \leq i \leq N$\\
$q_\epsilon = \{q_\epsilon\}$\\
$\delta(q_{u}^{i},a) = \{q_{ua}^{i} : ua \in \text{Pref}(x_i)\}, \forall a \in \Sigma, 1 \leq i \leq N$\\
$\delta(q_\epsilon,a) = \{q_{a}^{i} : a \in \text{Pref}(x_i)\}, \forall a \in \Sigma, 1 \leq i \leq N$\\
$1 \leq i \leq N, q_{x_{i}}^{i} \in \mathbb{F}_\mathbb{A} $\\
se $\epsilon \in I_+$ aggiungere $q_\epsilon \text{ ad } \mathbb{F}_\mathbb{A}$
\end{definizione}
Un \textit{MCA($I_+$)} per ogni stringa di $I_+$ ha un percorso dedicato a partire dallo stato iniziale.
Si noti che nella definizione data di $\delta$ accade che per molti simboli di $\Sigma$ non c'è la corrispondente transizione per un dato stato. Inoltre se in $I_+$ sono presenti stringhe che hanno lo stesso simbolo iniziale si avrà indeterminismo sullo stato iniziale $q_\epsilon$. Quindi in generale il \textit{MCA($I_+$)} è un \ac{NFA}.Un esempio è dato in figura \ref{fig:MCA}\\\\
\begin{figure}
\centering
\begin{tikzpicture}[shorten >=1pt,node distance=2cm,on grid,auto] 
   \node[state,initial] (q_0)   {$\varepsilon$}; 
   \node[state] (q_1) [above right=of q_0] {$a$}; 
   \node[state] (q_2) [below right=of q_0] {$a$}; 
   \node[state] (q_3) [right=of q_1] {$p$};
   \node[state,accepting] (q_4) [right=of q_3] {$e$};
   \node[state](q_5) [right=of q_2] {$t$};
   \node[state](q_6) [right=of q_5] {$o$};
    \node[state](q_7) [right=of q_6] {$l$};
    \node[state](q_8) [right=of q_7] {$l$};
    \node[state,accepting](q_9) [right=of q_8] {$o$};
    \path[->] 
    (q_0) edge  node {$a$} (q_1)
           edge  node {$a$} (q_2)
    (q_1) edge  node  {$p$} (q_3)
    (q_3) edge  node  {$e$} (q_4)
    (q_2) edge  node  {$t$} (q_5)
    (q_5) edge  node  {$o$} (q_6) 
    (q_6) edge  node  {$l$} (q_7)
    (q_7) edge  node  {$l$} (q_8) 
    (q_8) edge  node  {$o$} (q_9); 
\end{tikzpicture}
\caption[Maximal Canonical Automaton]{\textit{MCA($I_+$)} per $I_+=\{\text{ape,atollo}\}$}
\label{fig:MCA}
\end{figure}

\subsubsection{Automa quoziente}
\label{subsub:aqu}
Sia A un \ac{DFA} su $\Sigma$ e sia  $\approx \: \subseteq Q^{A}\!\!\times{}\!Q^{A}$ una relazione d'equivalenza sull'insieme $Q^{A}$ soddisfacente le seguenti due condizioni:
\begin{enumerate}[label=(\roman*)]
\item $\approx$ satura $\mathbb{F}_{\mathbb{A}}^{A}$
\item $\forall q,p \in Q^{A} : q \approx p \Rightarrow (\forall a \in \Sigma : \delta^{A}(q,a) \approx \delta^{A}(p,a) )$ 
\end{enumerate}
Allora da A tramite $\approx$ è possibile ricavare il \textit{\textbf{DFA quoziente}} \textbf{$A/\!\!\approx$} così definito:
\begin{enumerate}
\item $\Sigma$ è lo stesso di $A$
\item $Q^{A/\!\approx} = Q^{A}/\!\!\approx$
\item $q_\epsilon^{A/\!\approx} = [q_\epsilon^{A}]_\approx$
\item $\mathbb{F}_\mathbb{A}^{A/\!\approx} = \{[q]_\approx : q \in \mathbb{F}_\mathbb{A}^{A}\}$
\item $\delta^{A/\!\approx}([q]_\approx,a) = [\delta^{A}(q,a)]_\approx \quad \forall q \in Q^{A},a\in{\Sigma}$
\end{enumerate}
L'automa quoziente $A/\!\!\equiv$ ,dove la relazione d'equivalenza utilizzata è quella di equivalenza tra gli stati, corrisponde al \ac{DFA} canonico. Quindi banalizzando si può concludere dicendo che l'automa quoziente di un \ac{DFA} è ciò che si ottiene fondendo insieme alcuni stati del \ac{DFA} di partenza in base a una relazione d'equivalenza. Quando la relazione usata è quella di stati equivalenti si ottiene il DFA minimo: quindi $Q^{A/\!\approx}$ sarà una partizione in cui in ogni blocco (sottoinsieme) ci saranno stati equivalenti tra loro (in uno specifico blocco). Ogni blocco è una classe d'equivalenza.

\subsubsection{Prefix Tree Acceptor}
Si è visto che l'automa quoziente che si ottiene usando la relazione di equivalenza degli stati su un \ac{DFA} A è il \ac{DFA} minimo. Analogamente è possibile ottenere il \textit{\textbf{Prefix Tree Acceptor di $I_+$}} indicato con \textit{\textbf{PTA$(I_+)$}} applicando la definizione di automa quoziente al \textit{MCA$(I_+)$}\footnote{Anche se tecnicamente può essere un \ac{NFA} la definizione di automa quoziente può essere applicata comunque} con la relazione: \textit{stati che identificano lo stesso prefisso}. Quindi verrà effettuata la fusione degli stati che condividono lo stesso prefisso.\\
Si definisce la relazione d'equivalenza come:
\begin{equation*}
p \approx q \Leftrightarrow \text{ Prefix}(p) = \text{ Prefix}(q)
\end{equation*}
allora:
\begin{equation*}
MCA(I_+)/\!\!\approx \,\,=\, PTA(I_+)
\end{equation*}
In figura \ref{fig:PTA} il \textit{PTA} ricavato a partite dal \textit{MCA} di figura \ref{fig:MCA} .
\begin{figure}[htp]
\centering
\begin{tikzpicture}[shorten >=1pt,node distance=2cm,on grid,auto] 
   \node[state,initial] (q_0)   {$\varepsilon$}; 
   \node[state] (q_1) [right=of q_0] {$a$}; 
   \node[state] (q_3) [right=of q_1] {$p$};
   \node[state,accepting] (q_4) [right=of q_3] {$e$};
   \node[state](q_5) [below right=of q_1] {$t$};
   \node[state](q_6) [right=of q_5] {$o$};
    \node[state](q_7) [right=of q_6] {$l$};
    \node[state](q_8) [right=of q_7] {$l$};
    \node[state,accepting](q_9) [right=of q_8] {$o$};
    \path[->] 
    (q_0) edge  node {$a$} (q_1)
    (q_1) edge  node  {$p$} (q_3)
               edge  node  {$t$} (q_5)
    (q_3) edge  node  {$e$} (q_4)
    (q_5) edge  node  {$o$} (q_6) 
    (q_6) edge  node  {$l$} (q_7)
    (q_7) edge  node  {$l$} (q_8) 
    (q_8) edge  node  {$o$} (q_9); 
\end{tikzpicture}
\caption[Prefix Tre Acceptor]{\textit{PTA($I_+$)} per $I_+=\{\text{ape,atollo}\}$}
\label{fig:PTA}
\end{figure}

\subsubsection{Automa Universale}
Si indica con \textit{\textbf{UA l'automa universale}} che accetta tutte le stringhe definite su $\Sigma$. Si ha $L(UA) = \Sigma^{*}$. Ha un unico stato,che è accettante, con un \textit{self-loop} per ogni simbolo dell'alfabeto.

\subsection{Funzioni di output regolari}
In una sottosezione di \ref{subsub:aqu}  si è visto che è possibile ottenere il  \ac{DFA} canonico minimo di un \ac{DFA} A tramite $A/\!\!\equiv$ (l'automa quoziente sulla relazione di stati equivalenti). In questa sezione invece si vedrà come ottenere il \ac{DFA} canonico minimo non a partire da un \ac{DFA} preesistente, ma semplicemente sfruttando le proprietà di una funzione di output $\lambda : \Sigma^{*} \to \mathbb{B}$ che è la funzione caratteristica di qualche linguaggio regolare.
\subsubsection{Relazione di Nerode}
Si possono caratterizzare le \textit{funzioni di output regolare} come la classe di funzioni $\lambda : \Sigma^{*} \to \mathbb{B}$ per cui un \ac{DFA} con quella \textit{funzione di output} esiste\footnote{L'insieme delle funzione di output regolari identifica la classe dei linguaggi regolari}.  Il famoso teorema \textit{Myhill-Nerode} \cite{Ner58} fornisce una caratterizzazione alternative delle \textit{funzioni di output regolari} , che non fa affidamento sulla nozione di \ac{DFA}. Come primo step si definisce la \textit{relazione di Nerode} \cite{Ner58} sulle stringhe che definisce un'equivalenza sulle stringhe secondo $\lambda$:
\begin{definizione}[Relazione di Nerode]
\label{def:ner}
Sia $\lambda : \Sigma^{*} \to \mathbb{B}$ una funzione di output a due valori arbitraria definita su $\Sigma$. Due stringhe $u, u' \in \Sigma^{*}$ sono equivalenti secondo $\simeq_\lambda$\footnote{Si noti che per denotare l'equivalenza non si è usato il simbolo $\equiv$ (equivalenza tra stati) perchè qui si parla di equivalenza tra stringhe} denotato da $u \simeq_\lambda \!u'$ se e solo se:
\begin{equation*}
\forall v \in \Sigma^{*} \quad \lambda(uv) = \lambda(u'v)
\end{equation*}
dove $\simeq_\lambda \, \subseteq \Sigma^{*}\!\!\!\times\!\!\Sigma^{*}$ è una relazione binaria detta \textit{relazione di Nerode} o \textit{congruenza di Nerode} che definisce l'equivalenza tra stringhe secondo $\lambda$
\end{definizione}
\subsubsection{Teorema Myhill-Nerode}
La \textit{relazione di Nerode} $\simeq_\lambda$ può essere vista come l'equivalente a livello di stringhe della relazione di equivalenza $\equiv_A \subseteq Q^{A}\!\!\times\!Q^{A}$, sugli stati di un \ac{DFA} A. Dovrebbe essere osservato, tuttavia , che $\simeq_\lambda$ può essere definito per \textit{funzioni di output} arbitrarie, non solo regolari. Il teorema \textit{Myhill-Nerode}  fornisce una caratterizzazione delle \textit{funzioni di output regolari} basata su  $\simeq_\lambda$ :
\begin{teorema}[Teorema Myhill-Nerode o di caratterizzazione]
\label{teo:m-n}
Sia $\lambda : \Sigma^{*} \to \mathbb{B}$ una \textit{funzione di output} a due valori. $\lambda$ è regolare se e solo se la \textit{relazione di Nerode}  $\simeq_\lambda$ ha indice finito. 
\end{teorema}
Una dimostrazione del teorema si trova in \cite{Stef11}. L'implicazione in uno dei due versi del teorema dice che se $\simeq_\lambda$ ha indice finito, allora $\lambda$ è \textit{una funzione di output regolare} cioè esiste un \ac{DFA} A con $\lambda^{A} = \lambda$. Questo \ac{DFA} $A = \Braket{\Sigma,Q^{A},q_{\epsilon}^{A},\delta^{A},F_{\mathbb{A}}^{A}}$ è definito come:
\begin{itemize}
\item $\Sigma \text{ è il dominio di } \lambda$
\item $Q^{A} = \Sigma^{*}/\!\!\simeq_{\lambda}$
\item $q_{\epsilon}^{A} = [\epsilon]_{\simeq_{\lambda}}$
\item $F_{\mathbb{A}}^{A} = \{[u]_{\simeq_{\lambda}} | \: \lambda(u)=1\}$
\item $\delta^{A}([u]_{\simeq_{\lambda}} , a) = [u\!\cdot{}\!a]_{\simeq_{\lambda}}$ 
\end{itemize}
A è il \ac{DFA} minimo corrispondente al linguaggio identificato da $\lambda$.Si osservi come la costruzione del \ac{DFA} A è molto simile alla costruzione del \ac{DFA} minimo  usando la relazione di equivalenza sugli stati $\equiv$ usando l'automa quoziente a partire da un \ac{DFA}. Questo approccio alla costruzione degli automi è fondamentale nell'\textit{active learning}.

\chapter[Prel. e impl. ObP]{Preliminari e implementazione dell'Observation Pack}
\label{app:due}
Qui si presentano delle ulteriori notazioni inerenti prevalentemente il capitolo \ref{cap:quattro} e vengono svelati alcuni dettagli implementativi dell'\ac{ObP}.
\section{Notazione specifica per l'ObP}
In questo sezione vi è la delineazione di un'ulteriore notazione utilizzata principalmente nel capitolo \ref{cap:quattro} in congiunzione a quella introdotta in \ref{app:uno}, ma che essendo specifica dell'\ac{ObP} viene presentata in quest'appendice. Inoltre vi è anche la presentazione della notazione utilizzata per presentare un framework introdotto in \cite{StefCounterexample14} che facilità la compresione della correttezza e l'implentazione del metodo di gestione del controesempio in \ac{ObP}.
\subsection{Definizioni}
L'\ac{ObP} mantiene un insieme Sp,\textit{prefix-closed}, di \textbf{short prefix} detti anche \textbf{access sequence} ,che sono prefissi . Ogni short prefix identifica unicamente (cioè short prefix diversi identificano stati diversi in \ac{H} e nel target)  gli stati sia nel target A che nell'ipotesi \ac{H}. Ogni stato $q \in Q^{H}$ corrisponde unicamente ad una stringa (lo short prefix) $u \in Sp$, ed è assicurato che $H[u] = q$.  u è detta l'access sequence di q (in \ac{H}), ed è denotata da $\lfloor q \rfloor_{H}$ . Alternativamente quanto detto può essere formulato come $\forall q \in Q^{H} : \text{ \ac{H}}[\lfloor q \rfloor_{H}]=q$. Lo stato iniziale $q_{\epsilon}^{H}$ è lo stato con access sequence $\epsilon$.

Si estende questa notazione a stringhe arbitrarie $w \in \Sigma^{*} : \lfloor w \rfloor_{H} = \lfloor H[w] \rfloor_{H}$ che significa che w raggiunge uno stato in \ac{H} e questo stato ha un access sequence u cui w si associa. Quindi la funzione $\lfloor \cdot \rfloor_{H} : \Sigma^{*} \to Sp$ trasforma stringhe in access sequences.\\
Uno short prefix $u \in Sp$ corrisponde ad uno stato in A, cioè $A[u]$. Ci si riferisce ad $A[Sp]$ come gli stati scoperti (dal \textit{learner}) di A. Gli short prefixes quindi stabiliscono una funzione $f_{Sp}$ che collega stati nell'ipotesi e stati scoperti nel \ac{DFA} target A come segue:
\begin{equation*}
f_{Sp} : Q^{H} \to Q^{A} , f_{Sp}(q)=A[\lfloor q \rfloor_{H}]
\end{equation*}
\begin{comment}
Infine ,detto N l'insieme di nodi di un albero binario $\Upsilon$,si denota con o-child(n) la funzione d'utilità definita attraverso la funzione child(o,n) nella seguente maniera :
\begin{equation*}
\begin{multlined}
o-child=child : \mathbb{B} \times N  \to \{nil\} \cup N, \\
child(o,n)=\begin{cases}
n' & \parbox[t]{.6\textwidth}{se o=1 e n' è il figlio destro di n in $\Upsilon$ oppure o=0 e n' è il figlio sinistro di n in $\Upsilon$}\\
nil & \parbox[t]{.6\textwidth}{se o=1 e n non ha figli destri in $\Upsilon$ oppure o=0 e n non ha figli sinistri in $\Upsilon$}\\
\end{cases}
\end{multlined}
\end{equation*}
\end{comment}

%\thispagestyle{empty}

\subsection[Def. fram.]{Definizioni per il framework}
\label{sub:fra}
\subsubsection{Prefix Transformation}
\textit{Prefix transformation} è una procedura che consente di trasformare un prefisso di un controesempio $w \in \Sigma^{+}$ in un access sequence in Sp.
\begin{definizione*}[Prefix Transformation]
Prefix transformation rispetto ad \ac{H}, $\pi_{H}$ , è definita come segue:
\begin{equation*}
\pi_{H} : \Sigma^{*} \times \mathbb{N} \to \Sigma^{*} \text{ , } \pi_{H}(w,i) = \lfloor w_{[0,i)} \rfloor_{H} \cdot w_{[i,\abs{w})}
\end{equation*}
\end{definizione*}
Si osservi che, $\pi_{H}(w,0)=w \text{ e } \pi_{H}(w,\abs{w})=\lfloor w \rfloor_{H} \in Sp$.

\subsubsection{Altre definizioni}
Sia $w \in \Sigma^{+}$\footnote{In\ac{ObP} $\epsilon$ non può essere un controesempio} un controesempio che differenzia il target A da \ac{H}. Sia $m = \abs{w}$ ed i un indice $0\leq i \leq m$ allora si definisce la funzione $\alpha$ come:
\begin{equation*}
\alpha: [0,m+1) \to \mathbb{B}, \: \alpha(i) = \begin{cases}
1
& \text{se } \lambda^{A}(\pi_{H}(w,i)) = \lambda^{H}(w) \\
0 & \text{altrimenti}\\
\end{cases}\\\\
\end{equation*}
Dalla funzione $\alpha$ può essere ricavato anche la definizione della funzione $\beta$:
\begin{equation*}
\label{equ:beta}
\beta: [0,m) \to \{0,1,2\}, \: \beta(i) = \alpha(i) + \alpha(i+1)
\end{equation*}
\section[Dett. impl. ObP]{Dettagli implementativi dell' ObP}
\label{sec:implobp} 
Vengono ora riportati alcuni dettagli implementativi ritenuti più significativi senza pretesa di esaustività.
Per la memorizzazione della funzione OT (definizione \ref{def:obstable}) di un componente si è utilizzata una map che tipicamente è nella forma chiave-valore dove nella fattispecie la chiave è una stringa (ottenuta dalla concatenazione di un prefisso con un suffisso) ed il valore è l'esito delle \ac{MQ} nel target per quella stringa. Qui si è utilizzato per il valore un doppio campo oltre all'esito della \ac{MQ} costituito da un contatore del numero di volte che quella stringa viene a formarsi in un dato componente (una stessa stringa può essere formata da  coppie prefisso-suffisso diverse). Questo è dovuto al fatto che quando si effettua uno split di un componente alcuni dei prefissi del componente migrano nei prefissi del nuovo componente. Oltre all'insieme dei prefissi del componente splittato (che va decrementato) si può modificare anche il dominio della funzione OT restringendolo. Nel codice si è scelto di eliminare la concatenazione del prefisso che migra nel nuovo componente con l'insieme di suffissi del componente splittato, restringendo il dominio di OT. Tuttavia per garantire la correttezza è necessario appurare se quella concatenazione di quel prefisso con un dato suffisso si venga a formare anche tramite altri prefissi perchè in questo caso l'eliminazione non deve avvenire ed è per questo motivo che si usa un contatore per ogni stringa (per ogni concatenazione) che va incrementato ad ogni inserimento di una stringa (anche una già esistente) e decrementato nel caso suddetto.  Questa scelta è stata fatta nell'ottica di consentire velocemente di appurare se un componente è chiuso e in generale consentire una ricerca molto veloce di un prefisso in un componente. L'alternativa sarebbe stata quella di non utilizzare il contatore e non eliminare mai la stringa formata dalla concatenazione di un prefisso e di un suffisso ma solo il prefisso dal componente splittato. Ciò porterebbe ad una crescita del dominio di ricerca per la funzione OT che degraderebbe le prestazioni della ricerca di un prefisso in un componente (operazione effettuata sovente) anche se potrebbe  comportare una diminuizione del numero delle \ac{MQ} e l'eliminazione dell'\textit{overhead} per tenere aggiornato il contatore: questa prospettata diminuizione delle \ac{MQ} con questa seconda scelta è però possibile soltanto se quando si completano le componenti, ad esempio nella funzione UPDATE-FROM-COUNTEREXAMPLE  prima di effettuare una \ac{MQ} su una stringa x si controlli se per x non si conosca già l'esito della \ac{MQ} perchè contenuto già nel dominio della funzione OT\footnote{più è grande il dominio e maggiore sarà la probabilità di ottenere un \textit{hit} per x} anche se nel caso vi fossero molti \textit{miss} questa politica potrebbe essere addirittura controproducente. Anche quest'ulteriore strategia non è stata implementata ritenendo poco probabile un \textit{hit} (per implementarla basta decommentare un if  in UPDATE-FROM-COUNTEREXAMPLE e aggiungerne un altro nella funzione \textit{sift}). Se si fossero fatte delle scelte opposte a quelle fatte e appena descritte senz altro si sarebbe potuto abbassare ulteriormente il numero di \ac{MQ} ma si è ritenuto che il gioco non vale la candela cioè che il prezzo da pagare in termini di tempo di esecuzione per ottenere ciò è maggiore del beneficio ottenuto.\\
Si sottolinea che diversamente dal metodo OBP-SPLIT (algoritmo \ref{alg:split}) non viene ritornato il nuovo componente ottenuto perchè il chiamante OBP-CLOSEPACK ne è già a conoscenza , quindi il metodo non ritorna niente.\\
Infine si prende in esame l'implementazione del discrimination tree. Per quest ultimo si è scelto di usare un \textit{vector} di nodi e un insieme di archi. Per ogni nodo si memorizza l'etichetta e se è accettante o meno. Gli archi sono un \textit{vector} di array bidimensionali. L' indice di un nodo nel \textit{vector} di nodi è usato per accedere alla posizione nel \textit{vector} di archi che contiene gli indici dei nodi figli (nell'array contenuto nel \textit{vector} di nodi nella posizione individuata dall'indice del nodo). L'utilizzo di un insieme di nodi e di archi è tipico di un grafo piuttosto che di un albero binario. Si è scelta ugualmente questa implementazione essenzialmente per due ragioni:
\begin{itemize}
\item La \textit{Standard Template Library} non mette a disposizione nessuna struttura dati per modellare una albero binario. Esistono delle librerie esterne che mettono a disposizioni un albero n-ario ma l'overhead per gestire n figli ed altre operazioni inutili ai fini dell'\ac{ObP} ha fatto propendere per il declinare il loro utilizzo. Un'altra possibilità sarebbe stata quella d'implementare un albero binario \textit{ad hoc} nella maniera classica cioè tramite i puntatori ma essendo le prestazioni simili a quelle della soluzione adottata e descritta sopra non lo si è fatto. Inoltre si è supposto che chiamare un oggetto di una classe esterna (quella dell'eventuale implementazione dell'albero binario), dato che va fatto molte volte, sarebbe divenuto il costo preponderante. Il vantaggio principale nell'usare l'implementazione di un albero binario con i puntatori per il discrimination tree sta nel risparmio di memoria circa doppia ma comunque sempre lineare nella soluzione proposta, e che non avviene mai la riallocazione (operazione che accade quando le dimensioni del vettore superano la capacità dello stesso).
\item Utilizzare un vettore indicizzato. Questa soluzione è da scartare perchè se il discrimination tree non è bilanciato e presenta un ramo molto più lungo degli altri sarebbe necessario rendere il vettore molto grande. Il vettore può crescere dinamicamente ed è molto più probabile con un vettore indicizzato superare la capacità totale del vettore (se l'inserimento del nodo avviene nello stesso ramo ciò avviene molto velocemente) che verrebbe quindi riallocato e ricopiato frequentemente degradando le prestazioni.
\end{itemize} 
  


%----------------------------------------------------------------------------------------
%	BIBLIOGRAPHY
%----------------------------------------------------------------------------------------
%
%

%\bibliography{Bibliography} % The references (bibliography) information are stored in the file named "Bibliography.bib"
\cleardoublepage
\phantomsection
\hypersetup{linkcolor=cyan} %to have link of pages cyan rather than black
\addcontentsline{toc}{chapter}{\bibname}
\printbibliography



\cleardoublepage
\phantomsection
\addcontentsline{toc}{chapter}{\indexname}
\printindex

\backmatter
\end{document}  