%!TEX encoding = UTF-8 Unicode
%!TEX root = ./../main.tex
%!TEX TS-program = xelatex


\chapter*{Introduzione} % Main chapter title
\label{cap:intro}

%----------------------------------------------------------------------------------------
%	INTERPRETAZIONE ALTO LIVELLO
%----------------------------------------------------------------------------------------
 
L'oggetto di studio di questo elaborato è un algoritmo utilizzato per l'appredimento di linguaggi regolari: l' \ac{ObP}.
L'apprendimento di linguaggi regolari è collocabile nel più ampio tema dell' inferenza grammaticale --- usata in una varietà di campi come pattern recognition, biologia computazionale e elaborazione del linguaggio naturale --- ed è il processo di automaticamente inferire una grammatica esaminando delle stringhe di un linguaggio sconosciuto.
Il modus operandi degli algoritmi d'inferenza grammaticale o regolare e dell'\ac{ObP} è inquadrabile all'interno dell''apprendimento per induzione e per questo è spesso detta \ac{IIR}  (o grammaticale).
Allo stato dell'arte l' \ac{ObP} costituisce il secondo algoritmo di riferimento nell'ambito dell'apprendimento di linguaggi regolari.  L'algoritmo più performante è invece il più recente TTT algorithm \cite{SteffenTTT14}.
Si può trovare una presentazione completa dell' \ac{ObP}  in \cite{Howar12} e una sua implemetazione nella libreria LearnLib\footnote{\href{http://www.learnlib.de/}{http://www.learnlib.de/} Qui \ac{ObP} è menzionato come Discrimination Tree} . Corredata a questa tesina vi è anche l'implementazione dell' \ac{ObP} in C++ , codice che è stato integrato in una libreria preesistente.
Il lavoro qui esposto si divide in quattro parti. Nel primo capitolo si parlerà dell'inferenza induttiva, e riferendosi alla classificazione proposta in \cite{Mic86a}, si inquadrerà questo meccanismo nel complesso meccanismo dell'apprendimento.
Inoltre, dopo avere messo a confronto l'induzione con la deduzione e l'abduzione verranno passati in rassegna le peculiarità del processo induttivo
Nel secondo capitolo si definirà l'inferenza induttiva grammaticale specificando e approfondendo i punti di cui si compone. Poi saranno scandagliati i risultati teorici e i limiti dell'\ac{IIR}.
Nel terzo capitolo saranno presentati brevemente i principali algoritmi dell'\ac{IIR} RPNI e EDSM nel red-blue framework e poi nel paradigma dell'Active Learning sarà introdotto e approfondito L* \cite{Angluin87} che può essere considerato il capostipite dell'\ac{ObP}
Nel quarto capitolo verrà esposto in maniera dettagliata la ratio che muove l'\ac{ObP} le strutture dati e le prestazioni. Inoltre verranno riportate le scelte discostanti dal riferimento principale dell'algoritmo \cite{Howar12} e le motivazioni. Verranno proposti infine i risultati sperimentali per mettere a confronto le prestazioni dell'algoritmo oggetto di studio con L*


