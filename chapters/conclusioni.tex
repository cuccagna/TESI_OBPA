\chapter*{Conclusioni}
\label{cap:con}

Inserire il discorso che è possibile provare ObPA nel caso si approssimano solo le eq. query e le MQ sono esatte. CHe lo scenario cui si è ricondotti qui è molto simile agli algoritmi di passive learning. Gli algoritmi di active learning invece sono più efficienti proprio perchè hanno piu informazioni e qui riducendo a zero l'informazione aggiuntiva riconducendoci a uno scenario operativo degli algoritmi di passive learning ma sempre inseriti in un algoritmo di active learning si hanno riultati non ottimi. Diminuendo solo parzialmente l'informazione (appr. solo l'eq. come fanno molti lavori in letteratura sicuramente si otterrebbero risultati migliori).

Dato che il W-Method ha una complessità esponenziale (Vedi CHow) esistono metodi simili più efficienti come il Wp-Method e HSI method che si può pensare di implementare per provare a risolvere i problemi col W-Method

Inserire il fatto che la versione ONEGlobally di Observation Pack probabilmente non è adattissima in questo algoritmo. Perchè altra versione come OneLocally fa solo uno split e quindi se il controesempio è errato c'è solo uno split errato invece con OneGlobally ci possono essere (è molto probabile) più split e l'ipotesi diverge molto di più in una direzione errata. (Comunque sottolinea che un solo controesempio errato può essere catastrofico, perchè in pratica saranno 2 linguaggi diversi. Se ad esempio avviene uno split laddove non doveva avvenire indietro non si può tornare.

