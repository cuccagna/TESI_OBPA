%!TEX encoding = UTF-8 Unicode
%!TEX root = ./../main.tex
%!TEX TS-program = xelatex

\chapter{Risultati sperimentali} % 5th chapter title
\label{cap:sette}
La valutazione delle \textit{performances} di un algoritmo di \ac{GI} possono essere influenzate sia positivamente che negativamente da alcuni fattori come già evidenziato in \ref{sec:gensam}. Uno di questi fattori è la complessità del \ac{DFA} target con cui si intende\cite{Stamina10}:
\begin{itemize}
\item numero di stati
\item dimensione dell'alfabeto
\item la lunghezza del più lungo dei cammini minimi dallo stato iniziale a qualunque altro stato
\item il numero di transizioni
\item il numero di transizioni totali\footnote{ (escluse quelle che vanno nello stato pozzo, altrimenti sarebbero tutte in egual numero in un \ac{DFA})}
\end{itemize}
Anche la dimensione del \textit{training set} e del \textit{test set} è rilevante e quindi è necessario scandagliare il comportamento dell'algoritmo al variare della dimensionalità di questi parametri. Attenendosi all'impostazione della competizione STAMINA \cite{Stamina10} si è scelto di [INSERIRE DIMENSIONI ALFABETO STATI TRAINING SET]

Nell'effettuazione degli esperimenti è necessario tenerne conto onde evitare dei risultati pessimisticamente o ottimisticamente \textit{biased}. 

todo:
Riorganizzare quanto sopra introducendo prima Tomita.
Descrivi Tomita
Giustifica risultati anche perchè si approssimano anche le MQ.
Fai vedere che comunque esistono dei modelli complessi con cui si hanno buoni risultati (anche se la complessità è tutta da vedeere e non è decisa solo dagli stati dall'alfabeto e dalla profondità, ma dipende anche da quanto sono buoni i campioni estratti col random walk).
Illustra i set di esperimenti e i risultati.